\section{Synthesis with Optimization Objectives}
Using the $\nu Z$ extension of Z3 \cite{nuz3}, we can incorporate 
linear optimization
objectives and MaxSMT in the synthesis algorithm. 
We describe two applications paramount to network management:
traffic engineering and network repair, which can be tackled using
these extensions of SMT solvers.
\subsection{Traffic Engineering}
While the link capacity policies described in \Cref{sec:linkcap} can
be used to perform a strict form of traffic engineering in terms of 
adhering to link bandwidths, support for traffic engineering objectives
like minimizing the max link utilization, minimizing average link utilization
is highly \emph{desirable}. By specifying linear optimization objectives over
SMT formulas, \name can synthesize paths satisfying policies and minimizing
a global traffic engineering objective. 

To perform traffic engineering, link capacities of the network and traffic 
rates of the packet classes are specified as input to \name. The utilization 
of a link $U(sw_1, sw_2)$ is defined as the ratio of total traffic flowing through the link to the 
link capacity, and encoded in Z3 using the theory of rational linear arithmetic as:
\begin{equation}
U(sw_1, sw_2) = \frac{\sum_{\forall pc} IF(Fwd(sw_1,sw_2, pc), T(pc), 0)} {\omega}
\end{equation}
The TE objective of minimizing average link utilization is equivalent to minimizing
the total link utilization (as average = total/constant). Thus, the minimization
objective is:
\begin{equation}
	\text{minimize}\ \sum_{\forall sw_1, sw_2} U(sw_1, sw_2)
\end{equation}
To encode the TE objective of minimizing the maximum link utilization, we define
a variable $maxU$, and constraints to ensure that $maxU$ is greater than or equal to all 
individual link utilisations, and the objective: 
\begin{equation} \label{eq:maxu}
\forall sw_1, sw_2.\ \ maxU \geq U(sw_1, sw_2)
\end{equation} 
\begin{equation}
		\text{minimize}\ maxU
\end{equation}
While $maxU$ can be set to a large value trivially to satisfy \Cref{eq:maxu}
, since the objective is to minimize $maxU$, it will be set to the actual
minimized maximum link utilization. Using an encoding similar to the one presented in this section, \name can be used for other objectives like minimizing total latency and load balancing
traffic across the network middleboxes.

\subsection{Network Repair}
While policy-compliance is a major requirement in a network management system,
another important consideration is datacenters is the occurence of failures (switches, links etc.),
which requires recomputation of paths compliant to the policies for the modified topology. 
Also, the network requirements are constantly \emph{in flux}, and operators have the requirement to 
add new tenants/policies. While a naive approach is to use \name to resynthesize the modified instance,
the new solution may be drastically different from the original configuration, incurring a
large overhead of changing the forwarding rules. This can also be used to
port existing non-policy compliant configurations deployed in a network to
satisfy a set of policies.

We extend \name's synthesis algorithm for performing
minimal network repair, using the MaxSMT extension of Z3. The basic intuition is that
we add the existing configuration as \emph{soft} constraints to the set of \emph{hard} 
policy constraints. The solver would return a solution which \emph{maximizes} the 
number of satisfied soft constraints (the older configuration),
thus resulting in a new policy-compliant forwarding 
configuration with minimal 
number of changes from the older one and reducing the overhead of
updating the forwarding rules in the network.  

We describe one application of network repair, which tries to minimize
the number of switches which needs to be updated. \name is provided the policies
and the current forwarding rules $(\overline{Fwd})$ as input. The policy constraints are added as hard
constraints as described in the earlier sections. To maximize the number of unchanged
switches, we add \emph{soft} constraints for each switch $sw_1$ of uniform weight as follows:
\begin{equation}
\bigvee_{\forall sw_2, pc. (sw_1, sw_2, pc) \in \overline{Fwd} } Fwd(sw_1, sw_2, pc)
\end{equation}
Thus, the new forwarding rules $Fwd$ will maximise the number of unchanged
switch rules in $\overline{Fwd}$, because the constraint will be unsatisfied if a rule on the
switch changes. 


 







 