\begin{table*}
	\begin{center}
		\begin{tabular}{||m{6em} | m{12em} | m{28em} ||} 
			\hline
			Policy &  GPL Syntax & Description \\ 
			\hline\hline
			Reachability & 	$predicate : src >> dst$ & Forwarding Rules for path from switch $src$ to switch $dst$ for packets matching $predicate$ \\
			\hline
			Waypoint & 	$predicate :\newline src >> W >> dst$ & Forwarding Rules for path from switch $src$ to switch $dst$ traversing waypoints $w \in W$ in any order for packets matching $predicate$ \\  
			\hline
			Traffic \newline Isolation & $R1 \ || \ R2$ & Paths of two reachability/waypoint policies $R1$ and $R2$ do not share a link in the same direction \\
			\hline
			Link \newline Capacity & $sw_1 \rightarrow sw_2 : capacity$  & The weights of flows traversing the link $sw_1 \rightarrow sw_2$ does not exceed $capacity$\\
			\hline
			Switch \newline Table Size & $sw : size$ & The number of flows traversing through $sw$ does not exceed $size$ as each flow would require a forwarding rule at $sw$ \\
			\hline\hline
		\end{tabular}
	\end{center}
	\caption{Genesis Policy Support with GPL syntax} \label{tab:policysupport} 
\end{table*}


%\section{Policy Support} \label{sec:policy}
%We design a language GPL (Genesis Policy Language) for network operators to express the desired end-to-end policies in a declarative manner which is interpreted by the Genesis synthesizer to find the forwarding rules for the network topology which enforce the input policies (\cref{fig:arch}). Genesis supports the following policies : 
%% Figure of GPL's syntax
%\begin{enumerate} 
%	\item \textbf{Reachability}: $predicate : src >> dst$ \\
%	This policy specifies the packets satisying $predicate$ have ingress switch $src$ and egress switch $dst$, and requires rules forwarding packets satisfying $predicate$ from $src$ to $dst$. There must be no forwarding loops in the network. 
%	\item \textbf{Waypoint}: $predicate : src >> W >> dst$ \\
%	The waypoint policy is a stronger reachability policy, and specifies that packet satisfying $predicate$ with ingress switch $src$ and egress switch $dst$ must pass through the set of waypoints $W$ in no particular order. The waypoint policy helps operators and tenants to specify the middleboxes the packets must traverse through without worrying about order, or having to use header tags to enforce a particular order \cite{flowtags}. 
%	\item \textbf{Traffic Isolation}:  $R1 \ || \ R2$ \\
%	The traffic isolation policy ensures that the . This policy can be used to provide fairness guarantees, since the paths of $R1$ and $R2$ don't share a link, the bandwidth used by $R1$ will not affect the bandwidth used by $R2$ and vice-versa. The condition of sharing a link in the same direction is due to the fact that links are full-duplex so, traffic flowing in one direction is not affected by the traffic flowing in the other direction.
%	\item \textbf{Security Isolation}: $R1 <> R2$ \\
%	The security isolation policy is stronger than the traffic isolation policy, and ensures that the path of the reachabiltiy/waypoint policies $R1$ and $R2$ do not share a link in both directions for increased security.
%	\item \textbf{}: $sw_1 \rightarrow sw_2 : capacity$ \\
%	The link capacity policy specifies that the capacity for link $sw_1 \rightarrow sw_2$ is $capacity$, and the weights of flows traversing the link in the direction of $sw_2$ do not exceed the capacity of the link.  
%	\item \textbf{Switch Table Size}: $sw : size$ \\
%	The switch table size policy is used to specify the size of the forwarding table of the switch $sw$ and ensures that the number of flows traversing through $sw$ does not exceed $size$ as each flow would require a forwarding rule at the switch.
%\end{enumerate}