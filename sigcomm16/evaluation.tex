\begin{figure*}
	\centering
	\subfloat[Baseline]{\includegraphics[width=0.66\columnwidth]{figures/no-tactic-isolation-plot.eps}}
	\subfloat[No Edge Tactic]{\includegraphics[width=0.66\columnwidth]{figures/no-edge-tactic-isolation-plot.eps}}
	\subfloat[Valley-free Routing Tactic]{\includegraphics[width=0.66\columnwidth]{figures/eacae-isolation-plot.eps}}
	\caption{\label{fig:isolation}
		Total synthesis time (log scale) for isolation workloads over range of packet classes and different tenant-group sizes.}
\end{figure*}

\section{Evaluation}
In this section, we evaluate \Name using 
%\loris{really don't like the word realistic}
enterprise-scale multi-tenant data
center settings. 
Specifically, we ask:

\begin{compactitemize}

\item What is the performance of \Name's baseline synthesis
  algorithms? How does the performance vary with the size of the
  network, number of policies, and the nature of policies in use? (\cref{sec:baselineeval})
%  \loris{I usually write points like the following, but I'm not sure what's the standard
%  in systems conferences.
%We evaluate the performance using X with no optimization and consider
%different sizes of.... This experiment shows that...
%  }

\item To what extent do tactics help improve \Name's synthesis
  performance? Which tactics offer the best improvement? (\cref{sec:tacticeval})

\item To what extent does optimistic synthesis improve \Name's
  performance? When does it lead to degraded synthesis times? (\cref{sec:optimisticeval})

\item Can \Name be used effectively for performing synthesis for
incremental changes?  (\cref{sec:incrementaleval})

\end{compactitemize}
%\loris{if you say this here don't write it in the bullet points}
Our experiment settings have a few thousand servers, tens of
switches, and hierarchical fat-tree network topologies which reflect
a private datacenter. At the
most basic level, our experiments are parameterized by: (a) total size
of the network which we vary between 45 to 180 switches, (b) number of
tenants (1-80), (c) number of packet classes in a tenant (1-10). 

Our primary metric of interest is synthesis time, measured in
seconds. In measuring this, we focus on the time the Z3 solver takes
to solve the constraints\footnote{All experiments were conducted using a
	32-core Intel-Xeon 2.40GHz CPU machine and
	128GB of RAM.}. For evaluating the baseline performance, we impose a
synthetic limit on the path length $\mu$ to be $10$, which is more than adequate 
for a fat-tree topology with three levels. 

\subsection{Baseline Synthesis Performance} \label{sec:baselineeval} 
\noindent {\bf Multi-tenant Isolation}: To evaluate the baseline performance of \Name without tactics, we model a multi-tenant
80-node fat-tree topology network with   
tenant-isolation in \cref{fig:isolation}(a). 
For each workload we have $n$ tenants with group size $g$ which 
is the number of packet classes for each tenant. The x-axis shows the total packet classes $n*g$. 
Packet classes of a tenant are not isolated, while packet classes
of different tenants are isolated. Thus, no two tenants share a link, and can never affect
each other's performance.
We randomly\footnote{
	Smarter placement of tenants could speed-up synthesis as tenant endpoints would
	be located closer to each other. The placement algorithm can be used to develop specialised tactics.}
  place endpoints for the tenants' packet classes, ensuring that not more than 4 tenants share a single edge switch.
   Operators can aggregate a tenant's traffic from one rack to
another to a single reachability policy and establish pathways for communication amongst the multiple
machines in different racks. 

For a fixed group size, we can observe that,
the total synthesis time increases exponentially with increase in packet classes.
 As we decrease the group size,
 we can observe that synthesis time increases greatly for the same number of packet classes.
 In this case, the number of tenants increases, making the problem more constrained 
  due to increased number of isolation policies. 
 Group size = 1 denotes the extreme case where all flows are isolated to one other. 
 
 While we evaluated a multi-tenant isolation setting, there are other scenarios
 which would translate to these workloads. For example, specific flows of 
 tenants would get QoS guarantees and these flows must be isolated from all
 other flows. This would translate to a two-tenant isolation setting. Operators 
 can provide weaker isolation such that two flows must be isolated on only certain 
 links and can share other links. This is an easier problem to tackle than isolation over
 all links, and performance of such scenarios would be better.

\noindent {\bf Effect of Topology Size}:
 To evaluate \Name across increasing topology sizes for isolation workloads, 
we fix the tenant-group size to 5, and for each topology, we maintain
 the ratio of packet classes to number of edge-aggregate links to 0.25. 
We choose this metric because if we kept the number of classes constant,
as topology sizes increases, it is easier to find isolated paths. Thus, by keeping
the number of packet classes proportional to size of the topology, we maintain
the relative difficulty of the workload across topologies. 
We show the average synthesis time per class with
 increasing fat-tree topology sizes in \cref{fig:tactic-topo}. 
 We are able to synthesize forwarding rules for 
 12 tenants with tenant-group size 5 in a 125-node topology in 124 seconds. 
 %Some comment I dont understand.
  We can also observe that average time per flow increases exponentially 
  with larger topologies, 
  thus the synthesis problem is also exponential with number of nodes.
  
  
 \noindent{\bf Isolation with Link Capacity Policies}: 
 \Cref{fig:link-capacity}(orange trace) shows the average synthesis time per flow for the same setting, but
 additionally, there are 10 low-bandwidth links in the network for which the operator
 specifies policies to capacity uasage (all packet classes have uniform capacity). 
Since we use LIA for link capacity constraints, we see an 
increase in average time for synthesis 
when compared to pure isolation which is completely 
encoded using SAT. 

\noindent {\bf Waypoint Policies}: 
To evaluate \Name's performance for ordered set of 
waypoints, we fix the 
number of waypoints (range:1-5) and generate 100 waypoint
policies with different size and permutation of the ordered waypoint 
sets for a 80 node fat-tree topology. Each policy has edge switches as endpoints 
and randomly picked core or aggregate switches for waypoints. The 
synthetic limit $\mu$ on the path length is increased to 15 and no tactics
are used (because it is difficult to reason about the path satisfying a waypoint
policy). The average 
synthesis time for a waypoint policy is reported in \cref{tab:waypointeval}. 
We observe that synthesis time increases exponentially with number of waypoints,
owing to the complexity of the problem. 
\Name can synthesize rules for a path with 3 waypoints in less than a
second on average. These waypoint policies were either completely
ordered or unordered, so had ordered sets of sizes 1 and 2 in different
orders. 

\begin{table}
	\begin{center}
		\begin{tabular}{||P{6em} | P{7em} ||} 
			\hline
			Number of Waypoints & Avg. Synthesis time per Class (s) \\ [0.5ex] 
			\hline\hline
			1 & 0.0347\\ [0.5ex] 
			\hline
			2 & 0.1384\\ [0.5ex] 
			\hline
			3 & 0.9834\\ [0.5ex] 
			\hline
			4 & 15.41\\ [0.5ex] 
			\hline
			5 & 32.93\\ [0.5ex] 
			\hline
		\end{tabular}
	\end{center}
	\caption{Average synthesis time per class for Waypoint Policies for increasing number of waypoints } \label{tab:waypointeval} 
\end{table}
 % Write about waypoints
 \subsection{Tactic Reductions} \label{sec:tacticeval}
 Using the baseline synthesis approach can result in larger synthesis
 times due to the large solution space of forwarding plane configurations. We 
 demonstrate the improvements from using tactics for different tenants and group sizes on a 
 80-node fat-tree topology.
 
 \noindent {\bf No Edge Tactic}: \Cref{fig:isolation}(b) shows the synthesis time for isolation workloads using the no edge tactic 
 ($\neg(e .^* e .^* e)$). We achieve a best-case speedup of 9.5$\times$ over baseline synthesis with this tactic. 
 Using this tactic, \Name can synthesize forwarding rules for 12 tenants with group size 5 in under 200
 seconds.
  
\noindent {\bf Valley-free Routing Tactic}:  
For the same isolation workloads as above, we use the tactic $\neg (e .^5 .^* e)$ $\wedge \neg (e .^* e .^* e)$
 which ensures {\em valley-free routing}, that is paths are of the form $eacae$. 
 The results are shown in \cref{fig:isolation}(c). 
 Using this tactic, \Name synthesizes forwarding rules for each workload in under 20 seconds 
 and can achieve a best-case reduction of 400$\times$ compared to synthesis without tactics. 
 
 \begin{figure}[h]
 	\includegraphics[width=\columnwidth]{figures/tactic-topo.eps}
 	\caption{Average synthesis time per packet class versus fat-tree topology size for isolation workloads 
 		w/o different tactics with the ratio of packet classes to number of edge-aggregate links 0.25.}
 	\label{fig:tactic-topo}
 \end{figure}
 
\noindent {\bf Effect of Topology Size}: 
In \cref{fig:tactic-topo},
 we evaluate the performance of different tactics for different topology sizes. There is a
 significant reduction of synthesis time for each tactic when compared to the baseline synthesis.
 The performance of each tactic is directly related to the reduction of the solution space: a more
 restrictive tactic has lower synthesis times. 
  Using the no edge tactic
 and path length $\leq 7$ tactic, \Name synthesizes forwarding rules for 20 tenants of group-size 5 in 100 seconds in a 180-node
 topology (9$\times$ speedup over synthesis without tactics).
  
  \noindent{\bf Isolation with Link Capacity Policies}: 
 A similar setup with additional link capacity 
 constraints for 10 links is evaluated using the "no edge" tactic, and we get a best-case 14$\times$
 improvement over baseline synthesis, thus demonstrating the usefulness of tactics for 
 integrating different kinds of workloads like isolation and link capacities. 
 Tactics can provide 
 a considerable improvement over the baseline performance as illustrated by these experiments,
 and demonstrate the viability of synthesis approach of \Name to real-world networks.  
 
\begin{figure}[h]
	\includegraphics[width=\columnwidth]{figures/link-capacity-plot.eps}
	\caption{Average synthesis time per packet class versus fat-tree topology size for isolation workloads 
		with different tactics with the ratio of packet classes to number of edge-aggregate links 0.25 and 10 low bandwidth links in the topology 
		have capacity constraint policies.}
	\label{fig:link-capacity}
\end{figure}


\subsection{Optimistic Synthesis Performance} \label{sec:optimisticeval}
To evaluate the heuristic optimistic synthesis procedure, we perform 100 runs of non-optimistic
 synthesis and optimistic synthesis (with the "no edge" tactic) on isolation
 workloads with varying number of tenants and different group sizes 
we used in the \cref{fig:isolation} experiment. We compute the
 speedup $t(non-optimistic)/t(optimistic)$ and plot its cumulative frequency
  distribution in \cref{fig:opt-cdf} to quantify the benefits of optimistic 
  synthesis. For more than 80\% of the
workloads, optimistic offer a better or comparable performance to non-optimistic synthesis, 
achieving a speedup of 2$\times$ nearly 40\% of the workloads. For 20\% of the workloads, optimistic
performs worse than the non-optimistic approach, especially for workloads with tenant group size 1. 
This arises due to greater recovery attempts which results in increased synthesis time. 

\subsection{Incremental Synthesis Performance} \label{sec:incrementaleval}
While network management systems have support for enforcing complex policies, 
a important feature required from these systems is to be enforce incremental changes
like addition of new tenants, changes to policies or topology events like link/switch 
failures. To evaluate our synthesis algorithm for incremental changes, we use 100 runs of 
different isolation workloads of the experiment in  \cref{fig:isolation} and perform synthesis
of $n$ packet classes, and then
add a single policy to a tenant (with isolation to all other tenant packet classes) and
measure the time needed for synthesis of the updated policies. Both runs of synthesis use the
"no edge" tactic for efficiency. We plot a cumulative frequency 
distribution of the ratio of time taken for incremental synthesis time for the $n+1^{th}$ packet class
 versus the synthesis time for $n$ policies  in \cref{fig:incremental-cdf}. For 80 \% of the workloads, the incremental 
synthesis takes less than 10\% of the total synthesis time for $n$ policies, which makes \Name suitable for 
performing incremental changes. But in some cases, the new policies may trigger a re-synthesis 
of the previously synthesized policies, and we can see workloads where the ratio is 1 or greater. This 
is an intrinsic challenge with the problem setup. 
Thus, performing incremental synthesis is a challenging problem and is one of the future directions
to building a robust network management system. \newline
To summarize the main points of the evaluation : 
\begin{compactitemize}
	\item For a tenant-group size of 10 in a 80 node fat-tree topology, the baseline 
	synthesis performance for synthesizing the forwarding rules for 1 to 8 tenants with 
	complete tenant-isolation is in the range of 0.1-2000 seconds. 
	\item Tactics provide a considerable improvement over the baseline synthesis time.
	 Using the no edge tactic, we can synthesize the above mentioned workloads in the 
	 range of 0.1-300 seconds, while the valley-free routing tactics reduces synthesis
	 times to under 12 seconds for these workloads. 
	 \item We can further improve the performance of \Name using optimistic synthesis,
	 which provides a 2.0$\times$ speed-up over non-optimistic synthesis. We can use
	 tactics with optimistic synthesis and tackle real-world applications effectively. 
\end{compactitemize}


%\begin{figure*}
%	\includegraphics[height=7.5cm]{figures/tactic-reduction.png}
%	\caption{Graph used to show the reduction of terms using different tactics w.r.t the total number of terms}
%	\label{fig:tactic-reduction}
%\end{figure*}
%
%\begin{figure}
%	\includegraphics[width=\columnwidth]{figures/isolation-tactics.png}
%	\caption{Graph used to application of tactics for a isolation workload (percentage isolation w.r.t topology 25\%) and different topology sizes. An interesting observation in the graph is that tactics need not always help in reduction of constraints (One of the tactics, not  a very natural one) leads to more time to synthesis without tactics.}
%	\label{fig:isolation-tactics}
%\end{figure}
%
%\subsection{Optimistic Synthesis}

%\begin{figure}
%	\centering
%	\subfloat[Optimistic]{\includegraphics[width=0.5\columnwidth]{figures/opt-cdf.eps}}
%	\subfloat[Incremental]{\includegraphics[width=0.5\columnwidth]{figures/incremental-cdf.eps}}
%	\caption{\label{fig:cdf}
%		Cumulative frequency distribution over 100 runs of isolation workloads for optimistic and incremental synthesis.}
%\end{figure}
\begin{figure}
	\includegraphics[width=\columnwidth]{figures/opt-cdf.eps}
	\caption{Cumulative frequency distribution for speedup achieved by optimistic synthesis.}
	\label{fig:opt-cdf}
\end{figure}
\begin{figure}
	\includegraphics[width=\columnwidth]{figures/incremental-cdf.eps}
	\caption{Cumulative frequency distribution for ratio of incremental synthesis over "one-shot" total synthesis time.}
	\label{fig:incremental-cdf}
\end{figure}


%\caption{Synthesis Time for varying number of reachability (with and without tactics) and waypoints policies for a 45 node fat-tree topology}


