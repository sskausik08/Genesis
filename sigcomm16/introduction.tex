\section{Introduction}

Network operators of multi-tenant datacenters deal with a multitude of
flows traversing large number of heterogeneous devices. With growing
diversity of tenant applications, need for security and compliance,
and the advent of cloud computing, these may be subject to
increasingly complex policies.

Consider a tenant in such a data center. She may desire---in addition
to basic reachability among her hosts or applications---that traffic
attempting to reach some of her applications be examined by a set of
middleboxes for auditing and access controls. Another tenant may
additionally desire (a subset of) her flows to be isolated from
others' flows for strong security or QoS reasons. On top of this,
cloud operators may need to satisfy resource constraints pertaining to
link bandwidths and switch table sizes in order to perform traffic
engineering and network resource management.

Today, configuring network devices to implement these diverse
and complex policies in aggregate is manual, ad-hoc, and error-prone. 
This can lead to mis-configurations and violations of tenant
service-level agreements which can have a severe performance and
security impact.

%%  However, in real-life, the
%% process of policy enforcement by network operators is manual and
%% ad-hoc, leading to violations of service-level agreements and
%% mis-configurations which have severe performance and security
%% impacts. With the boom in cloud services, datacenter networks deal
%% with thousands of flows which are not constant, but in flux, thus,
%% making it difficult to enforce them in an ad-hoc manner.

%% Network operators desire various different end-to-end policies to
%% support in clouds and enterprise networks. Tenants or organisations
%% require support for basic policies like reachability between hosts,
%% and specifying different middlebox policies for certain
%% flows. Operators, on top of that require support for complex policies
%% like traffic isolation between flows to provide fairness and
%% specifying resource constraints like link bandwidth and switch table
%% sizes to perform traffic engineering and network resource management.

Though Software Defined Networking (SDN) has allowed operators
to program networks in a more intuitive manner, many existing SDN
frameworks are too low-level, making it challenging
%for providing the needed functionality}. 
to write controller applications that generate the
data plane to realize the above policies. %% For many
%% of the policies, generating the data plane is an NP-complete problem
%% and requires the design of efficient custom heuristics; combining
%% different policies' heuristics together is non-trivial.
Moreover, for many types of
  policies, generating a data plane 
  that enforces them is a
%  an NP-complete 
computationally hard problem and
this aspect requires the design of efficient custom heuristics.
%; combining  different policies' heuristics together is non-trivial. 
Existing
languages like Frenetic~\cite{frenetic} and Pyretic~\cite{pyretic} are
too constraining: data center operators would ideally want to specify
and realize policies network-wide, whereas these languages focus on
programming individual switch behaviors. %% \aditya{we need to be
% careful
%%   not to bin all SDN languages into this switch-by-switch model}
%% \kausik{Do you want to make changes here?}
Other research on network-wide policy enforcement, e.g.,
Merlin~\cite{merlin} and SIMPLE~\cite{simple}, are tailor-made to
specific sets of policies, and thus, difficult to extend to support
other kinds of policies such as isolation. We review relevant work in
Section~\ref{sec:relatedwork}.

%%  like
%% joint bandwidth provisioning and waypoint routing in Merlin
%% \cite{merlin}, and middlebox policy enforcement in SIMPLE
%% \cite{simple} or FlowTags~\cite{flowtags}. However, these approaches

In this paper, we seek a {\em general} approach that allows a variety
of rich policies to be specified in a simple and declarative way, so
that the complexities of correctly realizing the policies in the data
plane are hidden from the operators. This is an important
initial step toward {\em intent-based networking}~\cite{intent}, where
operators specify {\em what} they want the network to do instead of
worrying about {\em how} the network must be configured.
%their networks. %% To support a cornucopia of policies, an
%% important feature is \emph{generality} of the approach of policy
%% enforcement, so that it can be extended to enforce custom policies
%% required by the operator.
This paper makes a case for using \emph{synthesis} of switch table
forwarding rules as an effective and practical approach to realizing
this vision in the multi-tenant datacenter context.
%% switch
%% table forwarding rules to the solve the problem of policy enforcement
%% by use of off-the-shelf SMT-solvers.

We present \Name, a framework for declaratively specifying and
enforcing complex policies. Enforcement of several of
the policies supported by \Name, e.g., isolation and middlebox
traversal, is NP-complete. Thus, \Name 
encodes the problem of enforcing various
policies as a constraint solving problem and
leverages recent advances in
fast Satisfiability Modulo Theories (SMT) solvers (e.g., Z3~\cite{z3})
to efficiently search for a solution to the constraints. 
The solution is then translated into switch table rules. %% This
%% paper presents Genesis, a
%% network management tool where the network operators can express the
%% network-wide policies in a high-level declarative manner and Genesis
%% will synthesize the lower-level switch forwarding rules for realising
%% these policies, eliminating the need for operators to work on
%% switch-level behaviours.
\Name can be easily extended to incorporate new policies over time. By
leveraging the formal guarantees of constraint solving, \Name
eliminates the room for error in the enforcement of even the most
complex collections of policies.

Unfortunately, 
due to the the large space of forwarding plane configurations,
naively encoding policies using SMT solvers results in
synthesis speeds that don't match the scale of operations of 
today's cloud and enterprise networks.
%enterprise networks today because the space of forwarding plane configurations
%is huge. 
For instance, synthesizing 
rules that enforce simple
reachability among nodes in a 125 switch fat-tree topology can take
several seconds. This raises the need for better search strategies.

To this
end, \Name leverages domain-specific properties to simplify the
constraints handled by the SMT solver.  Specifically, \Name allows the
network operator to write 
restricted forms of regular expressions, called \emph{tactics}, 
that blacklist certain
forwarding plane configurations based on the path patterns   
that are not desired in a datacenter network. 
%\loris{how about: ...based on path patterns that are not desired...}
These tactics are used to discard several constraints, 
acting as a search strategy for the solver.
%\aditya{the previous sentence is vague}
%By identifying a restricted syntax for specifying
%tactics, we 
Tactics can improve the synthesis procedure and achieve
%constraints added to the solver without additional constraints
%required to ensure the solution satisfies the tactic and achieve
a 1.5$\times - $400$\times$ speedup. 

 Secondly, \Name can use a heuristic procedure termed as
 \emph{optimistic} synthesis, that leverages the structure of tenants'
 isolation policy interactions to improve synthesis performance. The
 optimistic procedure partitions the input policies
  into effective components to synthesize these
 components separately and faster than the complete problem (due to
 the exponential time complexity). Optimistic synthesis can halve the
 synthesis time for 40\% of isolation workloads  
 which vary in size and complexity of isolation.
 %\aditya{is this statement correct? what is 40\% of scenarios?}
 %\aditya{the
   %previous sentence is vague} 

%% are
%% huge, and by supporting a set of diverse and complex policies with
%% different search objectives, we require to create a model general and
%% expressive enough to support these. This poses a challenge as to can
%% synthesis performance be improved by leveraging knowledge specific to
%% the problem of policy enforcement in networks?

%We implement \Name using ... We evaluate it using .... Key highlights .... \aditya{all of these are todo}.\kausik{Do we need a para or will the next para suffice?}
\noindent Thus, the main contribution of this paper are: 
\begin{compactitemize}
\item \Name, an extensible declarative framework for describing
  complex policies that can handle new forms of policies in
  multi-tenant settings, such as isolation and unordered middlebox
  traversals.
\item A modular SMT-based algorithm for enforcing \Name policies that
  can easily be extended to new policies.
\item Tactics, which cleverly leverage data center network structure
  to blacklist undesirable path patterns. %% Tactics are specified using
  %% a restricted form of regular expressions and act as a search
  %% strategy for the solver and vastly reduce the number of generated
  %% constraints.
\item An implementation of \Name and its enforcing algorithm using the SMT
		solver Z3 together with an extensive evaluation on different policy workloads, topologies and multi-tenancy settings.
                \item Our experiments emulating a multi-tenant private
                  data center show that \Name can synthesize the
                  forwarding plane for networks containing upto 180
                  switches, 100 flows across 20 tenants, and with complex policies in
                  under 100 seconds using tactics which is a 9$\times$ speedup over baseline synthesis. 
		%to quantify the performance of Genesis. 
\end{compactitemize}
%\aditya{todo}

%% : We present the design and implementation of a network management
%% system with support for a diverse set of complex end-to-end
%% policies like isolation, waypoints and capacity. We designed a
%% novel search strategy using regular expressions to prune the space
%% of forwarding plane configurations by leveraging the network
%% structure to provide properties of the path, especially in
%% datacenter topologies. Lastly, we design a heuristical synthesis
%% routine leveraging the nature of policy interactions to improve
%% synthesis performance.
