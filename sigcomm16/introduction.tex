\section{Introduction}

Network operators of enterprise and cloud datacenter networks deal
with thousands of flow groups traversing large number of heterogeneous
devices. With growing diversity of applications, need for security and
compliance, and the advent of cloud comouting, these flow groups may
be subject to increasingly complex policies. Cloud tenants or
enterprises require basic reachability between hosts/application, and
middlebox traversal for certain flow groups. Operators, on top of that
require support for policies like traffic isolation between flows to
provide security and fairness, and satisfying resource constraints
pertaining link bandwidths and switch table sizes to perform traffic
engineering and network resource management.  However, configuring
network devices to implement these diverse and complex policies is
manual, ad-hoc and error prone today, leading to violations of
service-level agreements and mis-configurations which have severe
performance and security impacts.

%%  However, in real-life, the
%% process of policy enforcement by network operators is manual and
%% ad-hoc, leading to violations of service-level agreements and
%% mis-configurations which have severe performance and security
%% impacts. With the boom in cloud services, datacenter networks deal
%% with thousands of flows which are not constant, but in flux, thus,
%% making it difficult to enforce them in an ad-hoc manner.

%% Network operators desire various different end-to-end policies to
%% support in clouds and enterprise networks. Tenants or organisations
%% require support for basic policies like reachability between hosts,
%% and specifying different middlebox policies for certain
%% flows. Operators, on top of that require support for complex policies
%% like traffic isolation between flows to provide fairness and
%% specifying resource constraints like link bandwidth and switch table
%% sizes to perform traffic engineering and network resource management.

Though Software-defined Networking (SDN) has allowed network operators
to program networks in a more intuitive manner, many of existing SDN
tools/frameworks are too low-level in their functionality. Supporting
the aforementioned types of policies simply using SDN-capable switches
or with existing languages like Frenetic~\cite{frenetic} and
Pyretic~\cite{pyretic} is extremely challenging; operators would
ideally want to specify and realize policies network-wide without
programming individual switch behaviors. \aditya{we need to be careful
  not to bin all SDN languages into this switch-by-switch model}.
There has been research in the field of network-wide policy
enforcement in networks, like bandwidth provisioning in Merlin
\cite{Merlin}, and middlebox policy enforcement in SIMPLE
\cite{simple} or FlowTags~\cite{flowtags}. However, these approaches
are tailor-made to specific policies, and thus, difficult to extend it
to support other kinds of policies.

In this paper, we seek a {\em general} approach, where operators can
specify a variety of custom policies in a simple, declarative way, and
the complexities of correctly realizing the policies in the data plane
are hidden away from them. %% To support a cornucopia of policies, an
%% important feature is \emph{generality} of the approach of policy
%% enforcement, so that it can be extended to enforce custom policies
%% required by the operator.
By designing and implementing a system called \Name, this paper makes
a case for using \emph{efficient synthesis} of switch table forward
rules to realize this vision.
%% switch
%% table forwarding rules to the solve the problem of policy enforcement
%% by use of off-the-shelf SMT-solvers.

We show that enforcement of several of the policies supported by
\Name, specifically isolation, and middlebox traversal is
NP-complete. Thus, \Name leverages recent advances in creating fast
SMT solvers (e.g., Z3~\cite{z3}) to perform synthesis by encoding the
policy enforcement problem to a SMT instance, and using the SMT solver
to search for a solution, which is then translated into switch
rules. %% This paper presents Genesis, a
%% network management tool where the network operators can express the
%% network-wide policies in a high-level declarative manner and Genesis
%% will synthesize the lower-level switch forwarding rules for realising
%% these policies, eliminating the need for operators to work on
%% switch-level behaviours.
By leveraging a formal reasoning technique of SAT/SMT solving, \Name
eliminates the room for error in policy enforcement.

Unfortunately, naive application of SMT solvers results in synthesis
speeds that don't match the scale of operations of cloud and
enterprise networks today. \aditya{give an example as to how slow
  things can get} To overcome this challenge, \Name leverages a few
domain-specific ideas.  Specifically, it uses a novel search strategy
using regular expressions to prune the space of forwarding plane
configurations by leveraging data center network
structure. \aditya{the previous sentence is vague} knowlegde. Second,
\Name uses heuristical synthesis routine that leverages the nature of
policy interactions to improve synthesis performance. \aditya{quickly
  talk about in 1-2 sentences the speedup achieved with these
  optimizations}

%% are
%% huge, and by supporting a set of diverse and complex policies with
%% different search objectives, we require to create a model general and
%% expressive enough to support these. This poses a challenge as to can
%% synthesis performance be improved by leveraging knowledge specific to
%% the problem of policy enforcement in networks?

We implement \Name using ... We evaluate it using .... Key highlights .... \aditya{all of these are todo}.

Thus, the main contribution of this paper are: \aditya{todo}

%% : We present the design and implementation of a network management
%% system with support for a diverse set of complex end-to-end
%% policies like isolation, waypoints and capacity. We designed a
%% novel search strategy using regular expressions to prune the space
%% of forwarding plane configurations by leveraging the network
%% structure to provide properties of the path, especially in
%% datacenter topologies. Lastly, we design a heuristical synthesis
%% routine leveraging the nature of policy interactions to improve
%% synthesis performance.
