\begin{proof}
Let us assume there exists a packet class $pc$ in $PC$ whose path $\pi_{pc} =
(s_{pc}, s_1)$ $ (s_1, s_2)$ 
$\ldots (s_n, d_{pc})$ is not induced by $C$---i.e.,
$\pi_{pc} \not\in \paths^C(pc)$. Let the destination IP of $pc$ be
$\lambda_{pc}$.

Given the routing function $\route^C$ constructed from $SR$ and
$W$ (\secref{sec:routingmodel}), let the first router where routing diverges from $\pi_{pc}$ be $s_p$---i.e.,  
$s_{p+1} \not\in \route^C(s_{p}, \lambda_{pc})$. Let  $r \in \route^C(s_{p}, \lambda_{pc})$.  
\Cref{alg:unsat} on line \ref{line:unique} adds the following
constraint to ensure the sub-path of $\pi_{pc}$ 
from $s_{p}$ to $d_{pc}$ is the unique shortest path: 
\begin{equation} \label{eq:uniq}
\sum_{\mathclap{\substack{(s_{p},s_{p+1})\ldots(s_{n},d_{pc})}}} 
W(s_i, s_j) < W(s_{p}, r) + D(r, d_{pc})
\end{equation}

\Cref{alg:unsat}
only removes a subset of the shortest path 
constraints and 
not the distance constraints (\ref{line:removeconstraint}). We consider 
two cases: when \Cref{alg:unsat}
does not remove Constraint (\ref{eq:uniq}) and when it does. 

\minisection{Case 1:} 
\Cref{alg:unsat} does not remove Constraint (\ref{eq:uniq}). 
Thus, there is no static route $(s_p, s_{p+1})$ for
$\lambda_{pc}$ (line \ref{line:staticroute}), and 
OSPF-based forwarding occurs at $s_{p}$. 
For any path $(r, r_1), (r_1, r_2) \ldots (r_m, d_{pc})$, 
using the distance constraints (\ref{eq:distance}) to
expand $D(r, d_{pc})$, we get the following: 
\[
\sum_{\mathclap{\substack{(s_{p},s_{p+1})\ldots(s_{n},d_{pc})}}} 
W(s_i, s_j) < W(s_{p}, r) + W(r, r_{1}) + D(r_{1}, d_{pc})
\]
\begin{center}
	$\ldots$
\end{center}
\[
\sum_{\mathclap{\substack{(s_{p},s_{p+1})\ldots(s_{n},d_{pc})}}} 
W(s_i, s_j) < W(s_{p}, r) + W(r, r_{1}) + \ldots W(r_{m}, d_{pc}) + D(d_{pc}, d_{pc})
\]
Therefore, the weight of any path through $r$ is greater than 
the path through $s_{p}$ and, since
OSPF sends traffic through the shortest weighted
path, $r$ cannot be in $R^C_{ospf}(s_p, \lambda_{pc})$.
Hence a contradiction. 

\minisection{Case 2:}  
\Cref{alg:unsat} removes Constraint (\ref{eq:uniq}) 
and $SR(s_p, \lambda_{pc}) = \{s_{p+1}\}$ (lines \ref{line:staticroute}-\ref{line:removeconstraint}). 
Thus, $s_{p+1} \in \route^C(s_{p}, \lambda_{pc})$ as static routes 
have the higher priority than OSPF. This contradicts our assumption
that $s_{p+1} \not\in \route^C(s_{p}, \lambda_{pc})$. 
\end{proof}
