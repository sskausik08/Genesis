%!TEX root = paper.tex
\section{Related Work}\label{sec:related}
\minisection{Centralized control}  RCP~\cite{rcp}, supported logically central BGP
configuration. The more recent Fibbing~\cite{fibbing} system provides
centralized control over distributed routing by creating fake nodes
and fake links to steer the traffic in the network through paths that
are not the shortest, similar to how \name uses static routes.  Both
approaches face the issue of forwarding loops in the network during
failures. Fibbing and \name differ in the specific algorithms to solve
the synthesis problem.  \name does offer key advantages: first, by
using static routes for steering, we do not increase the control traffic in
the network, unlike the fake advertisements used in Fibbing. Second,
Fibbing by design targets a single domain and does not take into account
domain decomposition and inter-domain routing. 

%However, these fake advertisements can create
%forwarding loops in the network during failures, and the centralized
%controller has to respond to failures (response to failures can
%precomputed), thus making the controller a central point of
%failure. In contrast, our approach to distributed control plane
%synthesis can provide the same expressive power as Fibbing but avoids
%any centralized component by engineering the control plane parameters
%to match the input specifications.  


\minisection{Configuration synthesis} 
ConfigAssure~\cite{configassure}
uses a combination of logic programming and SAT solving to synthesize
network configurations for security, 
functionality, performance and
reliability requirements specified as constraints; 
but it does not
support any notion of policy- or connectivity-resilience 
or hierarchical domain splitting.  Fortz
et. al~\cite{ospf-te} tackle the problem of optimizing OSPF weights
for performing traffic engineering, but their work is tailor-made
to just this specific problem.
Propane~\cite{propane, propaneat} tackles the specific problem of
synthesizing BGP configurations for concrete and abstract topologies
to ensure network-wide objectives hold even under failures. 
Propane is suited to specify preferences on paths and
peering policies among different autonomous systems. Propane
translates policies to a graph-based intermediate representation,
which is then compiled to device-level BGP configurations. 
\kausik{
The compilation and resilience algorithms for Propane use automata
and are tailored for its underlying technology (BGP configurations with support for 
local preferences, MEDs and communities), in contrast to our LP-based
algorithms which synthesize OSPF and static routing configurations which have
a different forwarding model than BGP.}


SyNET~\cite{synet} tackles network-wide configuration synthesis
(Definition \ref{def:policycompliance}) by
modeling the behavior and interactions of the routing protocols as a
stratified Datalog program, and using SMT to synthesize the Datalog
input such that the fixed point of the Datalog program (which
represents the network's forwarding state after convergence) 
satisfies certain policies or path requirements.  While
both systems can take paths as input requirements, \name uses
a two-phase approach where OSPF weights are synthesized
separately using
LP-solvers---which are faster and parallelizable---rather than directly  
solving the whole configuration synthesis problem using SMT solvers.  
Moreover, SyNET's approach
does not deal with resilience, a key aspect we tackle in this paper,
and SyNET does not attempt to minimize the number of static routes,
which can cause undesirable behaviors like routing loops.  
\kausik{
Finally, SyNET supports routers that run both OSPF and BGP protocols and can be
configured with static routes, which does not fit naturally into a
hierarchical structure where some routers only run OSPF and not
BGP. In contrast, our stochastic MCMC algorithm can look for dynamic domain
assignments with constraints on OSPF domain sizes, lower 
BGP configuration complexity and higher resilience.}

\minisection{Policy languages} 
\kausik{\name uses the \genesis~\cite{genesis} framework
to synthesize paths which comply with reachability, 
waypoint and isolation policies. \genesis is tailored
for software-defined networks comprised of 
OpenFlow~\cite{openflow} switches, whose
forwarding tables can be directly programmed to specify the next-hop
switch for different classes of traffic based on the paths provided
by \genesis. \name's algorithms are catered for legacy control planes 
running OSPF/BGP which lack the programmability of SDNs. \genesis also
provides resilience in a different manner than \name. \genesis precomputes and install backup paths (using isolation policies) 
in the SDN dataplane, while \name synthesizes OSPF
weights and static routes such that, 
the distributed protocols converge to a
policy-compliant path even under failures. Unlike \genesis, which can provide resilience for arbitrary $k$-link failures, \name currently supports only $k=1$ link failure scenarios.}

In the future, \name could use
other policy language frameworks as a front-end (with 
less rich policy support but better
performance). %% Other works on
%% centralized policy enforcement for SDN are Merlin~\cite{merlin} and
%% NetKAT~\cite{netkat}.
In Merlin~\cite{merlin}, data planes that adhere to policies expressed
using regular expressions and min and max
bandwidth guarantees are synthesized using mixed
integer linear programming (ILP). 
NetKAT~\cite{netkat} is a domain-specific language and logic for 
specifying and verifying network packet-processing functions
for SDN, based on Kleene algebra with tests (KAT). NetKAT can  
express certain network-wide policies like reachability and waypoints.
However, both Merlin and NetKAT do not support link-isolation 
to produce edge-disjoint paths, which is needed 
for waypoint-compliance in ~\secref{sec:waypointres}.


%% However, the NetKAT semantics
%% cannot be used to express policies based on hyperproperties
%% ~\cite{hyperproperties}, i.e., 
%% the packet processing function requires multiple packet histories
%% as input. Traffic engineering or isolated paths are policies
%% based on hyperproperties.

%% While Genesis supports richer policies, its SMT-based approach is slower in synthesis. As part of future work, we will consider how

%Fine-grained traffic engineering based on online demand/flow size estimation and 
%rapid rerouting is also crucial for datacenter workloads, and extending \name's
%TE policies to fine-grained timescales is subject of future work.
%Also, the performance
%of SMT solvers with optimization objectives is quite slow, and calls for 
%domain-specific techniques to speed up the synthesis. Also, datacenter
%networks are highly symmetrical, and this symmetry can be leveraged
%to speed up synthesis (similar to the work of Plotkin et. al~\cite{symmetry} to
%speed up network verification using symmetry). The main challenges of
%using symmetry in synthesis is considering two aspects of symmetry: network
%symmetry and policy symmetry. Also, our treatment of resilience synthesis
%is preliminary and future work will be geared towards synthesizing resilient
%forwarding planes incorporating capacity constraints and traffic engineering.
 
