\section{From Policies to Connectivity-Resilient  Configurations}
\label{sec:config-synthesis}

In this section, we present an algorithm
for synthesizing distributed configurations
that adhere to path policies like the one described 
in Figure~\ref{tab:policysupport}.
In general, no existing tool for synthesizing 
distributed configurations
can handle this heterogeneous set of 
these policies---e.g., isolation---and, even less so,
generate resilient configurations.
Due to the complexity of the problem, in this
section, we focus on 
synthesizing  configurations that 
are \emph{policy-compliant} and \emph{connectivity-resilient}.



\subsection{A two-phased approach}
Existing approaches to configuration synthesis
try to directly generate configurations from the given policies~\cite{synet},
but a direct approach leads to scalability issues as well as limitation
in the set of supported policies.
Instead of directly generating configurations from policies, 
we proceed in two-steps:
first we generate policy-compliant paths using the techniques presented
in Section~\ref{sec:genesis}
and then we synthesize distributed configurations that realize these paths
and have high connectivity resilience.

Since we showed how to generate policy-compliant paths in Section~\ref{sec:genesis},
we are left with solving a path-compliance problem.
Formally, we are given a set of paths $\Pi$,
a topology $T=(V,L)$,
a domain-assignment function $\Theta$, 
%and configuration policies $P$,
and we want to find functions $SR$,
$IF$, $LP$, and $W$ such that for 
$C=(\Theta,W,LP,IF,SR)$, and
$\paths^C(PC) = \Pi$.
%., and
%$C$ satisfies $P$.
%Ideally, we want to directly generate configurations that maximize 
%connectivity resilience, but already generating policy-compliant configuration is
%a hard problem.
This problem admits a trivial solution that uses
static routes to exactly enforce the set of input paths. 
This can be done by adding a 
static route to $SR(\lambda)$ for every link $l$ in 
the paths in $\Pi$. 
However, this solution will use more static routes than necessary
and, in the presence of failures, static routes may
create routing loops that make certain connections 
unreachable and reduce connectivity-resilience.
Given the relation between the number of static 
routes and connectivity-resilience,
we set as our goal that of generating configurations 
with fewer static routes.

We show how to synthesize 
path-compliant intra-domain 
configurations---i.e., when there
is only a single domain---and then extend our technique to
 intra-domain configurations---i.e., when
there are multiple domains.

\subsection{Synthesising Path-compliant Intra-domain Configurations} \label{sec:intra-synthesis}
In this section, we show how to synthesize  path-compliant  intra-domain configurations that
use less number of static routes.
To justify the complexity of the techniques 
we are going to present,
we show that
%, even when policy-compliant paths are provided, 
obtaining configurations with minimal numbers of static routes is a 
computationally hard problem.
\begin{theorem}[Hardness of synthesis]
\label{thm:ospfsynth}
Given a
network with a single domain,
and a positive number $C_{sc}$,
the problem generating
a path-compliant configuration with at most $C_{sc}$ static routes
is NP-complete.
\end{theorem}
\iffull
%!TEX root = paper.tex
\begin{proof}
We show that the decision version of the minimum 
vertex cover problem, i.e., there exists a vertex cover
of size $ \leq k$, which is NP-complete, 
reduces to finding a set of static routes 
of size $ \leq k$ \
and OSPF weights for a network with only one domain. 
The latter is also in NP, so after the reduction we 
can conclude that it is also NP-complete.

Let $G = (V,E)$ be an instance of the 
minimum vertex cover problem. A set of
vertices $VC \subseteq V$ is the vertex cover
if $\forall (v_1, v_2) \in E. ~v_1 \in VC \vee v_2 \in VC$. 

We now show how to construct a topology $T=(S,L)$ 
and a corresponding set of paths $\Pi$ that can be enforced 
by configuration C which requires static routes $SR$ such that $|SR| \leq k$  
if and only if the corresponding $VC(SR)$ is a vertex cover of 
the graph $G$ and $|VC(SR)| \leq k$.

\paragraph{Construction.}
For every vertex $v \in V$: add a vertex $r_v$.
For every edge $(u,v) \in E$: add two vertices $s_{uv}$
and $t_{uv}$ to $S$. Add edges
connecting $s_{uv} \rightarrow r_{u}$, $s_{uv} \rightarrow r_{v}$,
$r_{u} \rightarrow t_{uv}$ and $r_{v} \rightarrow t_{uv}$. 
\Cref{fig:rfcomplexity} illustrates this construction.
\begin{figure}[H]
	\centering
	\begin{tikzpicture}[shorten >=0.5pt,node distance=1.5cm,on grid,auto,
	square/.style={regular polygon,regular polygon sides=4}] 
	\node[state] at (0,0) (s)  {$s_{uv}$}; 
	\node[state] at (3,-1) (v)  {$r_v$}; 
	\node[state] at (3,1) (u)  {$r_u$}; 
	\node[state] at (6,0) (t) {$t_{uv}$}; 	
	\node[state] (s1) [below=of s] {$s_{vw}$}; 	
	\node[state] (t1) [below=of t] {$t_{vw}$}; 	
	\path[-] 
	(s) edge  node {} (v)
	edge  node {} (u)
	edge [blue, dashed, bend right=40] node {} (t)
	edge [red, dashed, bend left=40] node {} (t)
	(u) edge node {} (t)
	(v) edge node {} (t)
	edge node {} (t1)
	edge node {} (s1);
	\path[->]
	(s)
	edge [blue, dashed, bend right=40] node {} (t)
	edge [red, dashed, bend left=40] node {} (t);
%	\path[-] (s) edge node[above] {} +(1,-0.1);
%	\path[-] (t1) edge node[above] {} +(-1,-0.1);
	\end{tikzpicture}
	\caption{Construction for reduction to Vertex Cover.}
	\label{fig:rfcomplexity}
\end{figure}
If there is another edge $(v,w) \in E$, then
$s_{vw}$ and $t_{vw}$ have an edge connecting to $r_v$ (shown
in \Cref{fig:rfcomplexity}). 

For each edge $(u,v) \in E$, we add two paths in $\Pi$: 
$s_{uv} \rightarrow r_u \rightarrow t_{uv}$
for destination host $d_u$ and 
$~s_{uv} \rightarrow r_v \rightarrow t_{uv}$ 
for destination host $d_v$.
(dashed paths in \Cref{fig:rfcomplexity}). 

We now prove that if there exists a set of static routes
$SR$ such that $|SR| \leq k$ such that the resulting configurations
are path-compliant for $\Pi$, then there exists a vertex cover $VC$
of $G$ such that $|VC| \leq k$. 

For each static route $sr \in SR$, the static route
has to be placed at the source, either going to $r_u$
or $r_v$ (structure of topology $T$). 
We construct a set $VC(SR)$ by adding the vertex $v$
based on the endpoints of each static route $sr \in SR$.
To show that $VC(SR)$ is a vertex cover of $G$, we first
prove \Cref{lemma:diamond}.

\begin{lemma} \label{lemma:diamond}
	 For each diamond formed by the input paths, atleast 1 
	 static route on one of the edges of the paths of the diamond 
	 is required to find a valid solution to the
	 OSPF edge weights.  
\end{lemma}

\begin{proof}
Given two paths $\pi_1$ and $\pi_2$ for destinations 
$d_u$ and $d_v$, we define these paths form a diamond
if these paths intersect at two routers ($s_{uv}$ and $t_{uv}$) 
without any common router in between. 
Consider the following diamond % in \Cref{fig:diamond}
constructed by paths $\pi_1$: $s_{uv} \rightarrow r_u \rightarrow t_{uv}$ 
for destination $d_u$ and $\pi_2$: $s_{uv} \rightarrow r_v \rightarrow t_{uv}$ 
for destination $d_v$. Let us assume there exists a solution 
for the OSPF edge weights without any static routes. 

We add the following inequality 
to make $\pi_1$ is the shortest 
path from $s_{uv}$ to $t_{uv}$ by
ensuring
 $\pi_1$ is shorter than the
path from $s_{uv}$ to $t_{uv}$ via $r_v$: 
\begin{equation} \label{eq:diamond1}
	W(s_{uv},r_u) + W(r_u, t_{uv}) < W(s_{uv}, r_v) + W(r_v,t_{uv})
\end{equation}
Since $\pi_2$ is also the shortest path from $s_{uv}$ 
to $t_{uv}$, the linear inequality added is:
\begin{equation}  \label{eq:diamond2}
W(s_{uv},r_v) + W(r_v, t) < W(s_{uv}, r_u) + W(r_u,t_{uv})
\end{equation}
Since there are no static routes on the edges
of $\pi_1$ and $\pi_2$, none of the above equations are 
eliminated. 
Adding equations \ref{eq:diamond1} and  \ref{eq:diamond2} 
yields the inequality $0 < 0$, which is inconsistent 
and therefore, no solution to 
the edge weights exists for this system of equations, 
which contradicts our assumption. Therefore,
for each diamond formed by the input paths, atleast 1 
static route on one of the edges of the paths of the diamond 
is required to find a valid solution to the
OSPF edge weights.  
\end{proof}

For every edge $(u,v) \in E$, the constructed paths from 
$s_{uv}$ to $t_{uv}$ form a diamond. Thus, by Lemma~\ref{lemma:diamond}, 
the diamond created by the paths corresponding to each edge in $G$ 
requires atleast one static route to eliminate
the inconsistency caused by the diamond. If a static route's endpoints
contains $r_u$, we put $u$ in $VC(SR)$ and similarily for $r_v$. 
Edge $(u,v)$ is covered since atleast one static route is added,
thus, atleast one of $\{u,v\}$ is in $VC(SR)$.  
Thus, if $SR$ eliminates all diamond inconsistencies
to find a solution to the OSPF weights, the corresponding set
$VC(SR)$ covers all edges in $E$. Therefore, $VC(SR)$ is a vertex
cover. 

Thus, by finding a set of static routes $SR$ such that $|SR| \leq k$
such that all the diamond inconsistencies are eliminated, and there
exists OSPF weights $W$ such that the configurations forward traffic
along $\Pi$, we can find a vertex cover $VC$ for graph $G$ such that
$|VC| \leq k$. 

This transformation is polynomial, the constructed 
network topology $T$ has $|V| + 2|E|$ nodes, 
$4|E|$ links and $2|E|$ paths. Therefore, OSPF
configuration synthesis with number of static routes $\leq k$ is
NP-complete. Thus, OSPF synthesis with minimal number of 
static routes is NP-hard. 
\end{proof}

%\else
%The reduction is from the vertex cover problem.
\fi
Since the problem of optimally placing 
static routes is computationally hard, 
%(Theorem~\ref{thm:ospfsynth}), 
we propose a non-optimal greedy strategy 
that works well in practice.
We first show how to solve the problem when 
there exists a solution with no static routes
and then extend our technique to greedily 
add static routes when needed. We use 
efficient off-the-shelf linear programming (LP) 
solvers for solving the system of constraints 
which are presented in the following sections. 
			
\subsubsection{Intra-domain Synthesis without Static Routes} \label{sec:ospf}
 
In the absence of static routes,
 OSPF routers use edge weights
 to choose the
 shortest weighted path for each pair of endpoints. 
Our goal is to synthesize the weight function $W$.
Consider the example in \Cref{fig:edgeweightexample}, 
if $\Pi$ contains the 
 path $\pi=s\rightarrow r_1 \rightarrow t$ for
 destination IP $\lambda$, \name will assign
 edge weights such that $\pi$ has
 smaller weight than the path $\pi'=s \rightarrow t$ ---e.g., $W(\pi)=1+2$
  and $W(\pi')=5$. 
 
\begin{wrapfigure}{r}{0.33\columnwidth}
	\resizebox {0.33\columnwidth} {!} {
		\begin{tikzpicture}[shorten >=0.5pt,node distance=,on grid,auto,
		square/.style={regular polygon,regular polygon sides=4}] 
		\node[state] at (0,0) (s)  {$s$}; 
		\node[state] at (1.8,1) (v1)  {$r_1$}; 
		\node[state] at (3.6, 0)(t) {$t$};
		\node[state, rectangle] at (5, 0) (d1) {$\lambda$};
		\path[->] 
		(s) edge [red] node [black] {1} (v1)
		edge  node {5} (t)
		(v1) edge [red] node [black] {2} (t)
		(t) edge [red, dashed] node {} (d1);
		\end{tikzpicture}
	}
	\compactcaption{OSPF edge weights
		such that the routers forward traffic along
		the colored path.}
	\label{fig:edgeweightexample}
\end{wrapfigure}
The problem of synthesizing the weight function $W$ that
realizes an input set of paths $\Pi$ is
a
variation of the so-called {\em inverse shortest path} 
problem~\cite{isp}. 
For a destination IP $\lambda$, we call $\xi_\lambda$ 
the directed tree of $T$ 
obtained by only keeping the nodes and edges 
that are traversed by paths in $\Pi$ for 
destination IP $\lambda$; the root of the tree
is the destination router connected to $\lambda$. 
 This destination tree
 property is due to the modifications to \genesis
 to support OSPF's destination-based forwarding. We
 define $\Delta=\{\xi_\lambda\mid \lambda \in \Lambda\}$ to be 
the set of all destination trees. 

Given a set of input paths $\Pi$, \name 
generates a set of linear constraints to 
find proper weights $W(l)$ 
for all $l \in L$.
The constraints use the variable $W_{r_1}^{r_2}$
to denote the weight of the edge $(r_1, r_2)$, and the variable
$D_{r_1}^{r_2}$ to denote the 
shortest distance from $r_1$ to $r_2$.
We add the equation $D_{s}^{s} = 0$ 
for every $s\in S$ to denote that the distance
from a node to itself is $0$.
\Cref{eq:distance} guarantees that $D_{s}^{t}$ is smaller or equal to
the shortest distance from $s$ to $t$.
\begin{equation} \label{eq:distance}
\forall s, t. ~\forall r \in N(s).~
D_s^t \leq W_s^r + D_r^t
\end{equation}
Intuitively, the shortest path connecting $s$ to $t$
must traverse through one of neighbor routers of $s$,
and thus, distance can be defined inductively as the 
shortest among distances from the neighbors. 

For each destination tree $\xi_\lambda\in\Delta$, we add equations to ensure 
that the input paths with destination $\lambda$ are the shortest ones.
Notice that, if a path $\pi$
is the shortest path between its endpoints, every 
subpath of $\pi$ also has to be the shortest path between its endpoints.
For each tree $\xi_\lambda$, we define the neighbour
 $N_{\xi_\lambda}(s)$ to denote the 
next-hop neighbour of router $s$ in the destination tree $\xi_\lambda$---i.e., the parent
of $s$ in the tree or no node if $s$ is the root of the tree. We denote the
router directly connected to $\lambda$ (root of the tree $\xi_\lambda$) by $R_\lambda$.

The following equations ensures that, for any node $s$ in 
$\xi_\lambda$, 
the path $l_0\cdots l_k$ from $s$ to $R_\lambda$ 
in $\xi_\lambda$ is the 
\emph{unique shortest path}\footnote{
If two paths have the same weight, then OSPF will 
split the traffic among the two paths.
} from $s$ to $R_\lambda$ in $T$.
\begin{equation} \label{eq:unique}
\forall r \in N(s) \setminus N_{\xi_\lambda}(s).~~
\sum_{\mathclap{\substack{l_i=(r_1,r_2)}}} 
W_{r_1}^{r_2} < W_{s}^{r}+ D_{r}^{R_\lambda}
\end{equation}
\Cref{eq:unique} guarantees that 
the sum of the weights belonging to the path 
from $s$ to $R_\lambda$ in $\xi_\lambda$ 
is strictly smaller than 
any path that goes to $R_\lambda$ via 
a node $n'$ that is a neighbour of $s$ in $T$ but not 
the next-hop in $\xi_\lambda$. Note that,
while $D_{r}^{R_\lambda}$ can be smaller 
than the actual shortest
distance from $r$ to $R_\lambda$, 
it is used to upper bound the sum of edge weights 
in $\xi_\lambda$, and  
thus, the synthesized edge weights will ensure 
paths in $\xi_\lambda$ are the shortest. 

%\begin{figure}[!t]
%	\centering
%	\subfloat[Edge Weights]{
%		\raisebox{0.5cm}{\resizebox {0.33\columnwidth} {!} {
%				\begin{tikzpicture}[shorten >=0.5pt,node distance=,on grid,auto,
%				square/.style={regular polygon,regular polygon sides=4}] 
%				\node[state] at (0,0) (s)  {$s$}; 
%				\node[state] at (1.8,1) (v1)  {$r_1$}; 
%				\node[state] at (3.6, 0)(t) {$t$};
%				\node[state, rectangle] at (5, 0) (d1) {$\lambda$};
%				\path[->] 
%				(s) edge [red] node [black] {1} (v1)
%				edge  node {5} (t)
%				(v1) edge [red] node [black] {2} (t)
%				(t) edge [red, dashed] node {} (d1);
%				\end{tikzpicture}
%			}}}
%			\subfloat[Static Routes]{
%				\resizebox {0.33\columnwidth} {!} {
%					\begin{tikzpicture}[shorten >=0.5pt,node distance=,on grid,auto,
%					square/.style={regular polygon,regular polygon sides=4}] 
%					\node[state] at (0,0) (s)  {$s$}; 
%					\node[state] at (2, 1) (v1)  {$r_1$}; 
%					\node[state] at (4, 0)(t) {$t$};
%					\node[state, rectangle] at (5.5, 0.75) (d1) {$\lambda_1$};
%					\node[state, rectangle] at (5.5, -0.75) (d2) {$\lambda_2$};
%					\path[->] 
%					(s) edge [red] node [black] {1} (v1)
%					edge [blue] node [above, black] {5} node [below, black] {$sr((s,t),\lambda_2)$} (t)
%					(v1) edge [red] node [black] {2} (t)
%					(t) edge [red, dashed] node {} (d1)
%					(t) edge [blue, dashed] node {} (d2);
%					\end{tikzpicture}
%				}}
%						\compactcaption{OSPF edge weights and static routes
%							such that the routers forward traffic along
%							the colored paths.}
%						\label{fig:ospfexample}
%					\end{figure}
%					
\subsubsection{Intra-domain Synthesis with Static Routes} \label{sec:ospfsr}

If the system of equations presented in Section~\ref{sec:ospf} admits a solution, 
the values of the $W_s^t$ variables are the weights we are trying to synthesize,
otherwise the intra-domain synthesis problem 
cannot be solved without using static routes.

\begin{wrapfigure}{r}{0.33\columnwidth}
	\resizebox {0.33\columnwidth} {!} {
			\begin{tikzpicture}[shorten >=0.5pt,node distance=,on grid,auto,
			square/.style={regular polygon,regular polygon sides=4}] 
			\node[state] at (0,0) (s)  {$s$}; 
			\node[state] at (2, 1) (v1)  {$r_1$}; 
			\node[state] at (4, 0)(t) {$t$};
			\node[state, rectangle] at (5.5, 0.75) (d1) {$\lambda_1$};
			\node[state, rectangle] at (5.5, -0.75) (d2) {$\lambda_2$};
			\path[->] 
			(s) edge [red] node [black] {1} (v1)
			edge [blue] node [above, black] {5} node [below, black] {$sr((s,t),\lambda_2)$} (t)
			(v1) edge [red] node [black] {2} (t)
			(t) edge [red, dashed] node {} (d1)
			(t) edge [blue, dashed] node {} (d2);
			\end{tikzpicture}
	}
	\compactcaption{An example of why static routes are needed.}
	\label{fig:srexample}
\end{wrapfigure}
Consider the example in \Cref{fig:srexample}
where the two paths between $s$ and $t$ are used by 
destinations $\lambda_1$ and $\lambda_2$. This
set of input paths does not admit a solution. 
This is because, for $\lambda_1$, 
$W(s \rightarrow r_1 \rightarrow t) < W(s \rightarrow t)$ 
and vice-versa for $\lambda_2$. 
To steer the traffic for a particular destination
to a next-hop router not on the OSPF shortest path, we can 
install a static route. For example, if we install
a static route at router $s$ to forward $\lambda_2$ 
traffic to router $t$, this route will have a higher
preference compared to the OSPF route
through $r_1$ (static routes by default have the
lowest administrative distance~\cite{admindistance}).
Since there are no static routes for
$\lambda_1$, the traffic to $\lambda_1$ will be forwarded to $t$
through $r_1$ using the OSPF shortest path.



Since the problem of minimizing the number of static routes
is NP-complete (Theorem~\ref{thm:ospfsynth}), we opt to place static routes using a best-effort approach.
Our algorithm starts by trying to synthesize a solution
that does not use static routes using the equations proposed in \secref{sec:ospf}. 
In the case of a failure, the algorithm uses the ``proof of unsatisfiability''---i.e., the unsatisfiable core---generated by 
the constraint solver 
to greedily add a small set of static routes. 
Certain equations are eliminated 
to model the added static routes 
and the approach is repeated until a solution is found.
%We now propose a
%technique for choosing the static routes based 
%on the unsatisfiable core. 

%\kausik{Equations with static routes:
%%\loris{why do we need to modify \eqref{eq:dist}? Can't we just add $D_s^t(d)\geq D_s^t$ for
%%all $d$}
%We assume we are given a route-filter function $RF$ and 
%use $s\rightarrow_d^* t$ to denote that $s$ can reach $t$
%without using any edge in $RF(d)$.
%We use the variable $D_{r_1}^{r_2}(d)$ to denote the shortest distance from $r_1$ to $r_2$
%using only edges that are not filtered for destination $d$.
%We can revise equation \eqref{eq:dist}   to restrict the values of 
%each variable $D_{r_1}^{r_2}(d)$
%by  ignoring all the filtered edges---i.e., only consider a switch $r\in N(s)$ if
%$(s,r)$ is not in $RF(d)$. 
%In summary, we need to add one modified version of equation \eqref{eq:dist} 
%and equation  \eqref{eq:unique} for each destination $d$.
%If the encoding without route filters produces $n$ equations, this
%encoding produces $kn$ equations where $k$ is the total number of destinations.  
%To mitigate this problem, we observe that the shortest distance $D_s^t(d)$ between two
%nodes $s$ and $t$ without using edges filtered for $d$ is
%greater or equal to the distance $D_s^t$ obtained without route filters.  
%We use this property to simplify the
%encoding by only computing $D_s^t$ and by replacing each instance of
%$D_s^t(d)$ with $D_s^t$ in the equations.  It is easy to see that 
%if the set of constraints with variables $D_s^t(d)$ admits a solution,
%the corresponding set of constraints with variables $D_s^t$ 
%also admits a solution (because $D_s^t\leq
%D_s^t(d)$).  However, the reverse is not true and the set of
%simplified equations can be unsatisfiable in cases in which the
%original set is satisfiable, causing addition of unnecessary filters.
%Since our algorithm is already greedy and it does not try to compute the optimal 
%number of route filters, this is not a problem.}


%\minisection{Adding static routes using unsatisfiable cores}
To choose what static route to place, we use
LP-solvers' ability to produce an
unsatisfiable core, also called IIS (Irreducible Inconsistent Subsystem)~\cite{chinneck2007feasibility}. 
Intuitively, an IIS is a subset of the input constraints such that,
if all constraints except those in the IIS are removed, the resulting set of
linear equations is still unsatisfiable. Moreover, the set is irreducible---i.e., removing 
any one constraint from the IIS produces a satisfiable set of constraints. 
In our case, an IIS cannot consist of only 
constraints from \Cref{eq:distance} as these constraints
admit a trivial solution with all variables set to 0. 
Therefore, an IIS must contain some constraint of the form
given in  Equations~(\ref{eq:unique})
that was added in to reason about a destination tree $\xi_\lambda$. 
\[
\sum_{\mathclap{\substack{l_i=(r_1,r_2)}}} 
W_{r_1}^{r_2} < W_s^r + D_r^{R_\lambda}  
\]	
This constraint is added to ensure 
that router $s$ forwards to traffic to  
$N_{\xi_\lambda}$ and not $r$, based on OSPF weights,
but the path through $r$ is causing the unsatisfiablity. 
To remove this inequality from the set of constraints, 
we add the static route $(s,N_{\xi_\lambda})$ to $SR(\lambda)$.
As a result of adding of static route, \name removes 
all constraints of \Cref{eq:unique} as router $s$ 
will forward $\lambda$ traffic  to 
next-hop $N_{\xi_\lambda}$ irrespective of
the OSPF distances to the destination; the 
unsatisfiability caused by this IIS is eliminated. 
However, the new set of
constraints may still be unsatisfiable due to other IISes.
We repeat the procedure and add static routes
until we obtain a satisfiable set of
constraints. 
In each iteration, there can be more than one way to place a static route and
\name picks one randomly. 

