\section{OSPF Extensions}
\subsection{Waypoint Policies}
When a tenant specifies a waypoint policy, for e.g., traffic from
subnet A to subnet B must traverse through a firewall, they are not
concerned with the exact path through the network except that the 
path goes through a firewall instance (which could be replicated for
redundancy and load-balancing purposes). \kausik{improve}
Thus, we can relax \Cref{eq:uniq1}
to ensure that traffic for a destination IP goes 
through the waypoint rather than 
enforcing the particular path provided as Genesis 
to be the shortest path. 

-> All traffic to a destination IP can pass through any of the
waypoint instances as defined in the policy.

\subsubsection{Non-waypoint Distances}
For a waypoint set $W \subseteq V$, we define $D_s^t(W)$ to be the 
distance between $s$ and $t$ such that the path does not
 traverse through any waypoint $w \in W$. For a topology 
 $T = (V,L)$, the following distance equations (similar to 
 \Cref{eq:dist}) for the non-waypoint distances capture the
 non-waypoint distances for a waypoint set $W$.
\begin{equation} \label{eq:dist}
\forall s, t \in V \setminus W. ~\forall r \in N(s) \setminus W.~
D_s^t(W) \leq W_s^r + D_r^t(W)
\end{equation}
Similar to $D_s^t$ variables, $D_s^t(W)$ is upper bounded by the
actual shortest non waypoint distance from $s$ to $t$.

\subsubsection{Waypoint Compliance Constraints}

\begin{equation} \label{eq:waypoint}
\forall r \in N(s) \setminus N_{\xi_\lambda}(s).~~
W_{s}^{r}+ D_{r}^{R_\lambda}(W) > \sum_{\mathclap{\substack{l_i=(r_1,r_2)}}} 
W_{r_1}^{r_2}
\end{equation}