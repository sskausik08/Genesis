\section{OSPF Extensions}
\subsection{Waypoint Policies}
When a tenant specifies a waypoint policy, for e.g., traffic from
subnet A to subnet B must traverse through a firewall, they are not
concerned with the exact path through the network except that the 
path goes through a firewall instance (which could be replicated for
redundancy and load-balancing purposes). \kausik{improve}
Thus, we can relax \Cref{eq:uniq1}
to ensure that traffic for a destination IP goes 
through the waypoint rather than 
enforcing the particular path provided as Genesis 
to be the shortest path. 

-> All traffic to a destination IP can pass through any of the
waypoint instances as defined in the policy.

\subsubsection{Non-waypoint Distances}
For a waypoint set $W \subseteq V$, we define $D_s^t(W)$ to be the 
distance between $s$ and $t$ such that the path does not
 traverse through any waypoint $w \in W$. For a topology 
 $T = (V,L)$, the following distance equations (similar to 
 \Cref{eq:dist}) for the non-waypoint distances capture the
 non-waypoint distances for a waypoint set $W$.
\begin{equation} \label{eq:dist}
\forall s, t \in V \setminus W. ~\forall r \in N(s) \setminus W.~
D_s^t(W) \leq W_s^r + D_r^t(W)
\end{equation}
Similar to $D_s^t$ variables, $D_s^t(W)$ is upper bounded by the
actual shortest non-waypoint distance from $s$ to $t$.

\subsubsection{Waypoint Compliance Constraints}

\begin{equation} \label{eq:waypoint}
\forall r \in N(s) \setminus N_{\xi_\lambda}(s).~~ \sum_{\mathclap{\substack{l_i=(r_1,r_2)}}} 
W_{r_1}^{r_2} < 
W_{s}^{r}+ D_{r}^{R_\lambda}(W) 
\end{equation}

\subsubsection{Routing Loop Avoidance Constraints} \label{sec:loopavoidance}
Consider the example configuration shown in \Cref{fig:ospfexample}(c). 
Traffic for $\lambda$ at router $s$ is forwarded to $r_1$ because of the
static route at $s$. At router $r_1$, the shortest OSPF route to
$\lambda$ is $r_1 \rightarrow s \rightarrow t$ with weight 6 (compared 
to $r_1 \rightarrow t$ of weight 50). Thus, $r_1$ will send the 
packet back to $s$, and thus, causing a routing loop as $s$ will send
it back to $r_1$ and back and forth till the \emph{ttl} (time to live) of the
packet expires. 

To prevent a routing loop, the shortest
path from $sr_2$ to $R_\lambda$ must not traverse through an upstream router 
$ur$ (otherwise a routing loop is formed). 
This is expressed by the following constraint: 
\[
\sum_{\mathclap{\substack{l_i=(r_1,r_2)}}} 
W_{r_1}^{r_2} < D^{ur}_{sr_2} + D_{ur}^{R_\lambda} 
\]		
