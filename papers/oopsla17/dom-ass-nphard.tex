\begin{proof}
We show the $k$-graph coloring problem, which is NP-complete
reduces to finding a domain assignment such that the number
of filters is 0. 
The latter is also in NP, so after the reduction we 
can conclude that it is also NP-complete.

Let $G = (V,E)$ be an instance of the 
$k$-graph coloring problem. Formally, 
we need to find a coloring $C: V \mapsto \{1,2,\ldots k\}$
such that for $\forall (u, v) \in E. C(u) 
\not= C(v)$.

\begin{figure}[H]
	\centering
	\begin{tikzpicture}[shorten >=0.5pt,node distance=1.5cm,on grid,auto,
	square/.style={regular polygon,regular polygon sides=4}] 
	\node[state] at (0,0) (s)  {$r_u$}; 
	\node[state] at (2,-0.75) (v1)  {$d_1$}; 
	\node[state] at (2,0.75) (u1)  {$d_2$};
	\node[state] at (4,0) (t) {$r_v$};
	\path[->] 
	(s) edge [red] node {} (v1)
	edge  [blue]  node {} (u1)
	(u1) edge [blue] node {} (t)
	(v1) edge [red] node {} (t);
	\path[-] 
	(s) edge node {} (t);
	\end{tikzpicture}
	\caption{Construction}
	\label{fig:domasscomplexity}
\end{figure}

Let us consider a network topology $T = (R,L)$. 
For each $v \in V$, add a router $r_v \in R$. To
ensure any domain assignment to the $r_v$'s is valid, we 
add edges to connect every router. 

For every edge $(u, v) \in E$, we construct a
$(u, v, \lambda_1, \lambda_2)$-$diamond$ (Section 6.2) %TODO
as illustrated in \Cref{fig:domasscomplexity}. For every
edge $(u, v) \in E$, we add $d_1, d_2$ to R, and add
paths: $u \rightarrow d_1 \rightarrow v$ and 
 $u \rightarrow d_2 \rightarrow v$ to $\Pi$. 
 
Suppose we find a domain assignment $\Theta$ 
with $k$ domains such that number of route filters is zero. 
Since, filter count is 0, for two routers $r_u, r_v$ which 
have a diamond, $\Theta(r_u) \not= \Theta(r_v)$. This is
because, if $\Theta(r_u) = \Theta(r_v)$, then atleast 
one route-filter would be required to eliminate the diamond. 
Therefore, for every edge in $(u,v) \in E$, 
the routers $r_u$ and $r_v$ belong to 
different domains, therefore $u$ and $v$ have different colors. 
There, the $k$-graph coloring problem reduces to finding 
a domain assignment with zero filters.  
Therefore, the policy-compliance synthesis problem is NP-complete.
\end{proof}



