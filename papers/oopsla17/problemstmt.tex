\section{Routing Model and Problem Definition}
In this section, we formally define the problem addressed in this
paper.  We first define how different
routing protocols are represented.  In particular, we model static
routes, OSPF shortest-path routing, and BGP preference-based routing.
Then, we formally define when a certain configuration meets a given
set of policies and present the configuration synthesis problems
we tackle.

\subsection{Routing Model}

\minisection{Preliminaries}
We represent the physical router topology as a directed graph $T=(V, L)$,
where $V$ is the set of routers and $L\subseteq V\times V$ is the set of links. 
Throughout the paper we assume $T$ is fixed.
We use the neighbour function $N(s) = \{s'\ | \ (s,s') \in L \}$ to denote 
the set of neighbour routers of $s$. 
We define $\Lambda$ to denote the set of destination IP subnets;
distributed protocols make forwarding decisions based on the 
destination address/subnet (see Figure~\ref{fig:ospfexample}).

A path $\pi = (u_1,v_1) (u_2, v_2) \cdots (u_n, v_n) \in L^*$ is a loop-free valid path if
a  vertex is not visited more than once---i.e.,
$u_1\neq v_n\wedge\forall i,j \leq n. 
~i \not= j \rightarrow u_i \not= u_j$---and adjacent links in the
path share a router---i.e., $\forall i < n. ~v_i = u_{i+1}$.
Sometimes we represent the path $\pi$ as $u_1\rightarrow u_2 \rightarrow  \cdots \rightarrow v_n$.
Given two nodes $s$ and $t$, we use $\Pi_s^t$ to denote the set of all paths
with source $s$ and target $t$---i.e., $u_1=s$ and  $v_n=t$. 
We write $l \in \pi$ when the path $\pi$ contains the link $l$. 

To model hierarchical control planes in which
 the network is divided into continuous domains,
we define a router domain assignment function
$\Theta: V \mapsto \nat$ which maps each router to a domain 
(denoted by a natural number).
Each domain uses OSPF as the intra-domain routing protocol
and BGP as the inter-domain routing protocol. 


The OSPF configurations are expressed 
using the edge weights function $W$. 
For BGP configurations: 
we have the local preference function $LP$ 
and the iBGP filter function $IF$. Static 
routes are expressed using the $SR$ function.
We express the complete network configuration $C$
as a tuple $(\Theta,W,LP,IF,SR)$.  In the 
rest of the section, we describe each component 
of the configuration $C$ and the routing function 
$\route^C$ for the network given the configuration $C$. 
\kausik{Remove this simplifying assumption}
We assume the following  
simplifying property for $C$: for any two 
routers $r_1$ and $r_2$,
traffic will be forwarded from $r_1$ to $r_2$ 
along a single path. 
When synthesizing $C$, we will ensure that this property is satisfied.


\minisection{Static Routes} Static routing refers to a router using a
manually-configured routing entry, rather than information from
dynamic protocols like OSPF or BGP.  Static routing is useful to enforce certain
paths that are not realizable using routing protocols because 
static
routes for a given destination have the highest preference over other routes for the same
destination.  
We define the static route partial
function $SR: V \times \Lambda \mapsto V$ with the following meaning:  
given a router $r$
and destination $\lambda$, if $SR(r,\lambda)=r'$, then traffic for
destination $\lambda$ reaching node $r$ is always forwarded to node
$r'$. \kausik{We assume that  the static routes induced by the function $SR$ form paths and
we use $\Pi_{SR}$ to denote the set of all  static paths. 
Formally, for a destination $\lambda$ and a router $r$, if $SR(r,\lambda)=r'$ then
either $r'$ is directly connected to the subnet $\lambda$ or there exists $r''$ such that $SR(r',\lambda)=r''$}

%These paths would have a highest preference.

\minisection{OSPF} Open Shortest Path First (OSPF) is a routing
protocol commonly used inside a domain. In OSPF routers receive link
state information from other routers and uses it to create a map of
the network. Each link is associated with a positive weights and traffic from a router $r$ to
a destination $\lambda$ is forwarded across the shortest path from $r$ to $\lambda$.
Shortest paths are computed using Dijkstra's algorithm.

Let us
define the OSPF weight function $W: L \mapsto \rat$ which 
maps edges of the topology to positive rational weights. 
% todo: When to talk about converting LRA to LIA.
For a given
path $\pi=l_1\cdots l_n$, we indicate the sum of weights of the
links in the path by $W(\pi)=\sum W(l_i)$. 


\kausik{OSPF also allows  route filters
to selectively disable an
edge for a given destination by  
blocking advertisements to a
particular destination along a link. 
%Formally, a  route filter is a pair $(l,\lambda)\in L\times \Lambda$
%which disables the link $l$ for destination $\lambda$. 
We define the  route filter function 
$RF: \Lambda \rightarrow 2^L$ which maps each destination  
to a set of disabled links for that destination. 
Given an   destination $\lambda\in \Lambda$, 
a legal path with respect to $RF$ and $\lambda$
is a valid path $\pi=l_1\cdots l_k$ such that for every $i$,
$l_i\not\in RF(\lambda)$.
We use $\Pi_\lambda$ to denote the set of all such paths.}

We define the routing function 
$\route_{ospf}^G$ 
that, given 
%a weight function $W$,
%a  route filter function $RF$,
a source node $s$,
a target node $t$,
and destination   
$\lambda$,
returns the OSPF-induced paths from $s$ to $t$ for destination   $\lambda$.
Informally, $\route_{ospf}^G(s,t,\lambda)$ returns the shortest path between
$s$ and $t$ that does not traverse any link in $RF(\lambda)$. The following definition also
accounts for the higher priority of static routes.
\[
\route_{ospf}^C(s,t,\lambda) = 
\begin{cases}
\pi  & \text{$\Pi_s^t \cap \Pi_{SR} =\{\pi\}$} \\
\pi & 
\text{$\Pi_s^t \cap \Pi_{SR} =\emptyset \wedge \pi\in \Pi_s^t\cap \Pi_\lambda ~\wedge $}  \\
& \text{~~~~$\neg\exists \pi'\in\Pi_s^t\cap \Pi_\lambda . W(\pi')< W(\pi)$} 
\end{cases}
\]

\minisection{BGP} BGP is a path-vector inter-domain routing protocol
that connects different domains (also called autonomous systems), where each domain
comprises of one or more routers (typically managed by a single
entity). A BGP router receives routes from BGP peers (internal peers
send iBGP routes, external peers send eBGP routes). Each route for the
destination comprises of a domain-level path to the destination
domain, and by default a BGP router will forward a packet through  the path that traverses fewer domains.
 
To support policy routing, BGP routers can be configured with a
\emph{local preference} variable to assign higher priorities to
specific routes for a particular destination. We define the local
preference function $LP: V \times V \times \Lambda \mapsto \nat$ as
follows.  Given a BGP router $r$, neighbouring router $r'$, and  
destination $\lambda$, $LP(r, r', \lambda)$ specifies the local
preference for the router $r$ to choose $r'$ as the next-hop router for destination $\lambda$.  This function is defined
on inter-domain links---i.e., $r$ and $r'$ belong to different adjacent domains.  
From the routes received at $r$, the router
picks one with greatest local preference.

The Internal BGP (iBGP) protocol is used to 
exchange external BGP routes 
among BGP routers belonging
to the same domain. Informally, iBGP propagates local preferences of the routes to all the BGP routers within the same domain. 
We can configure iBGP 
filters to prevent a router from advertising 
a route to another router. We do so using the 
function $IF: V \times \Lambda \mapsto 2^V$;
if a BGP router $r_1 \in IF(r_2, \lambda)$, then
$r_2$ will not advertise a route for $\lambda$ to
$r_1$ through iBGP. 
 
 \iffull
Algorithm~\ref{alg:bgppathrules} defines the routing function 
$\route_{bgp}^C$ 
that given 
%a local preference function $LP$,
a BGP router $r$
and destination  
$\lambda$,
returns 
the next-hop BGP router in the path. 
\else
We only describe the routing semantics of BGP
informally and the complete BGP path selection algorithm is submitted as part of the supplementary material.
\fi
BGP 
chooses a route with highest local preference, and
if there is a tie, it then chooses the route the traverses fewer domains~\cite{bgp}. 
\name synthesizes configurations where all ties are 
are broken using these criteria. Formally, we have a routing function 
$\route_{bgp}^C(r,\lambda)$, that given a switch $r$ and a destination $\lambda$,
returns the next-hop router for going from $r$ to $\lambda$. 
Finally, static routes 
have a higher preference than BGP routes, therefore a static route
$SR(r, \lambda)$ will be preferred over BGP.

\begin{algorithm} [t]
	\begin{footnotesize} 
		\caption{BGP Best Path Selection at router $r$ for dst. $\lambda$}
		\label{alg:bgppathrules}
		\begin{algorithmic}[1]
			\Procedure{BestPath}{} \\
			\hspace*{0.4cm} [Each route $\gamma$ of form $(\lambda, local\_pref, len, nexthop)$]
			\State{$\Upsilon_{ebgp}$ = $\bigcup$ Route from external BGP neighbour for $\lambda$} 
			\State{$\Upsilon_{ibgp}$ = $\bigcup$ Route from internal BGP router $r_1$:
				 $r \not\in IF(r_1, \lambda)$}
			\State{$\Upsilon = \Upsilon_{ebgp}\cup \Upsilon_{ibgp}$}
			\\
			\If{$SR(r,\lambda) = r_1$} 
			\State{$\route^C_{bgp}(r, \lambda) = r_1$\hfill [Static Route has highest pref.]}
			\Else \\
			\indent\indent \hspace{0.3cm}[Find routes with highest \emph{local preference}]
			\State{$\Upsilon_{lp} = \{\gamma \in \Upsilon \mid \forall \gamma_1 \in \Upsilon. ~\gamma.local\_pref \geq \gamma_1.local\_pref\}$}
			\If{$|\Upsilon_{lp}| = 1$}
			\State{$\gamma_{best} = \gamma \in \Upsilon_{lp}$}
			\Else \newline
			\hspace*{0.7cm}\indent [Prefer the path with the smallest \emph{AS Path length}]
			\State{$\Upsilon_{as} = \{\gamma \in \Upsilon_{lp}  \mid \forall \gamma_1 \in \Upsilon_{lp}. ~\gamma.len \leq \gamma_1.len\} $}
			\If{$|\Upsilon_{as}| = 1$}
			\State{$\gamma_{best} = \gamma \in \Upsilon_{as}$}	
			\EndIf
			\EndIf
			\If{$\gamma_{best} \in \Upsilon_{ebgp}$}
			\State{$\route^C_{bgp}(r, \lambda) = \gamma_{best}.nexthop$}
			\State{Redistribute $\gamma_{best}$ to OSPF domain}
			\Else \\
			\hspace*{0.7cm}\indent [Traffic for $\lambda$ will not exit domain through $r$]
			\State{$\route^C_{bgp}(r, \lambda) = \bot$}
			\EndIf
			\EndIf
			\EndProcedure
			%				\State{ . }
			%				\State{Prefer the path with the lowest \emph{multi-exit discriminator} (MED).}
			%				\State{Prefer eBGP over iBGP paths.}
			%				\State{Prefer the route that comes from the BGP router with the lowest \emph{router ID}.}
		\end{algorithmic}
	\end{footnotesize}
\end{algorithm}


\minisection{OSPF+BGP+Static Routes} We now describe how routing
happens in hierarchical domains.  Assume that in domain $d$, traffic 
for destination $\lambda$ originates at source
router $s$---i.e., $\Theta(s) = d$. 
If the target router $t$ connected to the
destination $\lambda$ belongs to domain $d$, then there exists a OSPF route for
$\lambda$ at $s$, and the path taken would be
$\route_{ospf}^C(s,t,\lambda)$.

Consider the case of destination $\lambda$ being connected 
to an external domain, thus, traffic for destination $\lambda$
will be sent to a BGP gateway router $r$ of domain $d$ and $r$ 
will forward  the traffic to its neighbouring domains till
it reaches the internal domain of $\lambda$. Formally, a BGP gateway
router is a router $g$ such that $\exists g' \in N(g). 
~\Theta(g') \not= \Theta(g)$. 

However, not all BGP routes for $\lambda$
are distributed 
into OSPF (to prevent explosion of forwarding tables). 
BGP gateways in a domain exchange routes using iBGP,
and if the best path chosen is an eBGP route, 
the gateway redistributes the eBGP route into OSPF. 
Therefore, if multiple BGP routers
can redistribute a route for $\lambda$,  
the closest gateway in terms of OSPF distance
is chosen. We define a gateway function $G^C$,
that given a domain number $n\in \nat$,
%a local preference $LP$,
%a weight function $W$,
%route filters $RF$,
a router $r$ such that $\Theta(r)=n$,
and destination $\lambda$ outside of domain $n$, 
returns the closest
gateway router to $r$ for destination $\lambda$. 
Finally, the routing function 
$\route^C$
%:\Theta \times LP \times W \times RF \times V \times V \times \Lambda \rightarrow L^* 
returns paths from source to destination in a inter-domain network:
given a source $s_1$ and a destination $\lambda$ connected to target router $t$, 
\[
\begin{array}{c}
	\route^C(s_1, t, \lambda) = 
	\route_{ospf}^C(s_1,g_1, \lambda) + 
	 (g_1, s_2 )+\qquad\qquad\qquad  \\
%	~~~~~~~~~~~~~~~~~~~~~~~~~\route_{ospf}^C(s_2,g_2, \lambda) + (g_2, s_3) + \\
	\qquad\qquad\qquad\qquad \cdots  + (g_{n-1}, s_n) +\route_{ospf}^C(s_n,t,\lambda)
\end{array}
\]
Here, for each $i>1$, $g_i=G^C(\Theta(s_i),s_i,\lambda)$, 
$s_i=\route_{bgp}^C(g_{i-1}, \lambda)$,
and  $+$ is the concatenation operator for paths. 

\minisection{Induced paths}
We assume a set of packet classes $PC : [0,C_{pc}]$ 
and map each path (only reachability policies produce paths,
the rest specify properties on these paths)
to a unique integer in $PC$.
For a packet class $pc$, $s_{pc}$ denotes the source router,
$t_{pc}$ denotes the destination router, and $\lambda_{pc}$
denotes the destination   address. 


\begin{definition}[Induced Paths]
Given a configuration $C$ and a set of packet classes $PC$, the set of paths
$\paths^C(PC)$ induced by $C$ is defined as follows: 
$\forall pc \in PC.$ $\route^C(s_{pc}, t_{pc}, \lambda_{pc})$ $\in \paths^C(PC)$.
\end{definition}

\subsection{Problem Definition}
\begin{table}[!t]
\begin{small}
	\begin{center}
		\begin{tabular}{m{7.8em}  m{15.9em} } 
			{\bf Policy} & {\bf Description} \\ 
			\hline
			Reachability & There is a path from router $src$ to router $dst$ for destination $\lambda$ \\ \hline
			Reachability with \newline Ordered Waypoints & The path  from $src$ to $dst$ for destination $\lambda$ 
			traverses some switch in the set $W_1$, \ldots, then some switch in the set $W_k$.\\ \hline
			Traffic Isolation & Paths of two reachability policies $R1$ and $R2$ do not share  links \\ \hline
			Traffic Engineering  & Minimize total/max link utilization \\
		\end{tabular}
	\end{center}
	\compactcaption{\genesis path-based policy support.} \label{tab:policysupport} 
\end{small}
\end{table}


%\section{Policy Support} \label{sec:policy}
%We design a language GPL (Genesis Policy Language) for network operators to express the desired end-to-end policies in a declarative manner which is interpreted by the Genesis synthesizer to find the forwarding rules for the network topology which enforce the input policies (\cref{fig:arch}). Genesis supports the following policies : 
%% Figure of GPL's syntax
%\begin{enumerate} 
%	\item \textbf{Reachability}: $predicate : src >> dst$ \\
%	This policy specifies the packets satisying $predicate$ have ingress router $src$ and egress router $dst$, and requires rules forwarding packets satisfying $predicate$ from $src$ to $dst$. There must be no forwarding loops in the network. 
%	\item \textbf{Waypoint}: $predicate : src >> W >> dst$ \\
%	The waypoint policy is a stronger reachability policy, and specifies that packet satisfying $predicate$ with ingress router $src$ and egress router $dst$ must pass through the set of waypoints $W$ in no particular order. The waypoint policy helps operators and tenants to specify the middleboxes the packets must traverse through without worrying about order, or having to use header tags to enforce a particular order \cite{flowtags}. 
%	\item \textbf{Traffic Isolation}:  $R1 \ || \ R2$ \\
%	The traffic isolation policy ensures that the . This policy can be used to provide fairness guarantees, since the paths of $R1$ and $R2$ don't share a link, the bandwidth used by $R1$ will not affect the bandwidth used by $R2$ and vice-versa. The condition of sharing a link in the same direction is due to the fact that links are full-duplex so, traffic flowing in one direction is not affected by the traffic flowing in the other direction.
%	\item \textbf{Security Isolation}: $R1 <> R2$ \\
%	The security isolation policy is stronger than the traffic isolation policy, and ensures that the path of the reachabiltiy/waypoint policies $R1$ and $R2$ do not share a link in both directions for increased security.
%	\item \textbf{}: $sw_1 \rightarrow sw_2 : capacity$ \\
%	The link capacity policy specifies that the capacity for link $sw_1 \rightarrow sw_2$ is $capacity$, and the weights of flows traversing the link in the direction of $sw_2$ do not exceed the capacity of the link.  
%	\item \textbf{Switch Table Size}: $sw : size$ \\
%	The router table size policy is used to specify the size of the forwarding table of the router $sw$ and ensures that the number of flows traversing through $sw$ does not exceed $size$ as each flow would require a forwarding rule at the switch.
%\end{enumerate}



\minisection{Policies}
\Cref{tab:policysupport} describes the path policies supported by \name.
Given  a set $\Psi$ of
path policies, we say that
a set of paths $\Pi$ is policy-compliant with respect to $\Psi$, $\Pi \models \Psi$,
if the paths in $\Pi$ satisfy all the policies in $\Psi$ (see~\cite{genesis} for the formal definition). 


\Cref{tab:configpolicysupport} describes all the configuration policies supported by \name.
A configuration $C$ satisfies a set of configuration policies $P$
if $C$ satisfies every constraint in $P$.
For each configuration policy 
we define what it means for  a configuration $C=(\Theta,W,LP,IF,SR)$ to satisfy it.

The first type of policy restricts how many OSPF domains $N_D$ a
configuration might have
and it is satisfied by $C$ iff $c_1\leq |\{\Theta(r)\mid r\in V\}|\leq
c_2$.  This policy is useful for avoiding situations resulting in too
many domains in the hierarchical split which can be difficult to
administer.  The second type of policy allows the operator to restrict
the size $ds$ of each OSPF domain, $l\leq ds\leq u$, because OSPF does
not scale gracefully with network size.  This policy is satisfied if
for every domain $i$, $l\leq |\{r \mid \Theta(r)=i\}|\leq u$.


Certain 
	routers may not be suited to run BGP due to resource
	constraints. Thus, the operator can specify what set of 
	routers $B\subseteq V$ is BGP compatible.  
	This policy is satisfied if none of the routers $V\setminus B$
	is a gateway router.
	Formally, for every $r\in V\setminus B$,
	there does not exist a router $r'\in N(r)$ such that $\Theta(r) \not= \Theta(r')$

\kausik{Finally, \name provides ways to specify upper bounds on the number of
1) route filters, which in general reduce the resilience of the network---i.e., $rc=|\cup_{\lambda\in \Lambda} RF(\lambda)|\leq C_{rc}$},
2) static routes---i.e., $sc=|\{(v,\lambda)\mid \exists v'. SR(v,\lambda)=v'\}|\leq C_{sc}$, and
3) BGP configurations, which increase the complexity of the network---i.e., 
$bc=|\bigcup_{v,\lambda\in V\times\Lambda} IF(v,\lambda)\cup \{(v,v',\lambda)\mid LP(v,v',\lambda)\neq \bot\}|\leq C_{bc}$.
\name also allows to minimize certain expressions over the quantities $rc$, $sc$, and $bc$---e.g., $max(rc, sc+bc)$. 
\kausik{Ideally, the most important metric we want to maximize is resilience---i.e.,
the number of alternate paths between endpoints.
However, this metric is too hard to handle as it depends on complex graph properties. 
In practice, when we want improve resilience, we try to minimize
the number of filters; we will show that this strategy is effective in our experiments. }
\begin{table}[t!]
\begin{small}
	\begin{center}
		\begin{tabular}{m{7em} | m{17em} } 
			Policy & Description \\ 
			\hline
			Count & Number of continous OSPF domains  \\ \hline
			Domain Size  & Lower and upper
			limit of size of domain (number of routers) \\ \hline
			BGP \newline Disable & Disable BGP on router (hardware constraint). 
			Router cannot lie on boundary of domain \\ \hline
			Endpoint \newline Resilience: $rc$ & Lower bound endpoint resilience: \newline
			$rc > C_{rc}$\\ \hline
			Static Route: ${sc}$ & Upper bound number of static route rules: $sc < C_{sc}$ \\ \hline
			BGP Config. Overhead: $bc$ & Upper bound the number of local preference entries and iBGP filters $bc < C_{bc}$ \\ \hline
			Cost Minization & Minimize user-defined cost $expr(rc, sc, bc)$
		\end{tabular}
	\end{center}
	\compactcaption{\name Domain assignment policy support.} \label{tab:configpolicysupport} 
\end{small}
\end{table}

%different configuration metrics. For example,
%we allow  route filters in OSPF synthesis 
%(\secref{sec:ospfsynthesis}), which may yield undesirable
%endpoint resilience, thereby, synthesizing configurations
%which provide a specified metric of resilience or maximize
%the metric. Similarily, to enable path-based inter-domain routing, 
%\name needs	to set up static routes along the path, or configure BGP 
%variables like local preferences to ensure the configurations 
%induce the input paths.
%These increase the size of the configurations,
%thus increasing the complexity of verifying correctness either 
%manually or using verification tools~\cite{batfish, arc, era}. 
%Therefore,
%another objective \name considers is the configuration overhead
%to ensure path-compliance.



\minisection{Synthesis problems}
\noindent We now define when a configuration $C$ is policy-compliant
and present our synthesis problems.
\begin{definition}[Policy-Compliance] \label{def:policycompliance}
	Given a set of path policies $\Psi$ and a set of configuration policies $P$,
	a configuration $C$ is policy-compliant with $(\Psi,P)$,
	$C \models (\Psi,P)$, if the set of
	induced paths satisfies $\Psi$, $\paths^C(PC) \models \Psi$,
	and $C$ satisfies the policies in $P$.
	The \emph{configuration synthesis problem} is to find, given $\Psi$ and $P$,
a configuration $C$ that is policy compliant with $(\Psi,P)$.
\end{definition}

As anticipated in Section~\ref{sec:motivation}, our approach will proceed in two phases,
one of which solves the following sub-problem.  
\begin{definition}[Path-Compliance] \label{def:pathcompliance}
Given a set of configuration policies $P$
and a set of paths $\Pi$ over packet classes $PC$,
	a configurations $C$ is path-compliant with 
	$(\Pi,P)$,
	if $\paths^C(PC)=\Pi$ and $C$ satisfies the policies in $P$.
	The \emph{path-compliance synthesis problem} is to find, given $P$ and $\Pi$,
a configuration $C$ that is path-compliant with $(\Pi,P)$.
\end{definition}



