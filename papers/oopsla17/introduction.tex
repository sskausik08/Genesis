\section{Introduction}
Programming networks to correctly forward flows according to user- and
application-induced policies is difficult and
error-prone~\cite{troubleshooting, bgpmisconfig}. At least three
common characteristics of network policies are to blame: (1) a network
may need to satisfy several types of policies, including reachability,
isolation, service chaining, resilience, and traffic engineering; (2)
the network must provide certain guarantees in the event of failures;
and (3) most policies are global---i.e., they concern end-to-end
paths, not individual devices/links.

The global nature of network policies is one motivation for
software-defined networking (SDN). SDN allows paths to be centrally
computed over a global view of the network. However, it is difficult
to ensure forwarding paths are correctly computed and installed in the
presence of failures, even if the SDN controller is
distributed~\cite{hasdn}.  Traditional control planes that rely on
distributed protocols to compute paths offer greater fault tolerance,
but determining the appropriate distributed realization of policies is
hard.

Our goal is to {\em automate the process of creating a correct
and failure-resilient
  distributed realization of policies in a traditional control
  plane}. We wish to handle a wide range of policies, e.g.,
reachability, service chaining, and traffic engineering, to meet
applications' diverse security and compliance requirements. 
With
increasing sizes of networks, we must further provide support for
realizing hierarchical control planes to ensure scalability,
necessitating the use of multiple intra- and inter-domain routing
protocols (e.g., OSPF and BGP). 
Finally, and most importantly, we want
configurations that are resilient to network malfunctions such as link
failures.
Thus, our work contrasts with prior efforts~\cite{netegg,
  propane, merlin,simple,fattire, netkat, netkatcompiler, sol}, which
generate SDN- or BGP-specific control planes for a limited range of
policies (e.g., peering) and do not attempt to be resilient to failures.



The problem of synthesizing router configurations
for which the distributed control plane 
resilient to failures and it
generates policy-compliant paths 
is computationally hard. 
First, even generating a set of policy-compliant 
paths for a SDN  is 
computationally hard---e.g., enforcing isolated
paths is NP-complete. 
Second, to infer the concrete
set of paths induced by network configurations, 
one has to incorporate
into synthesis
complex concepts---e.g., reasoning about shortest path algorithms
requires constraints in complex
theories that combine propositional logic (SAT) 
with linear rational
arithmetic (LRA). Even with recent 
advances in Satisfiability Modulo Theories
(SMT) solvers, 
approaches that directly generate configurations  from policies
do not scale to
even moderately-sized networks or 
sets of policies~\cite{synet}.
Third, to generate resilient control planes one has to reason
about how different protocols react to failures, 
which further complicates an already intractable synthesis
problem. 


In this paper, we present \name, a new approach for synthesizing
highly resilient policy-compliant configurations.
Unlike previous approaches, \name uses a two-phased approach
that does not attempt to generate 
a policy-compliant control plane in a single step.
First, \name 
uses the tool \genesis~\cite{genesis}
to synthesize paths---i.e., the forwarding state---that are compliant
with given policies, such as, waypoint, isolation,
and traffic engineering.
Using these paths, \name generates 
intra-domain (shortest-path OSPF) and inter-domain (BGP) router configurations
that induce the forwarding
state synthesized by \genesis and provide high resilience. 
%We consider hierarchical networks  routers run
%OSPF for intra-domain routing, while domains use BGP for inter-domain
%routing. Moreover, routers can install static routes---i.e., default routes that can bypass the other two protocols.
We investigate three different settings with different policies and resilience requirements
and show how \genesis, using its two-phase approach can effectively generate
highly-resilient and policy-compliant solutions.

First, we consider a setting in which the operator 
of a hierarchically structured network
wants to 
enforce a large set of complex policies such as waypoint-compliance, isolation, and traffic engineering.
The operator also requires a simple notion of \emph{connectivity-resilience} to 
guarantee that, under most common link failures,
packets can still reach their destination.
%In general, in the absence of static routes a network can support
%a number of link failures equal to the min-cut of the network, regardless
%of the OSPF and BGP configurations.
%However, we might need static routes to generate policy-compliant configurations.
%Static routes can cause unexpected network behaviors after a failure---e.g., a forwarding loop---and
%we argue that the \emph{connectivity-resilience} of a configuration is related to the its number of static routes.
To increase connectivity-resilience, \name tries to produce policy-compliant configurations with a small number of static routes---i.e., static configurations that bypass OSPF and BGP routes.
%---helps 
%increasing \emph{connectivity-resilience} and
%decreasing configuration complexity. 
%thus, easing automated
%verification~\cite{batfish, arc, era} and improving readability for
%performing future manual changes.
%Therefore, \name aims at producing policy-compliant configurations with a small number of static routes.
To do so, \name synthesizes OSPF configurations using linear constraint solving to compute
link weights and uses the unsatisfiable cores
of failing solving attempts to learn when to introduce static routes.
\name synthesizes BGP configuration directly from the domain mapping and the paths.
Using these techniques, \name can generate configurations that are
\loris{XX} more resilient than configurations that only static static routes.

Second, we consider a setting in which the operator 
wants to 
enforce a restricted set of policies and wants to guarantee
that, for most of the common link failures, the resulting configuration is still policy-compliant.
We call this notion \emph{policy-resilience}.
Given the complexity of this problem, we focus on a restricted class of policies.
In particular, for each class of packets, we allow the operator to specify a set of waypoints---e.g.,
a set of firewalls---that packets must traverse before reaching their destination.
To generate configurations with high policy-resilience 
we modify our linear constraints
to guarantee that at least two paths that traverse the waypoints
have lower OSPF cost than any path not going through a waypoint. 
Using this improved technique, \name can generate configurations that are
\loris{XX} more policy-resilient than configurations generated with our first technique.


Finally,  we consider a setting in which the operator has the flexibility to
assign routers to different domains.
We present a stochastic search technique that uses this flexibility it to look for
ways to assign switch to different domains so that the synthesized configurations have higher resilience.
Using this search technique, \name can further improve the resilience of the configurations 
configurations generated by the previous two techniques by
\loris{XX}.

%We evaluate \name on medium-sized topologies and show that
%\name can synthesize OSPF configurations for 200 paths in 200
%seconds for a 40 node ISP topology, and achieve greater than 
%50\% resilience for fat-tree topologies.
%\todo{what does resilient mean here? this last para and contrib need to be rewritten
%after eval}
% On average, using MCMC, \name can increase
%endpoint resilience by $1.6\times$ and 
%reduce configuration overhead
%by $0.3\times$
%for a 125-node ISP topology.


%% Automatically synthesizing distributed realizations 
%% of network policies is an
%% important step towards simplifying 
%% network management and providing an 
%% SDN-like interface for programming networks 
%% running distributed routing protocols. 
%% Our approach is an important
%% contribution towards the vision of 
%% intent-driven networking~\cite{intent}.

\minisection{Contributions} We make the following contributions. \todo{after intro is approved}
\begin{itemize}
	\item \name: a modular framework 
	for policy enforcement in `traditional' networks
	by synthesizing router configurations from policy-compliant paths (\S~\ref{sec:architecture}). 
	\item Algorithms that synthesize OSPF and BGP routing configurations that use
	constraint solving in Linear rational arithmetic (LRA) and 
	a procedure to extracts OSPF configurations from 
	unsatisfiable cores (\S~\ref{sec:config-synthesis}). 
	\item Stochastic search to find 
	partitions of the network into multiple routing domains which
	satisfy policies and optimization objectives on configurations (\S~\ref{sec:synth-dom-ass}).
	\item An implementation of \name and evaluation of the 
	intra- and inter-domain configuration synthesis for different
	topologies and workloads (\S~\ref{sec:evaluation}). 
\end{itemize}
