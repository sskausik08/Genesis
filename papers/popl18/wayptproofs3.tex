\begin{theorem}[Soundness]
	For a set of edge-disjoint waypoint paths $\{(\pi^1_{pc}, \pi^2_{pc}, \waypt_{pc}) ~|~ pc\}$, 
	if edge weights $W$ satisfy constraints (\ref{eq:wdistance}), (\ref{eq:waypoint}) and
	(\ref{eq:resilience}), 
	then for any arbitrary single link failure, 
	the shortest path between each packet class's 
	endpoints traverses one of the waypoints in $\waypt_{pc}$.
\end{theorem}
\kausik{Proof very similar to Thm 5.1, could be avoided}
\begin{proof}
	Let us assume there exists a packet class $pc$ with waypoint set $\waypt_{pc}$ 
	and edge-disjoint waypoint paths $\pi^1_{pc} = (s_{pc}, s_1)(s_1, s_2)\ldots (s_n, d_{pc})$, 
	and $\pi^2_{pc} = (s_{pc}, t_1)(t_1, t_2)\ldots (t_l, d_{pc})$; and there exists 
	a link $l$ failure 
	such that for $pc$, the 
	shortest path $\sigma_{pc}=(s_{pc}, r_1)$ $(r_1, r_2)\ldots (r_m, d_{pc})$ 
	is not waypoint-compliant------i.e.,  
	for every $i$, we have $r_i\not\in \waypt$.
	
	A single link $l$ failure cannot disable both $\pi_{pc}^1$ and $\pi_{pc}^2$ as they are 
	edge-disjoint. 
	\paragraph{Case 1:} Link $l$ is not in path $\pi_{pc}^1$. Since, $\sigma_{pc}$ is 
	the shortest path: 
	\[
	\sum_{\sigma_{pc}}W \leq \sum_{\pi_{pc}^1}W
	\]
	Constraint (\ref{eq:resilience}) imposes the following: 
	\[
	\sum_{\pi_{pc}^1}W < D(s_{pc}, d_{pc}, \waypt_{pc}) 
	\]
	Since, $\sigma_{pc}$ does not traverse any waypoint, we can use 
	constraints (\ref{eq:wdistance}) for expanding $D(r_k, d_{pc}, \waypt_{pc})$ 
	for $1 \leq k \leq m$: 
	\[
	\sum_{\pi_{pc}^1}W < W(s_{pc}, r_1) + D(r_1, d_{pc}, \waypt_{pc}) 
	\]
	\begin{center}
	$\ldots$
	\end{center}
	\[
	\sum_{\pi_{pc}^1}W < W(s_{pc}, r_1) + W(r_1, r_2) + \ldots + W(r_m, d_{pc}) + D(d_{pc}, d_{pc}, \waypt_{pc}) = \sum_{\sigma_{pc}}W
	\]
	This contradiction the assumption that $\sigma_{pc}$ is the shortest path from $s_{pc}$ to $d_{pc}$.

	\paragraph{Case 2:} Link $l$ is not in path $\pi_{pc}^2$. Proof same as Case 1 for $\pi_{pc}^2$. 
	\paragraph{Case 3:} Link $l$ is not in path $\pi_{pc}^1$ or $\pi_{pc}^2$. Using \Cref{thm:waypoint}, 
	$\sigma_{pc}$ cannot be the shortest path as it does not traverse any waypoint in $\waypt_{pc}$. 
	
	Hence proved, under any arbitrary single-link failure, the shortest path between each packet class's 
	endpoints traverses one of the waypoints in $\waypt_{pc}$.
\end{proof}

\kausik{
Waypoint synthesis with two paths is not complete---i.e.,  A configuration $C$ 
could violate constraints (\ref{eq:resilience}) 
but each packet class's path is waypoint-compliant
under single link failures. One or both $\pi_1$ and $\pi_2$ can have weights
greater than $D(s,t,\waypt)$, but there are two edge-disjoint waypoint-compliant 
paths in the network.
}
