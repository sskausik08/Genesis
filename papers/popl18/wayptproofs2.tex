%!TEX root=paper.tex
\begin{theorem}[Soundness]
	Given a set of waypoint paths $\{(\pi_{pc}, \waypt_{pc}) ~|~ pc\}$,
	\Cref{alg:wayptunsat} outputs a configuration $C(W,SR)$ 
	such that for every packet class, 
	there exists a path in $\paths^C(pc)$ which
	traverses one of the waypoints in $\waypt_{pc}$.
\end{theorem}
\kausik{This proof is quite long, and is complicated
by the fact that we dont enforce a particular path, 
and static routes, which force me to enumerate a lot of 
cases, I will think if there is an easier proof than the 
half-completed one I have below. But I am pretty sure that 
the theorem is correct.}
\begin{proof}
	Let us assume there exists a packet class $pc$ 
	with destination $\lambda$, 
	waypoint set $\waypt_{pc}$ 
	and waypoint path $\pi_{pc} = (s_{pc}, s_1)(s_1, s_2)\ldots (s_n, d_{pc})$, 
	such that for $pc$, 
	there exists no path in $\paths^C(pc)$
	which is waypoint-compliant. There are two
	cases: either there exists no path in $\paths^C(pc)$ 
	or the paths do not traverse any waypoint in $\waypt_{pc}$.

	\paragraph{Case 1:} $\paths^C(pc) = \emptyset$---i.e., there
	is a routing loop caused by static routes (OSPF forwarding
	is loop-free).
	We denote the set of static routes for $\lambda$ be $SR(\lambda)$. Note that
	multiple packet classes can share the same destination IP, and we 
	construct a destination-based tree from these paths, denoted by 
	$\xi_lambda$. 

	Let us assume the routing loop to be 
	$(r_1, r_2) (r_2, r_3) \ldots (r_n, r_1)$.
	Let $r_i$ and $r_j$ with $i < j$ be two routers
	such that 

	\kausik{More involved than I expected!}

	\paragraph{Case 2:} All paths in $\paths^C(pc)
	\not=\emptyset$ are not waypoint-compliant. 
	Let $\sigma_{pc} = (s_{pc}, r_1)(r_1, r_2) \ldots
	(r_n, d_{pc}$ be one such path in $\paths^C(pc)$.
	Given the routing function $\route^C$ constructed from $SR$ and
	$W$ (\secref{sec:routingmodel}), let the first router where routing diverges from $\pi_{pc}$ be $s_p$---i.e.,  
	$s_{p+1} \not\in \route^C(s_{p}, \lambda_{pc})$. 
	\Cref{alg:wayptunsat} on line \ref{line:waypoint} adds the following
	constraint to ensure the sub-path of $\pi_{pc}$ 
	from $s_{p}$ to $d_{pc}$ is shorter than non-waypoint paths:
	\begin{equation} \label{eq:wuniq}
	\sum_{\mathclap{\substack{(s_{p}, s_{p+1})\cdots(s_n, d_{pc})}}} 
	W < W(s_{p}, r_{p+1})+ D(r_{p+1}, d_{pc}, \waypt_{pc})
	\end{equation}

	\Cref{alg:wayptunsat}
	only removes a subset of the waypoint and loop constraints
	constraints and 
	not the distance constraints (\ref{line:wremoveconstraint}). We consider 
	two cases: whether \Cref{alg:wayptunsat}
	removes Constraint (\ref{eq:wuniq}) or not. 

	\paragraph{Case 1:} 
	\Cref{alg:wayptunsat} does not remove Constraint (\ref{eq:wuniq}). 
	Thus, there is no static route $(s_p, s_{p+1})$ for
	$\lambda_{pc}$ (line \ref{line:wstaticroute}), and 
	OSPF-based forwarding occurs at $s_{p}$. 
	Since $\sigma_{pc}$ does not traverse any 
	waypoint in $\waypt_{pc}$,
	we use constraints (\ref{eq:wdistance}) 
	to expand the terms $D(r_k, d_{pc}, \waypt_{pc})$ for $p+1 \leq k \leq m$:
	\[
	\sum_{\mathclap{\substack{(s_{p}, s_{p+1})\cdots(s_n, d_{pc})}}} 
	W < W(s_{p}, r_{p+1})+ W(r_{p+1}, r_{p+2}) + D(r_{p+2}, d_{pc},\waypt_{pc})
	\] 
	\begin{center}
		$\ldots$
	\end{center}
	\[
	\sum_{\mathclap{\substack{(s_{p}, s_{p+1})\cdots(s_n, d_{pc})}}} 
	W < W(s_{p}, r_{p+1})+ \ldots + W(r_{m}, d_{pc}) + D(d_{pc}, d_{pc}, \waypt_{pc})
	\]
	By adding weights of $(s_{pc}, s_1)\ldots(s_{p-1}, s_p)$, we obtain that weight of $\sigma_{pc}$ is smaller than 
	$\$ 
	
	Therefore, the weight of any path through $r$ is greater than 
	the path through $s_{p}$. However, 
	OSPF sends traffic through the shortest weighted
	path, therefore, $r$ cannot be in $R^C_{ospf}(s_p, \lambda_{pc})$,
	which is a contradiction. 

	\paragraph{Case 2:}  
	\Cref{alg:wayptunsat} removes Constraint (\ref{eq:uniq}) 
	and $SR(s_p, \lambda_{pc}) = \{s_{p+1}\}$ (lines \ref{line:staticroute}-\ref{line:removeconstraint}). 
	Thus, $s_{p+1} \in \route^C(s_{p}, \lambda_{pc})$ as static routes 
	have the higher priority than OSPF. This contradicts our assumption
	that $s_{p+1} \not\in \route^C(s_{p}, \lambda_{pc})$. 
	\end{proof}