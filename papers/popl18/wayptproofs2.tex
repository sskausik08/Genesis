%!TEX root=paper.tex
\begin{theorem}[OSPF+SR Waypoint Soundness]
	Given a set of waypoint paths $\{(\pi_{pc}, \waypt_{pc}) ~|~ pc\}$,
	\Cref{alg:wayptunsat} outputs a configuration $C(W,SR)$ 
	such that, for every packet class $pc$, 
	there exists a path in $\paths^C(pc)$ that
	traverses one of the waypoints in $\waypt_{pc}$.
\end{theorem}
\iffull
\kausik{This proof is quite long, and is complicated
by the fact that we dont enforce a particular path, 
and static routes, which force me to enumerate a lot of 
cases, I will think if there is an easier proof than the 
half-completed one I have below. But I am pretty sure that 
the theorem is correct.}
\begin{proof}
	Let us assume there exists a packet class $pc$ 
	with destination $\lambda$, 
	waypoint set $\waypt_{pc}$,
	and waypoint path $\pi_{pc} = (s_{pc}, s_1)(s_1, s_2)\ldots (s_n, d_{pc})$, 
	such that for $pc$, 
	there exists no path in $\paths^C(pc)$
	which is waypoint-compliant. There are two
	cases: either there exists no path in $\paths^C(pc)$ 
	or the paths do not traverse any waypoint in $\waypt_{pc}$.

	\paragraph{Case 1:} $\paths^C(pc) = \emptyset$---i.e., there
	is a routing loop caused by static routes (OSPF forwarding
	is loop-free).
	We denote the set of static routes for $\lambda$ by $SR(\lambda)$. Note that
	multiple packet classes can share the same destination IP
	and destination router, and we 
	construct a destination-based tree from these paths, denoted by 
	$\xi_\lambda$ 

	Let us denote the routing loop as 
	$(r_0, r_1) (r_1, r_2) \ldots (r_{n-1}, r_0)$.
	Since, the path is a loop with atleast one static route,
	there must exist one router $r_i$ in the loop
	such that $(r_{i-1}, r_i) \in SR(\lambda)$ and the
	next router in the loop $r_j$ which has a static route
	$(r_j, r_{j+1}) \in SR(\lambda)$ and $r_j$ is not downstream
	to $r_i$ in $\xi_\lambda$ (meaning there is no directed path 
	from $r_i$ to $r_j$ in $\xi_\lambda$). 

	Consider $r_i$. The path from $r_i$ to $r_j$ is 
	due to OSPF forwarding, therefore, there exists a 
	path $\sigma =
	(r_i, r_{i+1})\ldots(r_{j-1}, r_j)(r_j, l_1) \ldots (l_{o}, d_{pc})$
	which is the shortest path  from $r_i$ to $d_{pc}$. 

	Since static routes are only added on edges in $\xi_\lambda$, 
	let us denote the path from $r_{i}$ to $d_{pc}$ in 
	$\xi_\lambda$ as $(r_{i}, u_1)(u_1, u_2)\ldots(u_p, d_{pc})$. 
	Consider constraint (\ref{eq:rla}) which is added for static route 
	$(r_{i-1}, r_i)$ and non-downstream router $r_j$:
	\[
	\sum_{\mathclap{\substack{(r_{i}, u_1)\ldots(u_p, d_{pc})}}} 
	W < D(r_i, r_j) + D(r_j, d_{pc}) 
	\]
	Using distance constraints (\ref{eq:distance}), we can expand 
	$D(r_i, r_j)$ and $D(r_j, d_{pc})$ along the paths 
	$(r_i, r_{i+1})\ldots(r_{j-1}, r_j)$ and $(r_j, l_1) \ldots (l_{o}, d_{pc})$
	respectively. 
	\[
	\sum_{\mathclap{\substack{(r_{i}, u_1)\ldots(u_p, d_{pc})}}} 
	W < \hspace{0.8cm} \sum_{\mathclap{\substack{(r_i, r_{i+1})\ldots(r_{j-1}, r_j)}}}W \hspace{0.6cm} + 
	\hspace{0.6cm} \sum_{\mathclap{\substack{(r_j, l_1) \ldots (l_{o}, d_{pc})}}}W  
	\]
	However, this contradicts the assumption that $\sigma$ is 
	the shortest path from $r_i$ to $d_{pc}$. 

	\paragraph{Case 2:} All paths in $\paths^C(pc)
	\not=\emptyset$ are not waypoint-compliant. 
	Let $\sigma_{pc} = (s_{pc}, r_1)(r_1, r_2) \ldots
	(r_n, d_{pc})$ be one such path in $\paths^C(pc)$.
	Given the routing function $\route^C$ constructed from $SR$ and
	$W$ (\secref{sec:routingmodel}), let the first router where routing diverges from $\pi_{pc}$ be $s_p$---i.e.,  
	$s_{p+1} \not\in \route^C(s_{p}, \lambda_{pc})$. 
	\Cref{alg:wayptunsat} on line \ref{line:waypoint} adds the following
	constraint to ensure the sub-path of $\pi_{pc}$ 
	from $s_{p}$ to $d_{pc}$ is shorter than non-waypoint paths:
	\begin{equation} \label{eq:wuniq}
	\sum_{\mathclap{\substack{(s_{p}, s_{p+1})\cdots(s_n, d_{pc})}}} 
	W < W(s_{p}, r_{p+1})+ D(r_{p+1}, d_{pc}, \waypt_{pc})
	\end{equation}

	\Cref{alg:wayptunsat}
	only removes a subset of the waypoint and loop constraints
	constraints and 
	not the distance constraints (\ref{line:wremoveconstraint}). We further consider 
	two sub-cases: whether \Cref{alg:wayptunsat}
	removes Constraint (\ref{eq:wuniq}) or not. 

\begin{description}
	\item[Case 2A:]
	\Cref{alg:wayptunsat} does not remove Constraint (\ref{eq:wuniq}). 
	Thus, there is no static route $(s_p, s_{p+1})$ for
	$\lambda_{pc}$ (line \ref{line:wstaticroute}), and 
	OSPF-based forwarding occurs at $s_{p}$. 
	Since $\sigma_{pc}$ does not traverse any 
	waypoint in $\waypt_{pc}$,
	we use constraints (\ref{eq:wdistance}) 
	to expand the terms $D(r_k, d_{pc}, \waypt_{pc})$ for $p+1 \leq k \leq m$:
	\[
	\sum_{\mathclap{\substack{(s_{p}, s_{p+1})\cdots(s_n, d_{pc})}}} 
	W < W(s_{p}, r_{p+1})+ W(r_{p+1}, r_{p+2}) + D(r_{p+2}, d_{pc},\waypt_{pc})
	\] 
	\begin{center}
		$\ldots$
	\end{center}
	\[
	\sum_{\mathclap{\substack{(s_{p}, s_{p+1})\cdots(s_n, d_{pc})}}} 
	W < W(s_{p}, r_{p+1})+ \ldots + W(r_{m}, d_{pc}) + D(d_{pc}, d_{pc}, \waypt_{pc})
	\]
	By adding weights of $(s_{pc}, s_1)\ldots(s_{p-1}, s_p)$, we obtain that weight of $\sigma_{pc}$ 
	is greater than $\pi_{pc}$. If there is no static route on $\sigma_{pc}$, then, this is a contradiction
	as OSPF has a shorter path $\pi_{pc}$ to send traffic to.  If $\sigma_{pc}$ has a static route on the
	path, we consider the first divergence from the static route and use (\ref{eq:wdistance}) from that
	router. 
	\kausik{Very sketchy. }
	
	\item[Case 2B:]
	\Cref{alg:wayptunsat} removes Constraint (\ref{eq:waypoint}) 
	and $SR(s_p, \lambda_{pc}) = \{s_{p+1}\}$ (lines \ref{line:wstaticroute}-\ref{line:wremoveconstraint}). 
	Thus, $s_{p+1} \in \route^C(s_{p}, \lambda_{pc})$ as static routes 
	have the higher priority than OSPF. This contradicts our assumption
	that $s_{p+1} \not\in \route^C(s_{p}, \lambda_{pc})$. 
	\end{description}
	
	Hence, for every packet class, 
	there exists a path in the set of induced paths which traverses through one of the waypoints. 
	\end{proof}
\fi

\paragraph{Completeness}
The same observations as in Section~\ref{sec:waypoint-compliance-constraints}
also apply to this case and completeness can be achieved by running the algorithm
on different sets of paths.
%
%	Given a set of waypoint paths $\{(\pi_{pc}, \waypt_{pc}) ~|~ pc\}$,
%	\Cref{alg:wayptunsat} always terminates with a configuration $C$ where
%	each packet class is waypoint-compliant, the trivial solution being that of 
%	using static routes on each link in $\pi_{pc}$. 
	Moreover, similarly to what we observed in \Cref{alg:unsat}, \Cref{alg:wayptunsat} is not 
	guaranteed to find $C$ with a given bound on number of static routes.
Completeness is recovered by running \Cref{alg:unsat} multiple times, and by choosing different
unsatisfiable cores and static routes assignments. 
