\begin{theorem}[Soundness] \label{thm:waypoint}
	Given a set of waypoint paths $\{(\pi_{pc}, \waypt_{pc}) ~|~ pc\}$, if edge weights 
	$W$ satisfy constraints (\ref{eq:wdistance}) and (\ref{eq:waypoint}), for
	each packet class $pc$, the shortest path between its endpoints
	traverses one of the waypoints in $\waypt_{pc}$.
\end{theorem}
\begin{proof}
	Let us assume there exists a packet class $pc$ with waypoint set $\waypt_{pc}$ 
	and waypoint path $\pi_{pc} = (s_{pc}, s_1)(s_1, s_2)\ldots (s_n, d_{pc})$, 
	such that  the 
	shortest path $\sigma_{pc}=(s_{pc}, r_1)$ $(r_1, r_2)\ldots (r_m, d_{pc})$ 
	is not waypoint-compliant---i.e.,  
	for every $i$, we have $r_i\not\in \waypt$.	
	Since $\sigma_{pc}$ is the shortest weighted path: 
	\begin{equation} \label{eq:wassumption}
	\sum_{\pi_{pc}} W \geq \sum_{\sigma_{pc}} W
	\end{equation}

	
	Let $s_i \not= r_i$ be the first point of divergence of the paths---i.e., for every $j<i$, $s_{j} = r_{j}$.
Constraints (\ref{eq:waypoint}) impose 
$(s_{i-1}, s_i)\ldots(s_n, d)$ to be 
shorter than any path which is not waypoint-compliant. 
Let us consider the neighbour router $r_1$ in $\sigma_{pc}$:
	\[
	\sum_{\mathclap{\substack{(s_{i-1}, s_i)\cdots(s_n, d)}}} 
	W < W(s_{i-1}, r_i)+ D(r_i, d, \waypt_{pc})
	\]
	Since $\sigma_{pc}$ does not traverse any waypoint in $\waypt_{pc}$,
	we use constraints (\ref{eq:wdistance}) 
	to expand the terms $D(r_k, d_{pc}, \waypt_{pc})$ for $i \leq k \leq m$:
	\[
	\sum_{\mathclap{\substack{(s_{i-1}, s_i)\cdots(s_n, d)}}} 
	W < W(s_{i-1}, r_i)+ W(r_i, r_{i+1}) + D(r_{i+1}, d,\waypt_{pc})
	\] 
	\begin{center}
		$\ldots$
	\end{center}
	\[
	\sum_{\mathclap{\substack{(s_{i-1}, s_i)\cdots(s_n, d)}}} 
	W < W(s_{i-1}, r_i)+ W(r_i, r_{i+1}) + \ldots W(r_{m}, d) + D(d, d, \waypt_{pc})
	\] 
	\[
	\sum_{\mathclap{\substack{(s_{i-1}, s_i)\cdots(s_n, d)}}} 
	W \hspace{0.4cm}< \hspace{0.4cm}
	\sum_{\mathclap{\substack{(r_{i-1}, r_i)\cdots(r_m, d)}}} 
	W
	\]
	Adding $\Sigma_{(s, s_1)\cdots(s_{i-2},s_{i-1})}$ to both sides:
	\[
	\sum_{\mathclap{\substack{\pi_{pc}}}} 
	W < 
	\sum_{\mathclap{\substack{\sigma_{pc}}}} 
	W
	\] 
However, this contradicts the assumption (\ref{eq:wassumption}) that 
$\sigma_{pc}$ is the shortest path from $s_{pc}$ to $d_{pc}$. 
%Hence, proved, there is no path $\sigma_{pc} \in \paths^C(pc)$ such that $\forall w \in \waypt.~
%w \not\in \sigma_{pc}$. 
\end{proof}
\paragraph{Completeness}
In this setting, the notion of completeness is slightly complicated as
we solve a different variant of the path synthesis problem.
If the problem admit solution, there always exists a set of waypoint-compliant
paths on which the proposed technique returns appropriate weights.
However, if we might provide the algorithm with waypoint-compliant paths
for which no $W$ satisfy constraints (\ref{eq:waypoint}), although a waypoint-compliant
configuration exists. 
Formally, even if $\pi_{pc}$ is greater than the non-waypoint distance between the endpoints of $pc$, thus,
violating constraints (\ref{eq:waypoint}), the shortest path could still traverse one of the waypoints in $\waypt_{pc}$.