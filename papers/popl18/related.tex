\section{Related Work}\label{sec:related}
\paragraph{Centralized control} Programming routers for centralized
control has been an active area of research in recent times. One of
the first systems, RCP~\cite{rcp}, supported logically central BGP
configuration. The more recent Fibbing~\cite{fibbing} system provides
centralized control over distributed routing by creating fake nodes
and fake links to steer the traffic in the network through paths that
are not the shortest, similar to how \name uses static routes. 
However, the algorithms to solve the synthesis problem 
in this work is different from ours.  
Both approaches face the issue of  
forwarding loops in the network during failures. 
One merit of our work over Fibbing is that by using 
static routes for steering, we do not increase the 
traffic in the network, unlike the fake advertisements 
which are used in Fibbing. Fibbing also does not take into
account inter-domain routing.

%However, these fake advertisements can create
%forwarding loops in the network during failures, and the centralized
%controller has to respond to failures (response to failures can
%precomputed), thus making the controller a central point of
%failure. In contrast, our approach to distributed control plane
%synthesis can provide the same expressive power as Fibbing but avoids
%any centralized component by engineering the control plane parameters
%to match the input specifications.  


\paragraph{Configuration synthesis} 
ConfigAssure~\cite{configassure}
uses a combination of logic programming and SAT solving to synthesize
network configurations for security, 
functionality, performance and
reliability requirements specified as constraints; 
but it does not
support any notion of policy-resilience 
or hierarchical domain splitting.  Fortz
et. al~\cite{ospf-te} tackle the problem of optimizing OSPF weights
for performing traffic engineering, but their work is tailor-made
to just this specific problem.

Propane~\cite{propane, propaneat} tackles the specific problem of synthesizing
BGP configurations for concrete and abstract topologies 
to ensure network-wide objectives hold even under
failures. The policy language of Propane is suited to specify
preferences on paths and peering policies among different autonomous
systems. Propane translates policies
to a graph-based intermediate representation, which is then compiled
to device-level BGP configurations. It is unclear how to extend
Propane to incorporate domains, configuration complexity, or OSPF.

Synet~\cite{synet} tackles network-wide configuration
synthesis (Definition \ref{def:policycompliance}) in an elegant manner by
modeling the behavior and interactions of the routing protocols as a
stratified Datalog program, and using SMT to synthesize the Datalog
input such that the fixed point of the Datalog program (which
represents the network's forwarding state after the protocols have
converged) satisfies certain policies or path requirements. 
While both systems can take paths as input requirements,  
\name uses LP-solvers (which are faster and parallelizable) 
to tackle configuration
synthesis rather than SMT solvers.
Synet's approach does not deal with resilience, 
a key aspect we tackle 
in this paper.  Synet also does not attempt to 
minimize the number of static routes, which can cause
undesirable behaviors like routing loops.
Finally, Synet supports routers that run both
OSPF and BGP protocols and can be configured with 
static routes, which does not fit naturally into an hierarchical structure
where some routers only run OSPF and not BGP. In contrast, our MCMC algorithm can
look for dynamic domain assignments.
%
%An earlier iteration of our
%work considered synthesizing network configurations directly from
%policies using SMT; however, we found that encoding OSPF's shortest
%path-based constraints in SMT was time-consuming for modern SMT
%solvers~\cite{z3} and does not scale well to larger networks 

\paragraph{Policy languages} While \name uses \genesis 
to synthesize policy-compliant paths, in the future it could use
other policy language frameworks as a front-end
too; existing frameworks offer perhaps less rich policy support but better
performance. %% Other works on
%% centralized policy enforcement for SDN are Merlin~\cite{merlin} and
%% NetKAT~\cite{netkat}.
In Merlin~\cite{merlin}, data planes that adhere to policies expressed
using regular expressions are synthesized by first intersecting the
topology with the regular expressions appearing in the policies and
then encoding reachability in the intersected graph using mixed
integer linear programming (ILP).  Merlin supports min and max
bandwidth guarantees.

NetKAT~\cite{netkat} is a domain-specific language and logic for 
specifying and verifying network packet-processing functions
for SDN, based on Kleene algebra with tests (KAT). Semantically,
a NetKAT predicate and policy is a function that takes a packet
history and produces a set of (possibly empty) packet histories. 
NetKAT can be used to express certain network-wide policies like 
reachability, waypoints using regular expressions for describing the paths, 
and programs on virtual topologies; it uses
BDDs and symbolic automata to translate global programs to local
switch programs~\cite{netkatcompiler}.
Both Merlin and NetKAT  do not support link-isolation 
to produce edge-disjoint paths, which is needed for  
our waypoint-compliance algorithm presented in Section~\ref{sec:waypointres}.
As future work, we will consider 
how to apply these other front-ends to \name.

%% However, the NetKAT semantics
%% cannot be used to express policies based on hyperproperties
%% ~\cite{hyperproperties}, i.e., 
%% the packet processing function requires multiple packet histories
%% as input. Traffic engineering or isolated paths are policies
%% based on hyperproperties.

%% While Genesis supports richer policies, its SMT-based approach is slower in synthesis. As part of future work, we will consider how

%Fine-grained traffic engineering based on online demand/flow size estimation and 
%rapid rerouting is also crucial for datacenter workloads, and extending \name's
%TE policies to fine-grained timescales is subject of future work.
%Also, the performance
%of SMT solvers with optimization objectives is quite slow, and calls for 
%domain-specific techniques to speed up the synthesis. Also, datacenter
%networks are highly symmetrical, and this symmetry can be leveraged
%to speed up synthesis (similar to the work of Plotkin et. al~\cite{symmetry} to
%speed up network verification using symmetry). The main challenges of
%using symmetry in synthesis is considering two aspects of symmetry: network
%symmetry and policy symmetry. Also, our treatment of resilience synthesis
%is preliminary and future work will be geared towards synthesizing resilient
%forwarding planes incorporating capacity constraints and traffic engineering.
 
