%!TEX root=paper.tex

\section{From Waypoint Policies to Policy-Resilient Configurations}
\label{sec:waypointres}


The algorithm we presented in the previous section can generate configurations
for complex types of policies and at the same time can achieve good
connectivity resilience.
In this section, we present a different algorithm that, for a restricted set of policies,
can synthesize configurations with high \emph{policy-resilience}.
In particular, we focus on \emph{waypoint policies}
that require traffic to traverse
firewalls,
proxies, optimizers, and intrusion detection/prevention 
systems and
play a crucial role in enterprise and
datacenter networks for security, performance,  
and auditing.

\subsection{Problem Setup}

We consider policies of the form 
\texttt{$\lambda$: s >> $\waypt$ >> t}
where $\lambda$ is a destination IP subnet,  
$s$ and $t$ are the source and destination router respectively, 
and $\waypt \subseteq V$ is a set of waypoints (also called the \emph{waypoint set}). 
Each policy is mapped to a packet class and
we say that a configuration complies to a policy 
\texttt{$\lambda$: s >> $\waypt$ >> t}
 for the corresponding packet class
$pc$ if
at least one of the induced paths in $\paths^C(pc)$ 
traverses one waypoint $w \in \waypt$.\footnote{
The attentive reader will note that in our definition of waypoint-compliance 
we do not require all induced paths for $pc$ to traverse the waypoints. 
Our 
definition can be realized by two mechanisms: 1) A router
which has multiple routes for $\lambda$ will split the traffic
among these routes, and 2) The waypoint can mark certain fields in
the packet header, and the network operator can employ 
simple edge-based checks at the destination 
(for e.g., in the hypervisor) to 
detect if a packet traversed the waypoint or not. Thus, only
packets which traversed a waypoint-compliant path would be 
accepted and forwarded to the tenant. 
}
Here, $\waypt$
can be the set of physical replicas of,  for example, a firewall.


 Given a set of such waypoint policies, our goal is to
find a policy-compliant configuration 
$C=(T,\Theta,W,LP$, $IF,SR)$ 
with high policy-resilience. Ideally, for 
any single link failure, there should be at least 
one path to $\lambda$ which is waypoint-compliant. 
In the following section, we focus on 
intra-domain configurations,
and present an algorithm
that solves this problem when the network 
has a single domain.
When the network has multiple domains, 
we assume that all waypoints
belong to the same domain and
use our new algorithm in the domain 
which contains the waypoints. For the
other domains and the inter-domain configurations,
 we use the algorithms presented in \secref{sec:config-synthesis}. 
Our assumption that all waypoints lie in the same domain can be realized
through Network Function Virtualization~\cite{opennf, netbricks},
a technique for providing waypoint replication and 
 flexible waypoint placement in the network.


%\todo{explain why we don't do inter, seems like first para from next section
%should be somewhere here}

%
%\loris{somewhere describe how it's a two-phase approach
%and now all the paper is about 1-resilience so don't say
%1 resilience everywhere}

%\todo{not sure where following text belongs}
%Support for  
%middlebox traversals is commonly 
%provided by network operators.
%Network Function Virtualization (NFV) has been a 
%major driving force for providing reliability 
%guarantees under failures  
%for middlebox traversals~\cite{opennf, netbricks}. 
%Using NFV technology, operators can replicate 
%their middleboxes across the network for providing 
%resilience guarantees. 





%\subsection{Problem Statement}
%Let us consider an 
%unordered waypoint policy of the form: 
%\texttt{$\lambda$: s >> $\waypt$ >> t},
%where $\lambda$ is the destination IP subnet,  
%$s$ and $t$ are the source and destination router respectively, 
%and $\waypt \subseteq V$ is the waypoint set. This
%policy is mapped to a packet class $pc \in \nat$. 



\subsection{Intra-domain Policy-Resilient Waypoint Compliance}\label{sec:ospfwaypoint}
To achieve 1-resilient waypoint-compliance within the domain
in which the waypoints are located, we use \genesis's
support for isolation to generate two edge-disjoint\footnote
{The edge-disjoint paths will actually be vertex-disjoint (except the 
	source and destination) due to the destination-tree constraints
	in \genesis.} paths
which are waypoint-compliant with respect to $\waypt$. A single link
failure will not disable both these paths
and we could guarantee policy-resilience by 
guaranteeing that, under any link failure,
traffic to $\lambda$ 
traverses one of these two paths. However, enforcing the control plane to 
always route through one of the two paths is difficult and overly 
restrictive. 
Instead, we relax our constraints to guarantee that
the two paths are shorter than any non-compliant path in 
the network. The existence of two such paths still guarantees that,
under any link failure,
 traverses a waypoint. 
We now discuss the
constraints we use to generate 
a waypoint-compliant path, and then show
how to use the two edge-disjoint 
paths to provide policy-resilience. 


\subsubsection{Waypoint-Compliance Constraints}
In this section, we show how to generate constraints that
guarantee that a single waypoint-compliant path is shorter
than any path which does \emph{not} traverse a waypoint. 

We define $D_s^t(\waypt)$ to be the 
distance between $s$ and $t$ for path that \emph{do not}
 traverse any waypoint $w \in \waypt$.
We call  $D_s^t(\waypt)$ the \emph{non-waypoint distance}.
  We
   add constraints to represent these distances by
  considering a network topology where all the  
  waypoints $w \in \waypt$ are removed. Formally, for a topology 
 $T = (V,L)$, we add the following equations:
\begin{equation} \label{eq:wdistance}
\forall s, t \in V \setminus \waypt. ~\forall r \in N(s) \setminus \waypt.~
D_s^t(\waypt) \leq W_s^r + D_r^t(\waypt)
\end{equation}
%These constraints are similar to the constraints specified in
%\Cref{eq:distance}, but for an ``altered'' topology where all
%the waypoint routers are removed, and thus, any path in this 
%topology is not compliant with the waypoint policy. 
Using these equations,
$D_s^t(\waypt)$ is upper bounded by the actual shortest non-waypoint distance from $s$ to $t$.

\definecolor{orange}{RGB}{255, 157, 30} 

Given a destination tree $\xi_\lambda$, we can use
non-waypoint distances to enforce waypoint-compliance. Consider a 
router in $r$ such the waypoint $w \in \waypt$ is downstream to $r$ in $\xi_\lambda$---i.e.,
there exists a path from $r$ to $w$ in $\xi_\lambda$.
The path from $r$ must traverse through 
a waypoint in $\waypt$.  
Suppose, the sum of weights of the edges of
the path in the tree $W(r \rightarrow^+_{\xi_\lambda} R_\lambda)$  
is strictly smaller than the non-waypoint 
distance from $r$ to $R_\lambda$. This property ensures that 
the any path from $r$ to $R_\lambda$ that does not go through
any $w \in \waypt$ has a distance strictly greater than the path through
$\xi_\lambda$.
Hence, the path from $r$ to $R_\lambda$ is guaranteed to traverse
a waypoint $w \in \waypt$. These constraints are expressed as follows:
\begin{equation} \label{eq:waypoint}
\forall r' \in N(s) \setminus N_{\xi_\lambda}(r).~~ \sum_{\mathclap{\substack{l_i=(r_1,r_2)}}} 
W_{r_1}^{r_2} < 
W_{r}^{r'}+ D_{r'}^{R_\lambda}(\waypt) 
\end{equation}
%\begin{figure*}
%		\subfloat[Waypoint-compliant edge weights] {
%			\begin{tikzpicture}[shorten >=0.5pt,node distance=,on grid,auto,
%			square/.style={regular polygon,regular polygon sides=4}] 
%			\node[state] at (0,0) (s)  {$s$}; 
%			\node[state, fill=orange] at (1.8,1) (v1)  {$r_1$}; 
%			\node[state, fill=orange] at (1.8,-1) (v2)  {$r_2$}; 
%			\node[state] at (3.6, 0)(t) {$t$};
%			\node[state, rectangle] at (5, 0) (d1) {$\lambda$};
%			\path[->] 
%			(s) edge [red] node [black] {1} (v1)
%			(s) edge  node {1} (v2)
%			edge  node [above] {10} (t)
%			(v1) edge [red] node [black] {5} (t)
%			(v2) edge  node [black] {1} (t)
%			(t) edge [red, dashed] node {} (d1);
%			\end{tikzpicture}
%	}
%		\subfloat[Routing Loop] {
%			\begin{tikzpicture}[shorten >=0.5pt,node distance=,on grid,auto,
%			square/.style={regular polygon,regular polygon sides=4}] 
%			\node[state] at (0,0) (s)  {$s$}; 
%			\node[state] at (1.8,1) (v1)  {$r_1$}; 
%			\node[state] at (3.6, 0)(t) {$t$};
%			\node[state, rectangle] at (5, 0) (d1) {$\lambda$};
%			\path[->] 
%			(s) edge [red] node [black, sloped, anchor=center, above]
%			{$sr(\lambda)$} node
%			[black, sloped, anchor=center, below] {1} (v1)
%			edge  node [below] {5} (t)
%			(v1) edge [red] node [black] {50} (t)
%			(t) edge [red, dashed] node {} (d1);
%			\end{tikzpicture}
%		}
%			\compactcaption{. Example of routing loop caused by a static route.} \label{fig:ospfloop}
%			%\label{fig:ospfwayptexample}	
%\end{figure*}
\begin{figure}[!t] 
	\centering
	\subfloat[Waypoint-compliant edge weights]{
		\resizebox {0.33\columnwidth} {!} {
			\begin{tikzpicture}[shorten >=0.5pt,node distance=,on grid,auto,
			square/.style={regular polygon,regular polygon sides=4}] 
			\node[state] at (0,0) (s)  {$s$}; 
			\node[state, fill=orange] at (1.8,1) (v1)  {$r_1$}; 
			\node[state, fill=orange] at (1.8,-1) (v2)  {$r_2$}; 
			\node[state] at (3.6, 0)(t) {$t$};
			\node[state, rectangle] at (5, 0) (d1) {$\lambda$};
			\path[->] 
			(s) edge [red] node [black] {1} (v1)
			(s) edge  node {1} (v2)
			edge  node [above] {10} (t)
			(v1) edge [red] node [black] {5} (t)
			(v2) edge  node [black] {1} (t)
			(t) edge [red, dashed] node {} (d1);
			\end{tikzpicture}
		}}
		\hspace*{1cm}
		\subfloat[Routing Loop]{
			\raisebox{0.7cm}{\resizebox {0.33\columnwidth} {!} {
					\begin{tikzpicture}[shorten >=0.5pt,node distance=,on grid,auto,
					square/.style={regular polygon,regular polygon sides=4}] 
					\node[state] at (0,0) (s)  {$s$}; 
					\node[state] at (1.8,1) (v1)  {$r_1$}; 
					\node[state] at (3.6, 0)(t) {$t$};
					\node[state, rectangle] at (5, 0) (d1) {$\lambda$};
					\path[->] 
					(s) edge [red] node [black, sloped, anchor=center, above]
					{$sr(\lambda)$} node
					[black, sloped, anchor=center, below] {1} (v1)
					edge  node [below] {5} (t)
					(v1) edge [red] node [black] {50} (t)
					(t) edge [red, dashed] node {} (d1);
					\end{tikzpicture}
				}}}
				\compactcaption{Example of waypoint-compliant edge weights 
					and example of a routing loop caused by a static route.} \label{fig:ospfloop}
\end{figure}



These equations do \emph{not} guarantee that
router $r$ will forward traffic to  
$N_{\xi_\lambda}(r)$, but they still guarantee that the generated path is  waypoint-compliant. 
Consider the example in \Cref{fig:ospfloop}(a) where traffic for
$\lambda$ must traverse through one of the waypoints in $\{r_1, r_2\}$. 
In this case,  \genesis provided the path $s \rightarrow 
r_1 \rightarrow t$, but in the solution provided by Zeppelin, 
the shortest path is $s \rightarrow r_2
\rightarrow t$.
However, both of the waypoint-compliant paths are shorter 
than $s \rightarrow t$, the shortest non-waypoint path.

\begin{theorem}[OSPF Waypoint Soundness] \label{thm:waypoint}
	Given a set of waypoint paths $\{(\pi_{pc}, \waypt_{pc}) ~|~ pc\}$, if edge weights 
	$W$ satisfy constraints (\ref{eq:wdistance}) and (\ref{eq:waypoint}), for
	each packet class $pc$, the shortest path between its endpoints
	traverses one of the waypoints in $\waypt_{pc}$.
\end{theorem}
\iffull
\begin{proof}
	Let us assume there exists a packet class $pc$ with waypoint set $\waypt_{pc}$ 
	and waypoint path $\pi_{pc} = (s_{pc}, s_1)(s_1, s_2)\ldots (s_n, d_{pc})$, 
	such that  the 
	shortest path $\sigma_{pc}=(s_{pc}, r_1)$ $(r_1, r_2)\ldots (r_m, d_{pc})$ 
	is not waypoint-compliant---i.e.,  
	for every $i$, we have $r_i\not\in \waypt$.	
	Since $\sigma_{pc}$ is the shortest weighted path: 
	\begin{equation} \label{eq:wassumption}
	\sum_{\pi_{pc}} W \geq \sum_{\sigma_{pc}} W
	\end{equation}

	
	Let $s_i \not= r_i$ be the first point of divergence of the paths---i.e., for every $j<i$, $s_{j} = r_{j}$.
Constraints (\ref{eq:waypoint}) impose 
$(s_{i-1}, s_i)\ldots(s_n, d)$ to be 
shorter than any path which is not waypoint-compliant. 
Let us consider the neighbour router $r_1$ in $\sigma_{pc}$:
	\[
	\sum_{\mathclap{\substack{(s_{i-1}, s_i)\cdots(s_n, d)}}} 
	W < W(s_{i-1}, r_i)+ D(r_i, d, \waypt_{pc})
	\]
	Since $\sigma_{pc}$ does not traverse any waypoint in $\waypt_{pc}$,
	we use constraints (\ref{eq:wdistance}) 
	to expand the terms $D(r_k, d_{pc}, \waypt_{pc})$ for $i \leq k \leq m$:
	\[
	\sum_{\mathclap{\substack{(s_{i-1}, s_i)\cdots(s_n, d)}}} 
	W < W(s_{i-1}, r_i)+ W(r_i, r_{i+1}) + D(r_{i+1}, d,\waypt_{pc})
	\] 
	\begin{center}
		$\ldots$
	\end{center}
	\[
	\sum_{\mathclap{\substack{(s_{i-1}, s_i)\cdots(s_n, d)}}} 
	W < W(s_{i-1}, r_i)+ W(r_i, r_{i+1}) + \ldots W(r_{m}, d) + D(d, d, \waypt_{pc})
	\] 
	\[
	\sum_{\mathclap{\substack{(s_{i-1}, s_i)\cdots(s_n, d)}}} 
	W \hspace{0.4cm}< \hspace{0.4cm}
	\sum_{\mathclap{\substack{(r_{i-1}, r_i)\cdots(r_m, d)}}} 
	W
	\]
	Adding $\Sigma_{(s, s_1)\cdots(s_{i-2},s_{i-1})}$ to both sides:
	\[
	\sum_{\mathclap{\substack{\pi_{pc}}}} 
	W < 
	\sum_{\mathclap{\substack{\sigma_{pc}}}} 
	W
	\] 
However, this contradicts the assumption (\ref{eq:wassumption}) that 
$\sigma_{pc}$ is the shortest path from $s_{pc}$ to $d_{pc}$. 
%Hence, proved, there is no path $\sigma_{pc} \in \paths^C(pc)$ such that $\forall w \in \waypt.~
%w \not\in \sigma_{pc}$. 
\end{proof}
\fi
\paragraph{Completeness}
In this setting, the notion of completeness is slightly complicated as
we solve a different variant of the path synthesis problem.
If the problem admit solution, there always exists a set of waypoint-compliant
paths on which the proposed technique returns appropriate weights.
However, if we might provide the algorithm with waypoint-compliant paths
for which no $W$ satisfy constraints (\ref{eq:waypoint}), although a waypoint-compliant
configuration exists. 
Formally, even if $\pi_{pc}$ is greater than the non-waypoint distance between the endpoints of $pc$, thus,
violating constraints (\ref{eq:waypoint}), the shortest path could still traverse one of the waypoints in $\waypt_{pc}$.

\paragraph{Avoiding Routing Loops} \label{sec:loopavoidance}
\begin{wrapfigure}{r}{0.5\textwidth}
	\vspace{-7mm}
	\begin{minipage}{\linewidth}
		\begin{algorithm}[H]
			\begin{footnotesize} 
				\caption{OSPF Waypoint Synthesis with Static Routes}
				\label{alg:wayptunsat}
				\begin{algorithmic}[1]
					\Procedure{OSPF_W_SYNTH}{$\Pi$} 
					\State{$\Psi_D:$ Non-waypoint distance constraints (\ref{eq:wdistance})}
					\State{$\Psi_W:$ Waypoint constraints for $\Pi$ (\ref{eq:waypoint}) \label{line:waypoint}} 
					\State{$\Psi_R = \emptyset :$ Routing Loop avoidance constraints (\ref{eq:rla})}
					\State{$\Psi = \Psi_D \cup \Psi_W \cup \Psi_R$} 
					\While{$\Psi$ is \emph{unsat}} 
					\State{Extract unsat core $uc$ from LP Solver}
					\State{Pick static route $sr(sr_1, sr_2, \lambda)$ from $uc$ constraints}
					\State{Add $sr$ to static routes $SR$ \label{line:wstaticroute}}
					\State{$\Psi = \Psi \setminus (\Psi_W(sr_1, \lambda) \cup \Psi_R(sr_1,\lambda))$ \label{line:wremoveconstraint}}
					\State{$\Psi = \Psi \cup \Psi_R(sr_2,\lambda)$ \label{line:waddconstraint}} 
					\EndWhile
					\State{Obtain $W$ from solution model of $\Psi$}
					\Return{Configuration $C(W,SR)$} 
					\EndProcedure
				\end{algorithmic}
			\end{footnotesize}
		\end{algorithm}
	\end{minipage}
\end{wrapfigure}
\Cref{alg:wayptunsat} presents the OSPF 
synthesis algorithm with static routes.  
If one of the constraints in \Cref{eq:waypoint} is part of an unsatisfiable  
core, \name adds the static route $(r, N_{\xi_\lambda}(r))$ to
$SR(\lambda)$ to eliminate the equations at router $r$. 
However, since \Cref{eq:waypoint} does not guarantee that the path generated
by \genesis is indeed the shortest path, adding
static routes can lead to undesired behaviors like routing loops.  
Consider the example configuration shown in \Cref{fig:ospfloop}(b). 
Traffic for $\lambda$ at router $s$ is forwarded to $r_1$ because of the
static route at $s$. At router $r_1$, the shortest OSPF route to
$\lambda$ is $r_1 \rightarrow s \rightarrow t$ with weight 6 (compared 
to $r_1 \rightarrow t$ of weight 50). Thus, $r_1$ will send the 
packet back to $s$  causing a routing loop as $s$ will send
it back to $r_1$ and back and forth till the \emph{ttl} (time to live) of the
packet expires. 


Whenever \name adds a static route to resolve an unsat-core,
it removes the constraints pertaining to 
the static route and \emph{adds} new constraints to prevent
routing loops. 
%This
% approach ensures we do not add superfluous constraints for links
%that do not contain a static route in the final configuration 
%and cannot cause routing loops. 
Suppose \name added a static route $(sr_1, sr_2)$ for destination
$\lambda$ where $sr_2 = N_{\xi_\lambda}(sr_1)$ (because we do not add
static routes on any other links except $\xi_\lambda$). \name needs to
ensure that, for any router $r\in\xi_\lambda$ that does not 
lie in the downstream path $sr_2 \rightarrow^+_{\xi_\lambda} R_\lambda$, 
the shortest path from $sr_2$ to $R_\lambda$ does not traverse through 
$r$. For example, 
in \Cref{fig:ospfloop}(b), there is a routing loop because
 the shortest path from $r_1$ to $t$ goes through $s$, which is upstream of $r_1$. 

Formally, \name adds constraints to 
ensure that the weight
of path $(sr_2, r_1)(r_1, r_2)\ldots(r_n,R_\lambda)$
from $sr_2$ to $R_\lambda$ is 
strictly smaller than any path from $sr_2$ that traverses a
router $r'$ not in the downstream path from $sr_2$: 
\begin{equation} \label{eq:rla}
\forall r' \in \xi_\lambda. ~sr_2 \not\rightarrow_{\xi_\lambda}^+ r'. 
\hspace{0.6cm}\sum_{\mathclap{\substack{(sr_2, r_1)\ldots(r_n,R_\lambda)}}} 
W < D(sr_2, r') + D(r', R_\lambda) 
\end{equation}
We denote the routing loop avoidance 
constraints by $\Psi_R(sr_2, \lambda)$ 
(\Cref{alg:wayptunsat} line~\ref{line:waddconstraint}). 
If one of these constraints is part of an unsatisfiable
core, then there is a routing 
loop caused from $sr_2$. 
\name 
prevents
the occurrence of the routing loop from $sr_2$
by adding a static route 
$(sr_2, N_{\xi_\lambda}(sr_2))$ to $SR(\lambda)$. 
\name then removes the constraints pertaining to routing
loops at $sr_2$ (line~\ref{line:wremoveconstraint}) and 
adds loop-avoidance constraints $\Psi_R(N_{\xi_\lambda}(sr_2), 
\lambda)$ (line~\ref{line:waddconstraint}) for the new static
route and continues with the unsat-core learning algorithm. We will
now present the soundness result of the algorithm.

%!TEX root=paper.tex
\begin{theorem}[Soundness]
	Given a set of waypoint paths $\{(\pi_{pc}, \waypt_{pc}) ~|~ pc\}$,
	\Cref{alg:wayptunsat} outputs a configuration $C(W,SR)$ 
	such that, for every packet class $pc$, 
	there exists a path in $\paths^C(pc)$ that
	traverses one of the waypoints in $\waypt_{pc}$.
\end{theorem}
\kausik{This proof is quite long, and is complicated
by the fact that we dont enforce a particular path, 
and static routes, which force me to enumerate a lot of 
cases, I will think if there is an easier proof than the 
half-completed one I have below. But I am pretty sure that 
the theorem is correct.}
\begin{proof}
	Let us assume there exists a packet class $pc$ 
	with destination $\lambda$, 
	waypoint set $\waypt_{pc}$,
	and waypoint path $\pi_{pc} = (s_{pc}, s_1)(s_1, s_2)\ldots (s_n, d_{pc})$, 
	such that for $pc$, 
	there exists no path in $\paths^C(pc)$
	which is waypoint-compliant. There are two
	cases: either there exists no path in $\paths^C(pc)$ 
	or the paths do not traverse any waypoint in $\waypt_{pc}$.

	\paragraph{Case 1:} $\paths^C(pc) = \emptyset$---i.e., there
	is a routing loop caused by static routes (OSPF forwarding
	is loop-free).
	We denote the set of static routes for $\lambda$ by $SR(\lambda)$. Note that
	multiple packet classes can share the same destination IP
	and destination router, and we 
	construct a destination-based tree from these paths, denoted by 
	$\xi_\lambda$ 

	Let us denote the routing loop as 
	$(r_0, r_1) (r_1, r_2) \ldots (r_{n-1}, r_0)$.
	Since, the path is a loop with atleast one static route,
	there must exist one router $r_i$ in the loop
	such that $(r_{i-1}, r_i) \in SR(\lambda)$ and the
	next router in the loop $r_j$ which has a static route
	$(r_j, r_{j+1}) \in SR(\lambda)$ and $r_j$ is not downstream
	to $r_i$ in $\xi_\lambda$ (meaning there is no directed path 
	from $r_i$ to $r_j$ in $\xi_\lambda$). 

	Consider $r_i$. The path from $r_i$ to $r_j$ is 
	due to OSPF forwarding, therefore, there exists a 
	path $\sigma =
	(r_i, r_{i+1})\ldots(r_{j-1}, r_j)(r_j, l_1) \ldots (l_{o}, d_{pc})$
	which is the shortest path  from $r_i$ to $d_{pc}$. 

	Since static routes are only added on edges in $\xi_\lambda$, 
	let us denote the path from $r_{i}$ to $d_{pc}$ in 
	$\xi_\lambda$ as $(r_{i}, u_1)(u_1, u_2)\ldots(u_p, d_{pc})$. 
	Consider constraint (\ref{eq:rla}) which is added for static route 
	$(r_{i-1}, r_i)$ and non-downstream router $r_j$:
	\[
	\sum_{\mathclap{\substack{(r_{i}, u_1)\ldots(u_p, d_{pc})}}} 
	W < D(r_i, r_j) + D(r_j, d_{pc}) 
	\]
	Using distance constraints (\ref{eq:distance}), we can expand 
	$D(r_i, r_j)$ and $D(r_j, d_{pc})$ along the paths 
	$(r_i, r_{i+1})\ldots(r_{j-1}, r_j)$ and $(r_j, l_1) \ldots (l_{o}, d_{pc})$
	respectively. 
	\[
	\sum_{\mathclap{\substack{(r_{i}, u_1)\ldots(u_p, d_{pc})}}} 
	W < \hspace{0.8cm} \sum_{\mathclap{\substack{(r_i, r_{i+1})\ldots(r_{j-1}, r_j)}}}W \hspace{0.6cm} + 
	\hspace{0.6cm} \sum_{\mathclap{\substack{(r_j, l_1) \ldots (l_{o}, d_{pc})}}}W  
	\]
	However, this contradicts the assumption that $\sigma$ is 
	the shortest path from $r_i$ to $d_{pc}$. 

	\paragraph{Case 2:} All paths in $\paths^C(pc)
	\not=\emptyset$ are not waypoint-compliant. 
	Let $\sigma_{pc} = (s_{pc}, r_1)(r_1, r_2) \ldots
	(r_n, d_{pc})$ be one such path in $\paths^C(pc)$.
	Given the routing function $\route^C$ constructed from $SR$ and
	$W$ (\secref{sec:routingmodel}), let the first router where routing diverges from $\pi_{pc}$ be $s_p$---i.e.,  
	$s_{p+1} \not\in \route^C(s_{p}, \lambda_{pc})$. 
	\Cref{alg:wayptunsat} on line \ref{line:waypoint} adds the following
	constraint to ensure the sub-path of $\pi_{pc}$ 
	from $s_{p}$ to $d_{pc}$ is shorter than non-waypoint paths:
	\begin{equation} \label{eq:wuniq}
	\sum_{\mathclap{\substack{(s_{p}, s_{p+1})\cdots(s_n, d_{pc})}}} 
	W < W(s_{p}, r_{p+1})+ D(r_{p+1}, d_{pc}, \waypt_{pc})
	\end{equation}

	\Cref{alg:wayptunsat}
	only removes a subset of the waypoint and loop constraints
	constraints and 
	not the distance constraints (\ref{line:wremoveconstraint}). We further consider 
	two sub-cases: whether \Cref{alg:wayptunsat}
	removes Constraint (\ref{eq:wuniq}) or not. 

\begin{description}
	\item[Case 2A:]
	\Cref{alg:wayptunsat} does not remove Constraint (\ref{eq:wuniq}). 
	Thus, there is no static route $(s_p, s_{p+1})$ for
	$\lambda_{pc}$ (line \ref{line:wstaticroute}), and 
	OSPF-based forwarding occurs at $s_{p}$. 
	Since $\sigma_{pc}$ does not traverse any 
	waypoint in $\waypt_{pc}$,
	we use constraints (\ref{eq:wdistance}) 
	to expand the terms $D(r_k, d_{pc}, \waypt_{pc})$ for $p+1 \leq k \leq m$:
	\[
	\sum_{\mathclap{\substack{(s_{p}, s_{p+1})\cdots(s_n, d_{pc})}}} 
	W < W(s_{p}, r_{p+1})+ W(r_{p+1}, r_{p+2}) + D(r_{p+2}, d_{pc},\waypt_{pc})
	\] 
	\begin{center}
		$\ldots$
	\end{center}
	\[
	\sum_{\mathclap{\substack{(s_{p}, s_{p+1})\cdots(s_n, d_{pc})}}} 
	W < W(s_{p}, r_{p+1})+ \ldots + W(r_{m}, d_{pc}) + D(d_{pc}, d_{pc}, \waypt_{pc})
	\]
	By adding weights of $(s_{pc}, s_1)\ldots(s_{p-1}, s_p)$, we obtain that weight of $\sigma_{pc}$ 
	is greater than $\pi_{pc}$. If there is no static route on $\sigma_{pc}$, then, this is a contradiction
	as OSPF has a shorter path $\pi_{pc}$ to send traffic to.  If $\sigma_{pc}$ has a static route on the
	path, we consider the first divergence from the static route and use (\ref{eq:wdistance}) from that
	router. 
	\kausik{Very sketchy. }
	
	\item[Case 2B:]
	\Cref{alg:wayptunsat} removes Constraint (\ref{eq:waypoint}) 
	and $SR(s_p, \lambda_{pc}) = \{s_{p+1}\}$ (lines \ref{line:wstaticroute}-\ref{line:wremoveconstraint}). 
	Thus, $s_{p+1} \in \route^C(s_{p}, \lambda_{pc})$ as static routes 
	have the higher priority than OSPF. This contradicts our assumption
	that $s_{p+1} \not\in \route^C(s_{p}, \lambda_{pc})$. 
	\end{description}
	
	Hence, for every packet class, 
	there exists a path in the set of induced paths which traverses through one of the waypoints. 
	\end{proof}

\paragraph{Completeness}
The same observations as in Section~\ref{sec:waypoint-compliance-constraints}
also apply to this case and completeness can be achieved by running the algorithm
on different sets of paths.
%
%	Given a set of waypoint paths $\{(\pi_{pc}, \waypt_{pc}) ~|~ pc\}$,
%	\Cref{alg:wayptunsat} always terminates with a configuration $C$ where
%	each packet class is waypoint-compliant, the trivial solution being that of 
%	using static routes on each link in $\pi_{pc}$. 
	Moreover, similarly to what we observed in \Cref{alg:unsat}, \Cref{alg:wayptunsat} is not 
	guaranteed to find $C$ with a given bound on number of static routes.
Completeness is recovered by running \Cref{alg:unsat} multiple times, and by choosing different
unsatisfiable cores and static routes assignments. 


\subsubsection{Configurations for Two Paths} \label{sec:ospfresilience}
To guarantee policy-resilience, \name
uses \genesis to generate two waypoint-compliant 
edge-disjoint paths $\pi_1$ and $\pi_2$ from $s$ to $t$
and
adds the following constraints
 to ensure that the  weights 
of both paths are strictly shorter than 
the non-waypoint distance for $\lambda$. 
%Let us assume, for simplicity, that there are no static routes in the 
%OSPF domain. 
%\name  adds the following constraints
%constraints for both paths $\pi_1$ and $\pi_2$, 
%we enforce a \emph{partial order}: there are two paths to $\lambda$
%which are shorter than non waypoint distance, the 
%ordering of the weight of $\pi_1$ and $\pi_2$ is irrelevant. 
\begin{equation} \label{eq:resilience}
\sum_{\pi_1}W < D(s,t,\waypt) ~\wedge~ \sum_{\pi_2}W < D(s,t,\waypt) 
\end{equation}
Notice that the constraints do not enforce an order between the weights of $\pi_1$ and $\pi_2$.
Since the paths are edge-disjoint, if a single link fails, at most one 
of the paths is affected. Suppose
$\pi_1$ is affected, OSPF will not route traffic through a  
non-waypoint-compliant path, as the weight of $\pi_2$ is shorter 
than any path which is not waypoint-compliant. 
For a OSPF configuration with no static routes,
the synthesized configuration is policy-resilient. 

\begin{theorem}[Soundness]
	For a set of edge-disjoint waypoint paths $\{(\pi^1_{pc}, \pi^2_{pc}, \waypt_{pc}) ~|~ pc\}$, 
	if edge weights $W$ satisfy constraints (\ref{eq:wdistance}), (\ref{eq:waypoint}) and
	(\ref{eq:resilience}), 
	then for any arbitrary single link failure, 
	the shortest path between each packet class's 
	endpoints traverses one of the waypoints in $\waypt_{pc}$.
\end{theorem}
\kausik{Proof very similar to Thm 5.1, could be avoided}
\begin{proof}
	Let us assume there exists a packet class $pc$ with waypoint set $\waypt_{pc}$ 
	and edge-disjoint waypoint paths $\pi^1_{pc} = (s_{pc}, s_1)(s_1, s_2)\ldots (s_n, d_{pc})$, 
	and $\pi^2_{pc} = (s_{pc}, t_1)(t_1, t_2)\ldots (t_l, d_{pc})$; and there exists 
	a link $l$ failure 
	such that for $pc$, the 
	shortest path $\sigma_{pc}=(s_{pc}, r_1)$ $(r_1, r_2)\ldots (r_m, d_{pc})$ 
	is not waypoint-compliant---i.e.,  
	$\forall w \in \waypt_{pc}$, there doesn't exist a $p$ such that $r_p = w$. 
	
	A single link $l$ failure cannot disable both $\pi_{pc}^1$ and $\pi_{pc}^2$ as they are 
	edge-disjoint. 
	\paragraph{Case 1:} Link $l$ is not in path $\pi_{pc}^1$. Since, $\sigma_{pc}$ is 
	the shortest path: 
	\[
	\sum_{\sigma_{pc}}W \leq \sum_{\pi_{pc}^1}W
	\]
	Constraint (\ref{eq:resilience}) imposes the following: 
	\[
	\sum_{\pi_{pc}^1}W < D(s_{pc}, d_{pc}, \waypt_{pc}) 
	\]
	Since, $\sigma_{pc}$ does not traverse any waypoint, we can use 
	constraints (\ref{eq:wdistance}) for expanding $D(r_k, d_{pc}, \waypt_{pc})$ 
	for $1 \leq k \leq m$: 
	\[
	\sum_{\pi_{pc}^1}W < W(s_{pc}, r_1) + D(r_1, d_{pc}, \waypt_{pc}) 
	\]
	\begin{center}
	$\ldots$
	\end{center}
	\[
	\sum_{\pi_{pc}^1}W < W(s_{pc}, r_1) + W(r_1, r_2) + \ldots + W(r_m, d_{pc}) + D(d_{pc}, d_{pc}, \waypt_{pc}) = \sum_{\sigma_{pc}}W
	\]
	This contradiction the assumption that $\sigma_{pc}$ is the shortest path from $s_{pc}$ to $d_{pc}$.

	\paragraph{Case 2:} Link $l$ is not in path $\pi_{pc}^2$. Proof same as Case 1 for $\pi_{pc}^2$. 
	\paragraph{Case 3:} Link $l$ is not in path $\pi_{pc}^1$ or $\pi_{pc}^2$. Using \Cref{thm:waypoint}, 
	$\sigma_{pc}$ cannot be the shortest path as it does not traverse any waypoint in $\waypt_{pc}$. 
	
	Hence proved, under any arbitrary link failure, the shortest path between each packet class's 
	endpoints traverses one of the waypoints in $\waypt_{pc}$.
\end{proof}



\paragraph{Resilience with Static Routes}
\name only adds static routes on the paths obtained from \genesis.
Consider the example in \Cref{fig:ospfresexample} where 
$\pi_1=s\rightarrow r_0 \rightarrow r_1 \rightarrow t$ 
and $\pi_2=s\rightarrow r_2 \rightarrow t$ 
are two edge disjoint paths 
connecting the routers $s$ to $t$. In this example, 
$\pi_1$ is path selected by the configuration and it uses a static route
for $r_0 \rightarrow r_1$. 
Ideally, we want to ensure 
that if any link on $\pi_1$ fails, 
then the network must switch over to $\pi_2$. 
When link $r_1 \rightarrow t$ fails,
router $s$ forwards to $r_0$ (path weight 5: $ s \rightarrow r_0 \rightarrow r_2 \rightarrow t$)
over $r_2$ (path weight 6: $s \rightarrow r_2 \rightarrow t$). 
Because of the static route at $r_0$, traffic is forwarded
to $r_1$ which sends it back to $r_0$ through $r_3$, 
causing a routing loop. 


\begin{figure}
	\begin{tikzpicture}[shorten >=0.5pt,node distance=,on grid,auto,
	square/.style={regular polygon,regular polygon sides=4}] 
	\node[state] at (-0.6,1.2) (v3)  {$r_3$}; 
	\node[state] at (-2,0) (s)  {$r_0$}; 
	\node[state] at (-4,0) (src)  {$s$}; 
	\node[state, fill=orange] at (0.8,0) (v1)  {$r_1$}; 
	\node[state, fill=orange] at (-0.6,-1.2) (v2)  {$r_2$}; 
	\node[state] at (3, 0)(t) {$t$};
	\node[state, rectangle] at (4.4, 0) (d1) {$\lambda$};
	\path[->] 
	(s) edge [red] node [black, above] {$sr(\lambda)$} 
	node [black, below] {4} (v1)
	(s) edge [black] node {1} (v2)
	(src) edge [red]  node [black, below] {3 (1 $\checkmark$)} (v2)
	(src) edge [red] node [black] {1} (s)
	(v3) edge [blue] node [above, black] {1} (s)
	(v1) edge [blue] node [above, black] {1} (v3)
	(v1) edge [red, strike thru arrow] node [black] {1} (t)
	(v2) edge [red]  node [black, below] {3} (t)
	(t) edge [red, dashed] node {} (d1);
	\end{tikzpicture}
	\compactcaption{Example non-resilient configuration for the red paths provided by Genesis for $\waypt = \{r_1, r_2\}$. If link $r_1 \rightarrow t$ 
		fails, a routing loop is formed at $r_0 \rightarrow r_1 \rightarrow r_3 \rightarrow r_0$, no traffic reaches $t$. 
		The configuration is 1-resilient waypoint-compliant when the $s \rightarrow r_2$ weight is set to 1.}
	\label{fig:ospfresexample}
\end{figure}


In this case, the problem is caused by the fact that
path chosen by the configuration, $\pi_1$,
contains a static route in the middle of it.
Upon a failure, even though there exists another
a waypoint-compliant path, $r_2$, the configuration will not switch to it.
%but ``worse'' than
% $\pi_1$ and that  traverses a static route.
%Essentially there is a path that is longer than $\pi_1$,
%but shorter than $\pi_2$.
To avoid this problem, when this happens,
\name modifies the configurations to ensure that traffic is split.
When dealing with static routes,
the routes seen at $s$ may not 
correspond to weights of $\pi_1$ and $\pi_2$, 
thus simply 
the constraint $W(\pi_1) = W(\pi_2)$ does not ensure the 
traffic will split (see \Cref{fig:ospfresexample}).

Consider a path $\pi$ (different from $\pi_1$) for destination $\lambda$
that has  a  static route $(r_1, r_2)$ on one of its links.
\name would only have added this static route
if the shortest path from $r_1$ for location $\lambda$
was going through  a
router $r' \not= r_2$. 
%Instead, the static route circumvents 
%the OSPF shortest route at $r_1$ and \emph{steers} $\lambda$ 
%traffic to $r_2$ and not $r'$. 
Therefore, the weight of $\pi$,
$W(\pi)$, is greater or equal than the 
actual OSPF distance
between $s$ and $t$ corresponding to $\pi_1$. 
Suppose $\pi = l_1 \ldots l_k = (u_1, v_1) \ldots (u_k, v_k)$ 
and 
there exists a $\alpha \leq k$ such that 
$l_\alpha = (u_\alpha, v_\alpha)$  
is the first 
static route in $\pi$---i.e., 
$\forall \beta. ~1 \leq \beta < \alpha. ~(u_\beta, v_\beta) \not\in SR(\lambda)$.
Therefore, all routers between $s$ and $u_\alpha$ did
not require any static routes to steer traffic along a 
non-shortest path. 
Using this property we overapproximate
the
actual shortest OSPF distance  between $s$ to $t$ 
corresponding to $\pi_1$ 
as the sum of weights till $u_\alpha$ plus the distance
from $u_\alpha$ to $t$. In our running example, 
$u_\alpha = r_0$, and the distance between $s$ to $t$ 
is estimated as $W_s^{r_0} + D_{r_0}^t = 1 + (1 + 1 + 1)$
which is the shortest OSPF distance. 

To enforce splitting amongst
$\pi_1$ and $\pi_2$, we ensure that the weight 
of the path $W(\pi_2)$\footnote{
Note that ideally, the OSPF distance corresponding to 
$\pi_2$ would also depend on the first static route on
$\pi_2$, our distance variables are upper bounded by actual
distances and may not be equal, and thus, upper bounding 
those variables in these constraints do not ensure soundness.
As an approximation, we use $W(\pi_2)$ which is an upper bound
for the OSPF distance corresponding to 
$\pi_2$. 
} 
is smaller or equal to 
the estimated OSPF distance corresponding to $\pi_1$: 
\begin{equation}
	W(\pi_2) \leq \sum_{i=1}^{\alpha - 1} W_{u_i}^{v_i} + D_{u_\alpha}^t	
\end{equation}

Similar to the routing loop avoidance constraints, we add the
above constraints lazily when a static route is added to one of the
paths for $\lambda$. However, if one of these constraints is part of 
an unsatisfiable core, we cannot add a static route to eliminate 
this unsatisfiability. \name lazily eliminates these constraints from the
system of equations depending on whether these constraints are part of 
an unsat-core which cannot be eliminated by a static route, and thus,
the synthesized configuration may not achieve the best 
policy-resilience. However, 
judging by our experiments (\secref{sec:reseval}), 
this technique generates highly resilient configurations 
in practice. 
