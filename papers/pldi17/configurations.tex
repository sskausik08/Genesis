\section{Configuration Requirements}
We consider synthesis of inter-domain 
(for lack of a better word) routing 
configurations where each domain uses a 
IGP for intra-domain routing (like 
OSPF) and domains communicate among 
themselves using BGP. 
Routing inside a domain is controlled 
by synthesizing OSPF edge weights and 
route-filters. BGP is a path-vector
protocol and primarily selects routes by 
AS-PATH length. However, we can 
use BGP's powerful path selection metrics 
like local preferences to select 
paths with greater path lengths (local
preference value is considered first in 
path selection with multiple routes). 
OSPF has better convergence times than BGP,
thus, it is advantageous to use OSPF for 
routing in small domains and using BGP for
inter-domain routing. 

We introduce a framework of synthesizing
inter-domain configurations under certain
constraints. For example, due to scalability
concerns, OSPF domains must be smaller than 
a certain threshold for effective operations.
Similarily, the number of local preferences 
stored on a router is limited due to 
resource constraints. Thus, the routing 
configurations generated must adhere to 
these constraints. We describe the constraints
supported by the framework, points 1-4 are 
hard constraints we would ensure, while 5-8
are optimizations to synthesize better inter-domain
configurations. 

\begin{enumerate}
	\item \textbf{Number of domains ($N(D)$)}: 
	Can be used for administrative constraints 
	(each domain can be managed by different
	entities). 
	\begin{equation}
	N_l \leq N(D) \leq N_u
	\end{equation}

	\item \textbf{Size of domain ($|D|$)}: OSPF
	performs poorly as size of the domain increases
	(due to flooding of link-state updates). Thus,
	operators can specify bounds on the size of each
	domain.
	\begin{equation}
	S_l \leq |D| \leq S_u
	\end{equation}

%	\item \textbf{BGP-Compatibility}: Certain 
%	routers may not be suited to run BGP due to resource
%	constraints. Thus, the operator can specify if a 
%	router is non-BGP compatible. 

	\item \textbf{Optimization: Lines of configuration} 
	To enable path-based inter-domain routing, \name needs
	to set up static routes along the path, or configure BGP 
	variables like local preferences to 
	to ensure the routing emulates the input paths under no 
	failures. This would increase the size of the configurations,
	thus increasing the complexity of verifying correctness either 
	manually or using verification tools~\cite{batfish}. 

\end{enumerate}

\section{MCMC Sampling}
\name tackle the problem of minimization of LoC/route-filters
by stochastically searching for a domain assignment which 
minimizes the cost of the configuration with the help of Markov
Chain Monte Carlo (MCMC) sampling, specifically the Metropolis-Hasting
algorithm, a common technique used in different optimization 
problems~\cite{stoke}. 

MCMC sampling is a technique for drawing elements from a
probability density function in direct proportion to its value: 
regions of higher probability are sampled from more often than 
regions of low probability.
Applied in cost minimization problems,
MCMC sampling acts as an intelligent hill climbing method which
is robust for irregular cost functions and avoids convergence at 
local minimas. To transform an arbitrary cost function $c(\Theta)$, 
into a probability density function, we use the following 
technique~\cite{mcmcbook}:
\begin{equation}
	p(\Theta) = \frac{1}{Z}exp(-\beta * c(\Theta))
\end{equation}
$\beta$ is a positive constant and $Z$ is a partition function that
normalizes the distribution. Computing $Z$ is in general 
intractable, and the Metropolis-Hasting algorithm for 
generating Markov Chains can explore the probability density
function $p(\Theta)$ without computing the partition function. 
The intuition is as follows: given a current domain
assignment $\Theta$, the algorithm proposes a modified 
domain assignment $\Theta'$ as the next step. $\Theta'$
is accepted or rejected based on the Metropolis-Hasting
acceptance probability: 
\begin{equation}
Pr(\Theta \rightarrow \Theta') = min(1, \frac{p(\Theta')*q(\Theta| \Theta')}{p(\Theta)*q(\Theta'| \Theta)})
\end{equation}
$q(\Theta'| \Theta)$ denotes the proposal distribution from 
which $\Theta'$ is chosen given $\Theta$. If the proposal 
distributions is symmetric, i.e., 
$q(\Theta| \Theta') = q(\Theta'| \Theta)$, then the acceptance
probability is reduced to the simpler Metropolis ratio, which
can be computed directly from the cost function $c(\Theta)$:
\begin{multline}
Pr(\Theta \rightarrow \Theta') = min(1, \frac{p(\Theta')}{p(\Theta)}) \\
= min(1, exp(-\beta.(C(\Theta') - C(\Theta)))
\end{multline}
As we can observe from the acceptance probability, 
the algorithm will always accept a new proposal $\Theta'$
if its cost is lower than $\Theta$. If $\Theta'$ has a 
higher score than $\Theta$, the proposal will be 
probabilistically accepted depending on 
how far the cost of the proposals are. This ensures that 
the algorithm does not get stuck at local minimas and 
explore proposals with smaller differences in cost with 
higher probability. We describe the MCMC search procedure 
in Pseudocode~\cref{alg:mcmc}. 

\subsection{Hard Constraints}
\begin{algorithm}[t]
	\floatname{algorithm}{Pseudocode}
	\caption{MCMC}
	\label{dcsyn}
	\begin{algorithmic}[1] \label{alg:mcmc}
		\Procedure{MCMCSearch}{}
		\State{$\Theta \leftarrow$ random domain assignment}
		\State{$\overline{cf} = 0$ \hspace{2cm} [Worst Conf. overhead]}
		\State{$\overline{rf} = 0$ \hspace{2cm} [Worst route-filter est.]}
		\While{max iterations OR timeout}
		\State{$\gamma$ = \Call{Cost}{$\Theta$}}
		\State{$\Theta'$ = \Call{RandomChange}{$\Theta$}}
		\State{$\gamma'$ = \Call{Cost}{$\Theta'$}}
		\State{$Pr(\Theta \rightarrow \Theta')$ = 
			min$(1, exp(-\beta.(\gamma' - \gamma))$}
		\State{Set $\Theta$ = $\Theta'$ with 
			probability $Pr(\Theta \rightarrow \Theta')$}
		\EndWhile
		\EndProcedure
		
		\Procedure{Cost}{$\Theta$} 
		\State{$cf \leftarrow$ Configuration overhead (Static routes + \newline \hspace*{1.5cm} 
			BGP local preference entries + iBGP filters)}
		\If{$cf > \overline{cf}$} 
		\State{$\overline{cf} = cf$}
		\EndIf
		\State{$rf \leftarrow$ Number of diamonds with  \newline 
			\hspace*{1.3cm}  endpoints in same domain }
		\If{$rf > \overline{rf}$} 
		\State{$\overline{rf} = rf$}
		\EndIf
		\State{$\gamma$ = max($cf/\overline{cf},
			\alpha.rf/\overline{rf}$)  \newline
			\hspace*{3.5cm} + 0.1*min($cf/\overline{cf},
			\alpha.rf/\overline{rf}$)}
		\State{\Return $\gamma$}
		\EndProcedure
		
		\Procedure{RandomChange}{$\Theta$}
		\While{True}
		\State{$r \leftarrow$ pick random boundary router}
		\State{$\theta \leftarrow$ pick random neighbouring domain of $r$}
		\If{$|\Theta(r)| - 1 \geq l_\Theta \wedge |\theta| + 1 \leq u_\Theta$}
		\State{$\Theta' \leftarrow \Theta[r \rightarrow \theta]$} \hfill [$r$'s domain changed to $\theta$]
		\If{domains are continous}
		\State{\Return $\Theta'$}
		\EndIf
		\EndIf
		\EndWhile
		\EndProcedure
	\end{algorithmic}
	
\end{algorithm}

\subsection{Configuration Cost}
\todo{Explain path, as\_path transformation more formally}
\subsubsection{Static Routes} \label{sec:static}
As shown in \Cref{} (Refer to example of 
static routes in motivating example), we require static routes 
to enforce AS paths with loops. Static routes have the lowest 
administrative distance (1) by default~\cite{ad}, and will override BGP
and OSPF routes at a router, this feature can be used to 
exit and enter a domain multiple times. While static routes
do not reduce the resilience of the network (all routes are still
enabled, unlike route-filters), a network under flux will have
unpredictable routing behaviour, unlike with only OSPF and BGP
configured at the routers. 
Also, static routes have to be installed per-destination, thus increasing
the size of configurations drastically as number of policy paths increase.

Given a path $p$ for subnet $\lambda$ and 
the corresponding AS path $p_{as}
= as_1 \rightarrow as_2 \rightarrow \ldots \rightarrow as_m$ (where
$as_m = \Theta(\lambda)$), static
routes are required if $p_{as}$ has a AS-loop. 
To minimize
the number of static routes, we find 
the smallest $i \in [1,m]$ 
such that $\overline{p_{as}} = as_i \rightarrow as_{i+1}
\rightarrow \ldots \rightarrow as_m$ has no loops. 
Therefore, $\overline{p_{as}}$ is the longest AS-loop-free
subpath of $p_{as}$, and be can be enforced using BGP and OSPF. For the 
network path corresponding to AS path $as_1 \rightarrow as_2 
\rightarrow \ldots \rightarrow as_{i-1}$, we require static
routing rules for each next-hop. The static routing score
is the total number of static route hops required to enforce
the input paths.
\todo{Write about the BGP paths are extracted for the next phase}

\subsubsection{BGP Local Preference Entries}
\name uses BGP local preference to route traffic
for a particular subnet to the next AS via a specific 
gateway as per the input paths obtained. 
As shown in \Cref{} (Refer
to ex), if there are multiple exit gateways from an AS 
for a subnet $\lambda$, we require local preference entries at the 
gateways and iBGP filters among these gateways for $\lambda$.

For a domain $d$ and subnet $\lambda$, consider the set 
of paths to $\lambda$ exiting $d$ using BGP (and not statically
routed as described in \Cref{sec:static}). Let $E$ denote the
set of exit gateway routers for the paths of $\lambda$. 
If $|E| = 1$, if gateway $g \in E$ receives a route with 
strictly shortest AS path length (\Cref{alg:bgppathrules}) 
which enforces the paths
for $\lambda$, we do not need to configure local preference
entries on any BGP router in the domain for $\lambda$. If
the exit route chosen by the gateways for $\lambda$ does not 
enforce the paths, \name configures a local preference entry
for $\lambda$ at exit gateway, and thus, the exit route chosen
by BGP enforces the input paths for $\lambda$ in the domain.

If $|E| ~> 1$, multiple BGP routes must be redistributed to 
the OSPF domain. \\
E local prefs + E(E - 1) iBGP filters!

\todo{Changes to OSPF synthesis to ensure closest gateway}

\subsection{Route-Filter Cost}
\begin{figure}
	\centering
	\begin{tikzpicture}[shorten >=0.5pt,node distance=,on grid,auto,
	square/.style={regular polygon,regular polygon sides=4}] 
	\node[state] at (0,0) [fill=yellow] (s)  {$s$}; 
	\node[state] at (2,0.5) (v1)  {$r_1~~~~$}; 
	\node[state] at (2,-0.5) (u1) {$r_2~~~~$}; 
	\node[state, fill=black!30!green] at (4, 0)(t) {$t$};
	\path[->] 
	(s) edge  node {} (v1)
	edge  node {} (u1)
	edge [blue, dashed, bend right=45] node {} (t)
	edge [red, dashed, bend left=45] node {} (t)
	(u1) edge node {} (t)
	(v1) edge node {} (t);
	\path[-] (u1) edge [dashed] node[above] {} +(0,2);
	\path[-] (v1) edge [dashed] node[above] {} +(0,-2);
	\end{tikzpicture}
	\caption{Diamond}
	\label{fig:rfdiamond}
\end{figure}








