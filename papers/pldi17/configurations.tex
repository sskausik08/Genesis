\section{Configuration Requirements}
We consider synthesis of inter-domain 
(for lack of a better word) routing 
configurations where each domain uses a 
IGP for intra-domain routing (like 
OSPF) and domains communicate among 
themselves using BGP. 
Routing inside a domain is controlled 
by synthesizing OSPF edge weights and 
route-filters. BGP is a path-vector
protocol and primarily selects routes by 
AS-PATH length. However, we can 
use BGP's powerful path selection metrics 
like local preferences to select 
paths with greater path lengths (local
preference value is considered first in 
path selection with multiple routes). 
OSPF has better convergence times than BGP,
thus, it is advantageous to use OSPF for 
routing in small domains and using BGP for
inter-domain routing. 

We introduce a framework of synthesizing
inter-domain configurations under certain
constraints. For example, due to scalability
concerns, OSPF domains must be smaller than 
a certain threshold for effective operations.
Similarily, the number of local preferences 
stored on a router is limited due to 
resource constraints. Thus, the routing 
configurations generated must adhere to 
these constraints. We describe the constraints
supported by the framework, points 1-4 are 
hard constraints we would ensure, while 5-8
are optimizations to synthesize better inter-domain
configurations. 

\begin{enumerate}
	\item \textbf{Number of domains ($N(D)$)}: 
	Can be used for administrative constraints 
	(each domain can be managed by different
	entities). 
	\begin{equation}
	N_l \leq N(D) \leq N_u
	\end{equation}

	\item \textbf{Size of domain ($|D|$)}: OSPF
	performs poorly as size of the domain increases
	(due to flooding of link-state updates). Thus,
	operators can specify bounds on the size of each
	domain.
	\begin{equation}
	S_l \leq |D| \leq S_u
	\end{equation}

	\item \textbf{BGP-Compatibility}: Certain 
	routers may not be suited to run BGP due to resource
	constraints. Thus, the operator can specify if a 
	router is non-BGP compatible. 

	\item \textbf{Optimization: Number of local-preference entries} 
	To enable path-based inter-domain routing, BGP routers
	would store different local preferences for destinations
	to ensure the routing emulates the input paths under no 
	failures, which would increase the size and thus complexity
	of configurations to manually or verify using software tools
	like Batfish~\cite{batfish} would increase. 

	\item \textbf{Optimization: Number of BGP speakers}: 
	Minimize the number of BGP speaking routers.

	\item \textbf{Optimization: Equal sized domains}:
	Routers equally divided across the domains will ensure 
	each domain performs well with internal OSPF routing (no 
	disproportionately-sized domain). 

	\item \textbf{Optimization: Geographic proximity}:
	The network topology can be augmented with link distances. 
	This can be used to ensure that a domain's physical diameter
	is minimized 
\end{enumerate}

\section{MCMC Cost Function}
Let us denote a mapping $\Gamma: R \rightarrow D$ maps each 
router $r \in R$ to a domain $d \in D$. The cost function 
$c(\Gamma)$ is defined as follows: 
\begin{equation}
	c(\Gamma) = h(\Gamma) + \beta \times o(\Gamma)
\end{equation}