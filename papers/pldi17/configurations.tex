\section{Synthesising Domain Assignments}
\label{sec:synth-dom-ass}

In this section, we present an algorithm for 
solving the path-compliance when a domain assignment is not given to us.
The algorithm searches the space of possible domain assignment $\Theta$ to find
one that meets all configuration policies and has good quantitative properties.
In particular, we will focus our discussion on maximizing resiliency while minimizing
configuration overhead---e.g., number of static routes and BGP configuration variables. 

Formally, we are given a set of paths $\Pi$,
a topology $T=(V,L)$,
a domain-assignment function $\Theta$ such that $|\{\Theta(r) \mid r\in V\}|>1$, 
and we want to find functions
$LP$, $W$, $RF$,  and $\Theta$ such that
$C=(T,W,RF,LP,\Theta)$ and
$\paths^C(PC) = \Pi$.
We call this problem the \emph{domain placement problem}.

\loris{all definitions are missing static routes, please fix}
We first show that even the simplest variant of the domain assignment problem,
in which route filters are disallowed, is NP-complete.
\begin{theorem}
Given a set of paths $\Pi$,
a topology $T=(V,E)$, 
a route-filter function $RF$ such that $\sum_{r\in V} |RF(r)|=0$,
the problem of finding 
a domain assignment $\Theta$, 
a local preference function $LP$,
and edge weights $W$
 such that
$C=(T,W,RF,LP,\Theta)$ and
$\paths^C(PC) = \Pi$  is NP-complete.
\loris{messy}
\end{theorem}

Given the complexity of this problem, we opt for a greedy
(\kausik{stochastic?}) approach
to  search the space of domain
assignments. 
\name uses Markov
Chain Monte Carlo (MCMC) sampling methods, specifically the Metropolis-Hasting
algorithm, a common technique used in different optimization 
problems~\cite{stoke}. 
We first present the general structure of our searching algorithm and 
then describe the cost function we use to guide the search.

\subsection{Searching Assignments with MCMC}
MCMC sampling is a technique for 
drawing elements from a
probability density function in direct proportion to its value.
%Intuitively, the search samples regions of higher probability more often than 
%regions of low probability. 
Intuitively, the probability distribution can be viewed as a (potentially infinite) 
connected undirected graph
where nodes are elements of the probability distributions
and edges denotes possible next steps in the search---i.e.,
which nodes we can visit next.
Each node is associated with a cost so that nodes with higher costs have lower
probabilities associated with them.

In our case, each node is a domain assignment $\Theta$
with some cost $c(\Theta)$. 
We will discuss how our cost function is implemented in the next sections.
The cost function $c$ can be transformed 
into a probability density function in the following way~\cite{mcmcbook}:
\begin{equation}
	p(\Theta) = \frac{1}{Z}exp(-\beta * c(\Theta))
\end{equation}
where $\beta$ is a positive constant and $Z$ is a partition function \loris{over what?} that
normalizes the distribution. Computing $Z$ is in general 
intractable, and the Metropolis-Hasting algorithm 
explores the search graph representing $p$ without computing the partition function. 
Intuitively, given a current domain
assignment $\Theta$, the algorithm proposes a modified 
domain assignment as the next step  by picking a neighbour
$\Theta'$ of $\Theta$ in the graph. A coin is then toss with
probability 
\begin{equation}
Pr(\Theta \rightarrow \Theta') = min(1, \frac{p(\Theta')*q(\Theta| \Theta')}{p(\Theta)*q(\Theta'| \Theta)})
\end{equation}
If the result is head, $\Theta'$
becomes the current node, otherwise $\Theta$ remains the current node.
$q(\Theta'| \Theta)$ denotes the proposal distribution from 
which $\Theta'$ is chosen given $\Theta$. If the proposal 
distributions is symmetric, i.e., 
$q(\Theta| \Theta') = q(\Theta'| \Theta)$, then the acceptance
probability is reduced to the simpler Metropolis ratio, which
can be computed directly from the cost function $c(\Theta)$:
\begin{multline}
Pr(\Theta \rightarrow \Theta') = min(1, \frac{p(\Theta')}{p(\Theta)}) \\
= min(1, exp(-\beta.(C(\Theta') - C(\Theta)))
\end{multline}
The algorithm will always accept a new proposal $\Theta'$
that has cost lower than $\Theta$. If $\Theta'$ has a 
higher cost than $\Theta$, the proposal will be 
accepted with probability depending on 
how far the costs of $\Theta$ and $\Theta'$ are. This ensures that 
the algorithm does not get stuck at local minimas, but 
explores proposals with smaller differences in cost with 
higher probability. We describe the MCMC search procedure 
in Pseudocode~\cref{alg:mcmc}. 


\begin{algorithm}[t]
	\floatname{algorithm}{Pseudocode}
	\caption{MCMC}
	\label{dcsyn}
	\begin{algorithmic}[1] \label{alg:mcmc}
		\Procedure{MCMCSearch}{}
		\State{$\Theta \leftarrow$ random domain assignment}
		\State{$\overline{cc} = 0$ \hspace{2cm} [Worst Conf. overhead]}
		\State{$\overline{rc} = 0$ \hspace{2cm} [Worst route-filter est.]}
		\While{max iterations OR timeout}
		\State{$\gamma$ = \Call{Cost}{$\Theta$}}
		\State{$\Theta'$ = \Call{RandomChange}{$\Theta$}}
		\State{$\gamma'$ = \Call{Cost}{$\Theta'$}}
		\State{$Pr(\Theta \rightarrow \Theta')$ = 
			min$(1, exp(-\beta.(\gamma' - \gamma))$}
		\State{Set $\Theta$ = $\Theta'$ with 
			probability $Pr(\Theta \rightarrow \Theta')$}
		\EndWhile
		\EndProcedure
		
		\Procedure{Cost}{$\Theta$} 
		\State{$cc \leftarrow$ Configuration overhead (Static routes + \newline \hspace*{1.5cm} 
			BGP local preference entries + iBGP filters)}
		\If{$cc > \overline{cc}$} 
		\State{$\overline{cc} = cc$}
		\EndIf
		\State{$rc \leftarrow$ Number of diamonds with  \newline 
			\hspace*{1.3cm}  endpoints in same domain }
		\If{$rc > \overline{rc}$} 
		\State{$\overline{rc} = rc$}
		\EndIf
		\State{$\gamma$ = max($cc/\overline{cc},
			\alpha.rc/\overline{rc}$)  \newline
			\hspace*{3.5cm} + 0.1*min($cc/\overline{cc},
			\alpha.rc/\overline{rc}$)}
		\State{\Return $\gamma$}
		\EndProcedure
		
		\Procedure{RandomChange}{$\Theta$}
		\While{True}
		\State{$r \leftarrow$ pick random boundary router}
		\State{$\theta \leftarrow$ pick random neighbouring domain of $r$}
		\If{$|\Theta(r)| - 1 \geq l_\Theta \wedge |\theta| + 1 \leq u_\Theta$}
		\State{$\Theta' \leftarrow \Theta[r \rightarrow \theta]$} \hfill [$r$'s domain changed to $\theta$]
		\If{domains are continous}
		\State{\Return $\Theta'$}
		\EndIf
		\EndIf
		\EndWhile
		\EndProcedure
	\end{algorithmic}
\end{algorithm}

\subsection{Domain Assignment Cost}
In the previous section, we presented the general structure of our search algorithm,
but we did not specify how the cost function $c(\Theta)$
is computed. Intuitively, the cost function should capture
how far we are from meeting all the given control-plane policies.
For example, if we are trying to obtain good resiliency,
domain assignments with better resiliency should have lower costs.
Similarly, a configuration with lower number of static routes is better
than one with more number of static routes. Therefore, we also consider 
the configuration overhead in terms of static routes and BGP configuration
variables in our cost function.
The techniques presented in Section~\ref{sec:synth-multi} provide a way to 
synthesize a configuration for each domain assignment and therefore compute
very precise costs for each domain assignment. 
However, 
we want MCMC to explore as many assignments as possible,
therefore 
synthesizing the actual configurations for
each domain assignment is infeasible (Theorem~\ref{thm:ospfsynth} shows that, 
even for a single domain, computing accurate costs---e.g., 
number of required route filters---is an NP-complete problem).


\minisection{Approximating number of route filters}
We present a heuristic technique for estimating the number of route 
filters required by a particular domain assignment and use this 
to estimate the resiliency of a domain assignment---i.e., a domain assignment with fewer 
route-filters is likely to have higher resiliency. 

\loris{in preliminaries early on define that we use $l\in\pi$ to say path $\pi$ contains a link $l$.}

We say that two paths $\pi=(r_1,r_2)\cdots (r_{n-1},r_n), \pi'=(r_1',r_2')\cdots (r_{n-1}',r_n')$ with destinations $\lambda$ and $\lambda'$
form an $(r_i, r_j, \lambda, \lambda')$\emph{-diamond} if and only if
there exists $i,i',j$, and $j'$ such that $i<j-1$, $i'<j'-1$, and
\begin{multline}
r_i{=}r_{i'}' \wedge  r_j{=}r_{j'}' \wedge  \forall i{<}k{<}j.~\forall i'{<}k'{<}j'.~r_{k}{\neq} r_{k'}'  
\end{multline}
Intuitively, a diamond is the smallest structure formed by two
paths intersecting at $r_i$ and $r_j$ with edge-disjoint paths in 
between these routers. 

If the paths $\pi$ and $\pi'$ completely lie in the same domain,
the presence of a diamond 
implies that OSPF needs to compute two different shortest paths between $r_i$ and $r_j$, 
which means that
at least one route-filter is required.
On the other hand, if $r_i$ and $r_j$ lie in
different domains and this diamond
was the only structure causing two paths to reach the same destination, \name could find OSPF weights for
both the domains without using route-filters. 
\loris{not sure if following sentence needed}
In the limiting
case where each router is a separate domain of size 1,
no route-filters are required, and the entire 
network can be configured using BGP. 

Before  starting the MCMC search, \name precomputed
the set of all diamonds induced by the paths $\Pi$. 
At each MCMC iteration, \name computes the cost of a
domain assignment $\Theta$ by counting for how many $(r_i, r_j, \lambda, \lambda')$-diamonds
$\Theta(r_i) = \Theta(r_j)$. This count $rc$ is our route-filter cost and is used to estimate the number of route-filters
required to eliminate the diamonds.
Notice that two different diamond that share an edge could be resolved
by placing a single filter on the shared edge, whereas our estimated route-filter cost 
would be 2. 
\loris{I think this is true. 
Intuitively, our algorithm is similar to the greedy algorithm for solving vertex cover~\cite{}, and 
using a similar argument,
we can show that our estimate produces at most twice as many route filters as the ones 
needed to eliminate all diamonds.}

While diamonds in the same domain definitely require route filters, there might be
sets of paths that do not contain diamonds but that require route filters. 
These cases are not
taken into account in our estimate. 
On the other hand, our diamond-based estimate
can be computed efficiently and 
our experiments show that reductions in the route-filter cost induced by our estimate
lead to reductions in number of filters (~\Cref{some experiment}). 

\minisection{Cost of Configuration Overhead} 
Given a domain assignment $\Theta$, the exact number of static routes,
BGP local preferences, and iBGP filters can be computed 
efficiently  using the techniques from \Cref{sec:synth-multi}).
We can use the sum of these three quantities to quantify the
configuration cost $cc$.\loris{is the content of the footnote consistent with our policy lang}\footnote{
	Operators can specify relative weights of each overhead, for e.g.--an
	operator may want to reduce only static routes.}.

\minisection{Overall Cost Function} 
\loris{rewrite this after policy language already takes this into account. 
The it will be all about estimating the quantities in the objective and the aggregation will be for free.
This para will go.}
Finally, we need to combine the
route-filter cost ($rc$) and configuration cost ($cc$) 
based on the preference specified by the operator. 
Interestingly, these two quantities are inversely related. 
If the whole network is a single OSPF domain, $rc$ is maximum, while
$cc$ is 0. Similarly, if all routers are in different domains
$cc$ is maximum while $rc$ is 0. 
If the operator requires to
 jointly minimize both these quantities, we
will use a combined cost of the form $\rho=max(RC, \alpha*CC)$ where $\alpha$ is a tunable parameter
that can assign weight to the two quantities. Here,
$RC$ and $CC$ are normalized versions of the costs
obtained by dividing $rc$ and $cc$  by the worst costs seen during the MCMC search.
Therefore, initially $\rho=1$ and later on $\rho$ is the improvement of the current configuration over
the worst in terms of either configuration overhead or route-filter cost.
\loris{I think the above trick breaks MCMC by making you stuck in local optima.
What happens if we remove it? Also unnecessary detail.}