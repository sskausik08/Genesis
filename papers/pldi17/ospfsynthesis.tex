\section{Synthesising OSPF Configurations} \label{sec:synthesis}
In this section, we present an algorithm for 
solving the path-compliance problem for 
single-domain OSPF networks
---i.e., $|\{\Theta(r) \mid r\in V\}|=1$.

\loris{some text about fewer RF= higher resiliency}

Before we present our technique and its intricacies,
we justify the complexity of our approach by showing that
the problem of synthesizing an OSPF configuration
that contains an optimal number of route-filters is NP-complete.

\begin{theorem}[OSPF synthesis]
Given a set of paths $\Pi$,
a topology $T=(V,E)$,
a domain-assignment function $\Theta$, such that $|\{\Theta(r) \mid r\in V\}|=1$,
a number $n\geq 0$,
the problem of finding 
an $LP$, $W$, and $RF$,  such that
$C=(T,W,RF,LP,\Theta)$,
$\paths^C(PC) = \Pi$,
and 
$\sum_{\lambda\in\Lambda} |RF(\lambda)|=n$ is NP-complete.
We call this problem the \emph{OSPF synthesis problem}.
\end{theorem}
\iffull
%!TEX root = paper.tex
\begin{proof}
We show that the decision version of the minimum 
vertex cover problem, i.e., there exists a vertex cover
of size $ \leq k$, which is NP-complete, 
reduces to finding a set of static routes 
of size $ \leq k$ \
and OSPF weights for a network with only one domain. 
The latter is also in NP, so after the reduction we 
can conclude that it is also NP-complete.

Let $G = (V,E)$ be an instance of the 
minimum vertex cover problem. A set of
vertices $VC \subseteq V$ is the vertex cover
if $\forall (v_1, v_2) \in E. ~v_1 \in VC \vee v_2 \in VC$. 

We now show how to construct a topology $T=(S,L)$ 
and a corresponding set of paths $\Pi$ that can be enforced 
by configuration C which requires static routes $SR$ such that $|SR| \leq k$  
if and only if the corresponding $VC(SR)$ is a vertex cover of 
the graph $G$ and $|VC(SR)| \leq k$.

\paragraph{Construction.}
For every vertex $v \in V$: add a vertex $r_v$.
For every edge $(u,v) \in E$: add two vertices $s_{uv}$
and $t_{uv}$ to $S$. Add edges
connecting $s_{uv} \rightarrow r_{u}$, $s_{uv} \rightarrow r_{v}$,
$r_{u} \rightarrow t_{uv}$ and $r_{v} \rightarrow t_{uv}$. 
\Cref{fig:rfcomplexity} illustrates this construction.
\begin{figure}[H]
	\centering
	\begin{tikzpicture}[shorten >=0.5pt,node distance=1.5cm,on grid,auto,
	square/.style={regular polygon,regular polygon sides=4}] 
	\node[state] at (0,0) (s)  {$s_{uv}$}; 
	\node[state] at (3,-1) (v)  {$r_v$}; 
	\node[state] at (3,1) (u)  {$r_u$}; 
	\node[state] at (6,0) (t) {$t_{uv}$}; 	
	\node[state] (s1) [below=of s] {$s_{vw}$}; 	
	\node[state] (t1) [below=of t] {$t_{vw}$}; 	
	\path[-] 
	(s) edge  node {} (v)
	edge  node {} (u)
	edge [blue, dashed, bend right=40] node {} (t)
	edge [red, dashed, bend left=40] node {} (t)
	(u) edge node {} (t)
	(v) edge node {} (t)
	edge node {} (t1)
	edge node {} (s1);
	\path[->]
	(s)
	edge [blue, dashed, bend right=40] node {} (t)
	edge [red, dashed, bend left=40] node {} (t);
%	\path[-] (s) edge node[above] {} +(1,-0.1);
%	\path[-] (t1) edge node[above] {} +(-1,-0.1);
	\end{tikzpicture}
	\caption{Construction for reduction to Vertex Cover.}
	\label{fig:rfcomplexity}
\end{figure}
If there is another edge $(v,w) \in E$, then
$s_{vw}$ and $t_{vw}$ have an edge connecting to $r_v$ (shown
in \Cref{fig:rfcomplexity}). 

For each edge $(u,v) \in E$, we add two paths in $\Pi$: 
$s_{uv} \rightarrow r_u \rightarrow t_{uv}$
for destination host $d_u$ and 
$~s_{uv} \rightarrow r_v \rightarrow t_{uv}$ 
for destination host $d_v$.
(dashed paths in \Cref{fig:rfcomplexity}). 

We now prove that if there exists a set of static routes
$SR$ such that $|SR| \leq k$ such that the resulting configurations
are path-compliant for $\Pi$, then there exists a vertex cover $VC$
of $G$ such that $|VC| \leq k$. 

For each static route $sr \in SR$, the static route
has to be placed at the source, either going to $r_u$
or $r_v$ (structure of topology $T$). 
We construct a set $VC(SR)$ by adding the vertex $v$
based on the endpoints of each static route $sr \in SR$.
To show that $VC(SR)$ is a vertex cover of $G$, we first
prove \Cref{lemma:diamond}.

\begin{lemma} \label{lemma:diamond}
	 For each diamond formed by the input paths, atleast 1 
	 static route on one of the edges of the paths of the diamond 
	 is required to find a valid solution to the
	 OSPF edge weights.  
\end{lemma}

\begin{proof}
Given two paths $\pi_1$ and $\pi_2$ for destinations 
$d_u$ and $d_v$, we define these paths form a diamond
if these paths intersect at two routers ($s_{uv}$ and $t_{uv}$) 
without any common router in between. 
Consider the following diamond % in \Cref{fig:diamond}
constructed by paths $\pi_1$: $s_{uv} \rightarrow r_u \rightarrow t_{uv}$ 
for destination $d_u$ and $\pi_2$: $s_{uv} \rightarrow r_v \rightarrow t_{uv}$ 
for destination $d_v$. Let us assume there exists a solution 
for the OSPF edge weights without any static routes. 

We add the following inequality 
to make $\pi_1$ is the shortest 
path from $s_{uv}$ to $t_{uv}$ by
ensuring
 $\pi_1$ is shorter than the
path from $s_{uv}$ to $t_{uv}$ via $r_v$: 
\begin{equation} \label{eq:diamond1}
	W(s_{uv},r_u) + W(r_u, t_{uv}) < W(s_{uv}, r_v) + W(r_v,t_{uv})
\end{equation}
Since $\pi_2$ is also the shortest path from $s_{uv}$ 
to $t_{uv}$, the linear inequality added is:
\begin{equation}  \label{eq:diamond2}
W(s_{uv},r_v) + W(r_v, t) < W(s_{uv}, r_u) + W(r_u,t_{uv})
\end{equation}
Since there are no static routes on the edges
of $\pi_1$ and $\pi_2$, none of the above equations are 
eliminated. 
Adding equations \ref{eq:diamond1} and  \ref{eq:diamond2} 
yields the inequality $0 < 0$, which is inconsistent 
and therefore, no solution to 
the edge weights exists for this system of equations, 
which contradicts our assumption. Therefore,
for each diamond formed by the input paths, atleast 1 
static route on one of the edges of the paths of the diamond 
is required to find a valid solution to the
OSPF edge weights.  
\end{proof}

For every edge $(u,v) \in E$, the constructed paths from 
$s_{uv}$ to $t_{uv}$ form a diamond. Thus, by Lemma~\ref{lemma:diamond}, 
the diamond created by the paths corresponding to each edge in $G$ 
requires atleast one static route to eliminate
the inconsistency caused by the diamond. If a static route's endpoints
contains $r_u$, we put $u$ in $VC(SR)$ and similarily for $r_v$. 
Edge $(u,v)$ is covered since atleast one static route is added,
thus, atleast one of $\{u,v\}$ is in $VC(SR)$.  
Thus, if $SR$ eliminates all diamond inconsistencies
to find a solution to the OSPF weights, the corresponding set
$VC(SR)$ covers all edges in $E$. Therefore, $VC(SR)$ is a vertex
cover. 

Thus, by finding a set of static routes $SR$ such that $|SR| \leq k$
such that all the diamond inconsistencies are eliminated, and there
exists OSPF weights $W$ such that the configurations forward traffic
along $\Pi$, we can find a vertex cover $VC$ for graph $G$ such that
$|VC| \leq k$. 

This transformation is polynomial, the constructed 
network topology $T$ has $|V| + 2|E|$ nodes, 
$4|E|$ links and $2|E|$ paths. Therefore, OSPF
configuration synthesis with number of static routes $\leq k$ is
NP-complete. Thus, OSPF synthesis with minimal number of 
static routes is NP-hard. 
\end{proof}

\else
The reduction is from \loris{what problem?}.
\fi

We first show how to solve the OSPF synthesis problem when $n=0$---i.e., we are not allowed to use route filters---and
then extend our technique to handle route filters.

\begin{figure}
	\centering
	\subfloat[Edge Weights]{
		\raisebox{0.5cm}{\resizebox {0.5\columnwidth} {!} {
				\begin{tikzpicture}[shorten >=0.5pt,node distance=,on grid,auto,
				square/.style={regular polygon,regular polygon sides=4}] 
				\node[state] at (0,0) (s)  {$s$}; 
				\node[state] at (1.8,1) (v1)  {$r_1$}; 
				\node[state] at (3.6, 0)(t) {$t$};
				\node[state, rectangle] at (5, 0) (d1) {$\lambda$};
				\path[->] 
				(s) edge node {1} (v1)
				edge  node {5} (t)
				edge [red, dashed, bend left=90] node {} (t)
				(v1) edge node {2} (t)
				(t) edge [red, dashed] node {} (d1);
				\end{tikzpicture}
			}}}
			\subfloat[Route-Filters]{
				\resizebox {0.5\columnwidth} {!} {
					\begin{tikzpicture}[shorten >=0.5pt,node distance=,on grid,auto,
					square/.style={regular polygon,regular polygon sides=4}] 
					\node[state] at (0,0) (s)  {$s$}; 
					\node[state] at (2, 1) (v1)  {$r_1$}; 
					\node[state] at (4, 0)(t) {$t$};
					\node[state, rectangle] at (5.5, 0.75) (d1) {$\lambda_1$};
					\node[state, rectangle] at (5.5, -0.75) (d2) {$\lambda_2$};
					\path[->] 
					(s) edge node {1} (v1)
					edge  node [above] {1} node [below] {$rf((s,t),\lambda_1)$} (t)
					edge [red, dashed, bend left=90] node {} (t)
					edge [blue, dashed, bend right=45] node {} (t)
					(v1) edge node {2} (t)
					(t) edge [red, dashed] node {} (d1)
					(t) edge [blue, dashed] node {} (d2);
					\end{tikzpicture}
				}}
				\compactcaption{OSPF edge weights and filters such that the
					the routers forward traffic for destination along
					the input paths (dashed arcs).}
				\label{fig:ospfexample}
\end{figure}
			
			
			\loris{route-filters vs route filters}
\subsection{OSPF Synthesis Without Route-filters} \label{sec:sarc}
 
If $n=0$ and no route-filters are allowed,
upon assigning 
 weights $W$ to links between routers (directed edges
 in the network topology),
 OSPF routers use
 Djikstra's algorithm to choose the
 shortest weighted path for a pair of endpoints. 
We show how,
 given a set of input paths $Pi$, \name 
generates a set of linear constraints to 
find weights for the edges in $L$
such that 
 the shortest path through the network
 for these endpoints exactly match the input paths $\Pi$. 
 \loris{figure is weird, make the link themselves red as well? or double the links}
 For example in \Cref{fig:ospfexample}(a), if the input
 path is $s\rightarrow r_1 \rightarrow t$ for
 destination IP $\lambda$, \name assigns
 edge weights such that the input path has a strictly
 smaller weight ($W=1+2$) than the other path $s \rightarrow t$ 
 ($W=5$). 
 
The problem of synthesizing the weight function $W$ that
that realizes an input set of paths $\Pi$ is
a
variation of the so-called {\em inverse shortest path} 
problem~\cite{isp}. 
For a destination IP $\lambda$, we call $\xi_\lambda$ 
the directed tree of $T$ 
obtained by only keeping the nodes and edges 
that are traversed by paths in $\Pi$ for 
destination IP $\lambda$.
Since paths are acyclic, $\xi_d$ is a directed acyclic graph.
$\Delta=\{\xi_\lambda\mid \lambda \in \Lambda\}$ is   
the set of all destination trees. 
\loris{need to say what a destination tree is and tie to previous discussion of genesis modification}

We use  $r_1\rightarrow r_2$ to denote $(r_1,r_2)\in L$ and
$r_1\rightarrow^* r_2$ (resp. $r_1\rightarrow^+ r_2$) to denote 
that $r_2$ is reachable from $r_1$ by crossing zero 
(resp. one) or more links in $T$.
Similarly, we use $\rightarrow_{\xi_\lambda}$ 
to denote the same relations in the destination trees.


\minisection{Distance equations}
We generate a set of linear equations
to find the edge weight $W(r_1, r_2)$ 
for all $(r_1, r_2) \in L$.
We use 
$D(r_1, r_2)$ to denote the 
shortest distance from $r_1$ to $r_2$ and
add the equation $D(s,s) = 0$, 
for every $s\in S$, to denote that the distance
from a node to itself is $0$.
The
following equation guarantees that $D(s,t)$ is not greater than 
the actual shortest distance from $s$ to $t$.
\begin{equation} \label{eq:dist}
\forall s, t. ~\forall r. ~r \in N(s).~
D(s, t) \leq W(s, r) + D(r, t)
\end{equation}

For each destination tree $\xi_d\in\Delta$, we add equations to ensure 
that the input paths with destination $d$ are indeed the shortest ones.
 If a path
is the shortest path between its endpoints, then every 
subpath of the path has to be the shortest between its endpoints
as well (otherwise the complete path would not be the shortest).
Consider a tree $\xi_d$ for destination $d$. 
The function $N_{\xi_d}(s)$ denotes the set of
neighbours of switch $s$ in the destination tree $\xi_d$. 
Given a destination $d\in \Omega$,
we use the following equations to ensure that, given two nodes $s$ and $t$ in
$\xi_d$, 
the set of paths from $s$ to $t$ in $\xi_d$ are
exactly
\emph{the} shortest paths from $s$ to $t$ in $T$.
Let $s$ and $t$ be two nodes in $\xi_d$ and let  $Paths_{\xi_d}(s,t)$ be the set of paths from $s$ to $t$ in $\xi_d$.
\begin{multline} \label{eq:uniq1}
		\forall l_0\cdots l_n\in Paths_{\xi_d}(s,t).
		\forall n' \in N(s) \setminus N_{\xi_d}(s). \\
		W(s, n') + D(n', t) > \sum_{\mathclap{\substack{l_i=(s_i,t_i)}}} 
		W(s_i, t_i) 
\end{multline}
\begin{multline} \label{eq:uniq2}
		\forall l_0\cdots l_n\in Paths_{\xi_d}(s,t).
		\forall n' \in N_{\xi_d}(s). n' \not\rightarrow^+_{\xi_d} t.  \\
		W(s, n') + D(n', t) > \sum_{\mathclap{\substack{l_i=(s_i,t_i)}}} 
		W(s_i, t_i) 
\end{multline}
\vspace{-2mm}
\begin{multline} \label{eq:uniq3}
		\forall l_0\cdots l_n, l_0'\cdots l_n'\in Paths_{\xi_d}(s,t).\\
		\sum_{\mathclap{\substack{l_i=(s_i,t_i)}}} 
		W(s_i, t_i)  =\sum_{\mathclap{\substack{l_i'=(s_i',t_i')}}} 
		W(s_i', t_i') 
\end{multline}
Equation~\ref{eq:uniq1} guarantees that 
the sum of the weights belonging to a path from $s$ to $t$ in $\xi_d$ is smaller than 
any path that goes to $t$ via a node $n'$ that is a neighbour of $s$ in $T$ but not in $\xi_d$.
Equation~\ref{eq:uniq2} guarantees that
the sum of the weights belonging to a path from $s$ to $t$ in $\xi_d$ is smaller than 
any path that goes to $t$ via a node $n'$ that is a neighbour of $s$ in $\xi_d$ but such that
$t$ is not reachable from $n'$ in $\xi_d$.
Finally, Equation~\ref{eq:uniq3} guarantees that all the paths from $s$ to $t$ in $\xi_d$ have the same weight.


\subsection{OSPF Synthesis with Route-filters} \label{sec:routefilter}

If the system of equations presented Section~\ref{sec:sarc} admits a solution, 
the values of the $W$ variables are the weights we are trying to synthesize,
otherwise the OSPF synthesis problem cannot be solved without using route-filters.
In this section, we showed how the proposed technique can be modified
to support route-filters---i.e., when $n>0$.

A route-filter  can selectively disable an
edge for a given destination by  blocking advertisements to a
particular destination along a link.
We start by observing that the OSPF synthesis problem with route-filters
admits a trivial solution in which 
route-filters are used to enforce the exact set of input paths by blocking all other possible paths.
\loris{double-check this notation}
This can be done by creating a 
route-filter $(l,d)$ for every link $l$ not in $\xi_d$. 
However, this solution will place many more filters than necessary.
Since the problem of optimally placing route-filters is NP-complete, 
we propose a greedy strategy that does not guarantee to place a minimal number of filters, but works well in practice.


Our algorithm starts by trying to synthesize a solution
that does not use route-filters using the equations proposed in \secref{sec:sarc}. 
In the case of a failure, the algorithm uses the ``proof of unsatisfiability''---i.e., the unsatisfiable core---generated by 
the constraint solver 
to greedily add a small set of route-filters. 
New modified equations are then generated to model the added route-filters and the approach is repeated until a solution is found.
We first describe the 
modified linear equations generated when a set of
route-filters are enabled, and then describe two
techniques used to choose route-filters. 

\minisection{Equations with route-filters}
%We show how the technique used to solve the simplified synthesis
%problem can be modified to handle filtered and unfiltered paths.
We assume we are given a set of route-filters $RF$ and 
use $s\rightarrow_d^* t$ to denote that $s$ can reach $t$
\loris{again double check notation}
in $T$ without using any edge $l$, such that $(l,d)\in RF$.
We use $D_d(r_1, r_2)$ to denote the shortest distance from $r_1$ to $r_2$
using only edges that are not filtered for destination $d$.
We can revise the equation in \eqref{eq:dist} to correctly restrict the values of $D_d$
by simply ignoring all the filtered edges. 
In the same way, we can modify equations  \eqref{eq:uniq1} and \eqref{eq:uniq2}, while
equation \eqref{eq:uniq3} remains unchanged.

While the encoding without route-filters produces $n$ equations, this encoding produces $|\Omega|n$ due
to the multiple different distances $D_d$.
Notice that, the shortest distance $D_d(s,t)$ between two nodes $s$ and $t$ without using edges filtered for $d$ cannot be
smaller than the shortest distance $D(s,t)$ obtained without considering route-filters.
We use this property to simplify the encoding by only computing $D(s,t)$ and by replacing each instance of $D_d(s,t)$
with $D(s,t)$ in
equations \eqref{eq:uniq1} and \eqref{eq:uniq2}. 
It is easy to see that every solution of this simplified set of constraints
is also a solution to the original solution (because $D(s,t)\leq D_d(s,t)$).
However, the reverse is not true and the set of simplified equations can be unsatisfiable
in cases in which the original set is satisfiable, causing addition of unnecessary filters.
Our implementation uses this simplified set of equations, but we will show that this
does not effect our ability to find good solutions in the evaluation section.


\minisection{Adding filters using unsatisfiable cores}
When the set of linear equations does not admit a solution, we 
add router-filters until, eventually, a solution is found.
We discuss two schemes used to add route-filters:
the first scheme uses unsatisfiable cores generated
by the solver and the second scheme 
finds topological structures called diamonds that 
can never be handled without route-filters.\loris{do we leave this?}

LP-solvers have efficient procedures to return an
unsatisfiable core, also called IIS (Irreducible Inconsistent Subsystem)~\cite{chinneck2007feasibility}. 
Formally, an IIS is a subset of constraints such that,
if all constraints except those in the IIS are removed, the resulting set of
linear equations is still inconsistent (unsatisfiable). Moreover, the set is irreducible---i.e., removing 
any one constraint of the IIS produces a consistent set of constraints. 
When the solver returns unsatisfiable and produces
an unsatisfiable core,
some of the linear inequalities from 
Equations \eqref{eq:uniq1} and  \eqref{eq:uniq2}
will appear in the unsat-core 
(an unsat-core cannot consist of only 
constraints from \Cref{eq:dist} and \eqref{eq:uniq3}, as all distances and edges set to zero
would trivially be consistent with these constraints). 
In particular, the unsat-core contains some constraint
\begin{eqnarray}
E(s, n') + D(n', t) > \sum_{\mathclap{\substack{l_i=(s_i,t_i)}}} 
		E(s_i, t_i) 
\end{eqnarray}
that was added to reason about some DAG\loris{DAG or tree. bit confused.} $\xi_d$.

By adding the route-filter $((s,n'),d)$ to $F$, this inequality is removed from the set of constraints
and the combination of the other constraints appearing in the other unsat-core is now satisfiable.
The new set of equations may still be unsatisfiable and other unsat-cores might exist. 
The procedure is repeated until the resulting set of constraints becomes satisfiable
and we have reached a solution to the OSPF synthesis problem.

%Finding the optimal number of route-filters is NP-hard, and it
%is very difficult to incorporate optimality in our iterative learning
%technique for finding the set of filters. This is because we do not 
%have the set of all unsatisfiable cores (the solver returns one at 
%a time) to find an optimal set of filters. Instead of considering the 
%number of filters, we consider the metric: loss of resilience.
%\loris{not sure the next paragraph is needed}
%For a given unsat-core, there may be multiple ways to place a route
%filter to eliminate one constraint and we have not investigated
%We can 
%adopt a greedy approach (based on set cover \cite{setcover}) 
%of picking a route-filter which 
%eliminates the maximum number of unsat-cores. However, 
%finding the number of unsat-cores a route-filter eliminates
%is an open problem and instrumental in minimizing the number 
%of route-filters. Other schemes can be used to choose 
%a route-filter from an unsat-core satisfying certain
%objectives. 

Finding an IIS is an NP-hard problem~\cite{iiscomplexity}
and can result in slow synthesis times.
\loris{should we say, before starting we remove all diamonds?}
%We identify a topological property of the set of input paths that 
%is guaranteed to require route-filters and use it to produce an initial set of necessary route-filters.
%This technique allows us to reduce the number of times we are required to compute  unsat-cores.
%
%We define the structure shown in \Cref{fig:diamond}
%as a \emph{diamond}. 
%Formally, a problem instance contains a diamond iff there exists two different destinations $d$ and $d'$
%such that, in their corresponding DAGs $\xi_d$ and $\xi_{d'}$,
%there exists two nodes $s$ and $t$, such that $s\rightarrow_{\xi_d} t$,
%$s\rightarrow_{\xi_{d'}} t$, and
%there exists a path $l_0\cdots l_n$ from $s$ to $t$ in $\xi_d$ that is not a path from
%$s$ to $t$ in $\xi_{d'}$.
%As we mentioned at the beginning, synthesizing an ARC for  diamond structures requires
%the addition of a route-filter.
%In fact, each such a diamond can be ``removed'' by introducing a route-filter $(l_0, d')$ that hides
%the path $l_0\cdots l_n$ for the destination $d'$ (see \Cref{fig:diamond}).
%Diamonds can detected and removed in polynomial time.
%%Consider
%the diamond in \Cref{fig:diamond}. There are two choices
%of route-filters: the $s1-s3$ edge for destination $d1$ 
%and the $s1-s2$ edge for destination $d2$, out of which,
%at least one filter is required to eliminate the 
%inconsistency in the linear equations caused due to the diamond.
%Thus, we find all diamonds for all pairs of destination
%DAGs (this is done in polynomial time) and assign a filter
%to one of the two edges at the source of each diamond. 
%Thus, by removing the diamonds, we can reduce the 
%number of iterations
%of the unsat-core learning approach, which would have 
%provided diamonds as an unsat-core if 
%diamonds were not eliminated.




