\section{Synthesising Routing Configurations}

In this section, we present algorithms for 
solving the path-compliance problem when the domain
assignment function $\Theta$ is given to us.
Before we present our techniques,
we justify the complexity of our approach by showing that
even the simplest variant of this problem is NP-complete.

\loris{why minimizing RF?}
\begin{theorem}[Hardness of synthesis]
\label{thm:ospfsynth}
Given a set of paths $\Pi$,
a topology $T=(V,L)$,
a domain-assignment function $\Theta$, 
the problem of finding 
a local preference  function $LP$, 
a weight function $W$, and 
and a route filter function $RF$,  such that
$C=(T,W,RF,LP,\Theta)$,
$\paths^C(PC) = \Pi$,
and 
$\sum_{\lambda\in\Lambda} |RF(\lambda)|$ is minimal is NP-complete.
\end{theorem}
\loris{adapt proof to simply say $n=1$}
\iffull
%!TEX root = paper.tex
\begin{proof}
We show that the decision version of the minimum 
vertex cover problem, i.e., there exists a vertex cover
of size $ \leq k$, which is NP-complete, 
reduces to finding a set of static routes 
of size $ \leq k$ \
and OSPF weights for a network with only one domain. 
The latter is also in NP, so after the reduction we 
can conclude that it is also NP-complete.

Let $G = (V,E)$ be an instance of the 
minimum vertex cover problem. A set of
vertices $VC \subseteq V$ is the vertex cover
if $\forall (v_1, v_2) \in E. ~v_1 \in VC \vee v_2 \in VC$. 

We now show how to construct a topology $T=(S,L)$ 
and a corresponding set of paths $\Pi$ that can be enforced 
by configuration C which requires static routes $SR$ such that $|SR| \leq k$  
if and only if the corresponding $VC(SR)$ is a vertex cover of 
the graph $G$ and $|VC(SR)| \leq k$.

\paragraph{Construction.}
For every vertex $v \in V$: add a vertex $r_v$.
For every edge $(u,v) \in E$: add two vertices $s_{uv}$
and $t_{uv}$ to $S$. Add edges
connecting $s_{uv} \rightarrow r_{u}$, $s_{uv} \rightarrow r_{v}$,
$r_{u} \rightarrow t_{uv}$ and $r_{v} \rightarrow t_{uv}$. 
\Cref{fig:rfcomplexity} illustrates this construction.
\begin{figure}[H]
	\centering
	\begin{tikzpicture}[shorten >=0.5pt,node distance=1.5cm,on grid,auto,
	square/.style={regular polygon,regular polygon sides=4}] 
	\node[state] at (0,0) (s)  {$s_{uv}$}; 
	\node[state] at (3,-1) (v)  {$r_v$}; 
	\node[state] at (3,1) (u)  {$r_u$}; 
	\node[state] at (6,0) (t) {$t_{uv}$}; 	
	\node[state] (s1) [below=of s] {$s_{vw}$}; 	
	\node[state] (t1) [below=of t] {$t_{vw}$}; 	
	\path[-] 
	(s) edge  node {} (v)
	edge  node {} (u)
	edge [blue, dashed, bend right=40] node {} (t)
	edge [red, dashed, bend left=40] node {} (t)
	(u) edge node {} (t)
	(v) edge node {} (t)
	edge node {} (t1)
	edge node {} (s1);
	\path[->]
	(s)
	edge [blue, dashed, bend right=40] node {} (t)
	edge [red, dashed, bend left=40] node {} (t);
%	\path[-] (s) edge node[above] {} +(1,-0.1);
%	\path[-] (t1) edge node[above] {} +(-1,-0.1);
	\end{tikzpicture}
	\caption{Construction for reduction to Vertex Cover.}
	\label{fig:rfcomplexity}
\end{figure}
If there is another edge $(v,w) \in E$, then
$s_{vw}$ and $t_{vw}$ have an edge connecting to $r_v$ (shown
in \Cref{fig:rfcomplexity}). 

For each edge $(u,v) \in E$, we add two paths in $\Pi$: 
$s_{uv} \rightarrow r_u \rightarrow t_{uv}$
for destination host $d_u$ and 
$~s_{uv} \rightarrow r_v \rightarrow t_{uv}$ 
for destination host $d_v$.
(dashed paths in \Cref{fig:rfcomplexity}). 

We now prove that if there exists a set of static routes
$SR$ such that $|SR| \leq k$ such that the resulting configurations
are path-compliant for $\Pi$, then there exists a vertex cover $VC$
of $G$ such that $|VC| \leq k$. 

For each static route $sr \in SR$, the static route
has to be placed at the source, either going to $r_u$
or $r_v$ (structure of topology $T$). 
We construct a set $VC(SR)$ by adding the vertex $v$
based on the endpoints of each static route $sr \in SR$.
To show that $VC(SR)$ is a vertex cover of $G$, we first
prove \Cref{lemma:diamond}.

\begin{lemma} \label{lemma:diamond}
	 For each diamond formed by the input paths, atleast 1 
	 static route on one of the edges of the paths of the diamond 
	 is required to find a valid solution to the
	 OSPF edge weights.  
\end{lemma}

\begin{proof}
Given two paths $\pi_1$ and $\pi_2$ for destinations 
$d_u$ and $d_v$, we define these paths form a diamond
if these paths intersect at two routers ($s_{uv}$ and $t_{uv}$) 
without any common router in between. 
Consider the following diamond % in \Cref{fig:diamond}
constructed by paths $\pi_1$: $s_{uv} \rightarrow r_u \rightarrow t_{uv}$ 
for destination $d_u$ and $\pi_2$: $s_{uv} \rightarrow r_v \rightarrow t_{uv}$ 
for destination $d_v$. Let us assume there exists a solution 
for the OSPF edge weights without any static routes. 

We add the following inequality 
to make $\pi_1$ is the shortest 
path from $s_{uv}$ to $t_{uv}$ by
ensuring
 $\pi_1$ is shorter than the
path from $s_{uv}$ to $t_{uv}$ via $r_v$: 
\begin{equation} \label{eq:diamond1}
	W(s_{uv},r_u) + W(r_u, t_{uv}) < W(s_{uv}, r_v) + W(r_v,t_{uv})
\end{equation}
Since $\pi_2$ is also the shortest path from $s_{uv}$ 
to $t_{uv}$, the linear inequality added is:
\begin{equation}  \label{eq:diamond2}
W(s_{uv},r_v) + W(r_v, t) < W(s_{uv}, r_u) + W(r_u,t_{uv})
\end{equation}
Since there are no static routes on the edges
of $\pi_1$ and $\pi_2$, none of the above equations are 
eliminated. 
Adding equations \ref{eq:diamond1} and  \ref{eq:diamond2} 
yields the inequality $0 < 0$, which is inconsistent 
and therefore, no solution to 
the edge weights exists for this system of equations, 
which contradicts our assumption. Therefore,
for each diamond formed by the input paths, atleast 1 
static route on one of the edges of the paths of the diamond 
is required to find a valid solution to the
OSPF edge weights.  
\end{proof}

For every edge $(u,v) \in E$, the constructed paths from 
$s_{uv}$ to $t_{uv}$ form a diamond. Thus, by Lemma~\ref{lemma:diamond}, 
the diamond created by the paths corresponding to each edge in $G$ 
requires atleast one static route to eliminate
the inconsistency caused by the diamond. If a static route's endpoints
contains $r_u$, we put $u$ in $VC(SR)$ and similarily for $r_v$. 
Edge $(u,v)$ is covered since atleast one static route is added,
thus, atleast one of $\{u,v\}$ is in $VC(SR)$.  
Thus, if $SR$ eliminates all diamond inconsistencies
to find a solution to the OSPF weights, the corresponding set
$VC(SR)$ covers all edges in $E$. Therefore, $VC(SR)$ is a vertex
cover. 

Thus, by finding a set of static routes $SR$ such that $|SR| \leq k$
such that all the diamond inconsistencies are eliminated, and there
exists OSPF weights $W$ such that the configurations forward traffic
along $\Pi$, we can find a vertex cover $VC$ for graph $G$ such that
$|VC| \leq k$. 

This transformation is polynomial, the constructed 
network topology $T$ has $|V| + 2|E|$ nodes, 
$4|E|$ links and $2|E|$ paths. Therefore, OSPF
configuration synthesis with number of static routes $\leq k$ is
NP-complete. Thus, OSPF synthesis with minimal number of 
static routes is NP-hard. 
\end{proof}

\else
The reduction is from the vertex cover problem.
\fi
In the next two sections, we first show how to solve the intra-domain synthesis problem---i.e., when there
is only a single domain---and then how to solve the inter-domain synthesis problem---i.e., when
there are multiple domains.

\subsection{Synthesising Intradomain Configurations} \label{sec:synthesis}
In this section, we present an algorithm for 
solving the path-compliance problem for 
single-domain OSPF networks
---i.e., $|\{\Theta(r) \mid r\in V\}|=1$.
We first show how to solve the problem when
we are not allowed to use route filters
and then
extend our technique to handle route filters, when these are allowed.

\begin{figure}
	\centering
	\subfloat[Edge Weights]{
		\raisebox{0.5cm}{\resizebox {0.5\columnwidth} {!} {
				\begin{tikzpicture}[shorten >=0.5pt,node distance=,on grid,auto,
				square/.style={regular polygon,regular polygon sides=4}] 
				\node[state] at (0,0) (s)  {$s$}; 
				\node[state] at (1.8,1) (v1)  {$r_1$}; 
				\node[state] at (3.6, 0)(t) {$t$};
				\node[state, rectangle] at (5, 0) (d1) {$\lambda$};
				\path[->] 
				(s) edge [red] node [black] {1} (v1)
				edge  node {5} (t)
				(v1) edge [red] node [black] {2} (t)
				(t) edge [red, dashed] node {} (d1);
				\end{tikzpicture}
			}}}
			\subfloat[Route-Filters]{
				\resizebox {0.5\columnwidth} {!} {
					\begin{tikzpicture}[shorten >=0.5pt,node distance=,on grid,auto,
					square/.style={regular polygon,regular polygon sides=4}] 
					\node[state] at (0,0) (s)  {$s$}; 
					\node[state] at (2, 1) (v1)  {$r_1$}; 
					\node[state] at (4, 0)(t) {$t$};
					\node[state, rectangle] at (5.5, 0.75) (d1) {$\lambda_1$};
					\node[state, rectangle] at (5.5, -0.75) (d2) {$\lambda_2$};
					\path[->] 
					(s) edge [red] node [black] {1} (v1)
					edge [blue] node [above, black] {1} node [below, black] {$rf((s,t),\lambda_1)$} (t)
					(v1) edge [red] node [black] {2} (t)
					(t) edge [red, dashed] node {} (d1)
					(t) edge [blue, dashed] node {} (d2);
					\end{tikzpicture}
				}}
				\compactcaption{OSPF edge weights and filters such that the
					the routers forward traffic for destination along
					the colored paths.}
				\label{fig:ospfexample}
\end{figure}
			
			
			\loris{route-filters vs route filters}
\subsubsection{Intra-domain Synthesis Without Route-filters} \label{sec:sarc}
 
When no route-filters are allowed,
 OSPF routers use the edge weights $W$
 to choose the
 shortest weighted path for each pair of endpoints. 
Our goal is to synthesize a $W$.
 For example in \Cref{fig:ospfexample}(a), if $\Pi$ 
 contains the 
 path $\pi=s\rightarrow r_1 \rightarrow t$ for
 destination IP $\lambda$, \name will assign
 edge weights so that $\pi$ has
 smaller weight ($W(\pi)=1+2$) than the path $\pi'=s \rightarrow t$ 
 ($W(\pi')=5$). 
 
The problem of synthesizing the weight function $W$ that
realizes an input set of paths $\Pi$ is
a
variation of the so-called {\em inverse shortest path} 
problem~\cite{isp}. 
For a destination IP $\lambda$, we call $\xi_\lambda$ 
the directed tree of $T$ 
obtained by only keeping the nodes and edges 
that are traversed by paths in $\Pi$ for 
destination IP $\lambda$, the root of the tree
is the destination router connected to $\lambda$. 
 This destination tree
 property is due to the modifications in \genesis
 to support OSPF's destination-based forwarding. We
 define $\Delta=\{\xi_\lambda\mid \lambda \in \Lambda\}$,   
the set of all destination trees. 

Given a set of input paths $\Pi$, \name 
generates a set of linear constraints to 
find proper weights $W(r_1,r_2)$ 
for all $(r_1, r_2) \in L$.
The constraints use the variable $W_{r_1}^{r_2}$
to denote the weight of the edge $(r_1, r_2)$, and the variable
$D_{r_1}^{r_2}$ to denote the 
shortest distance from $r_1$ to $r_2$.
We add the equation $D_{s}^{s} = 0$ 
for every $s\in S$ to denote that the distance
from a node to itself is $0$.
\Cref{eq:dist} guarantees that $D_{s}^{t}$ is smaller or equal to
the shortest distance from $s$ to $t$.
\begin{equation} \label{eq:dist}
\forall s, t. ~\forall r \in N(s).~
D_s^t \leq W_s^r + D_r^t
\end{equation}

For each destination tree $\xi_\lambda\in\Delta$, we add equations to ensure 
that the input paths with destination $d$ are the shortest ones.
Notice that, if a path $\pi$
is the shortest path between its endpoints, every 
subpath of $\pi$ also has to be the shortest path between its endpoints.
For each tree $\xi_\lambda$, we define the neighbour
 $N_{\xi_\lambda}(s)$ to denote the 
next-hop neighbour of router $s$ in the destination tree $\xi_\lambda$---i.e., the parent
of $s$ in the tree or no node if $s$ is the root of the tree.
%Since $\xi_\lambda$ is a directed tree, every router $r \in \xi_\lambda$
%has one next-hop router, therefore $N_{\xi_\lambda}(s)$ is a singleton
%set (the root of the tree (router connected to $\lambda$) has no next-hop
%routers).

The following equations ensures that, for any 
 two nodes $s$ and $t$ in
$\xi_\lambda$, if $t$ is reachable from $s$, 
the path $l_0\cdots l_k$ from $s$ to $t$ in $\xi_\lambda$ is the 
\emph{unique shortest path} from $s$ to $t$ in $T$.
\begin{multline} \label{eq:uniq1}
\forall r \in N(s) \setminus N_{\xi_\lambda}(s).~~
W_{s}^{r}+ D_{r}^{t} > \sum_{\mathclap{\substack{l_i=(r_1,r_2)}}} 
W_{r_1}^{r_2}
\end{multline}
\Cref{eq:uniq1} guarantees that 
the sum of the weights belonging to a path from $s$ to $t$ in $\xi_\lambda$ 
is strictly smaller than 
any path that goes to $t$ via 
a node $n'$ that is a neighbour of $s$ in $T$ but not 
the next-hop in $\xi_\lambda$.

\subsubsection{Intra-domain Synthesis with Route-filters} \label{sec:routefilter}

If the system of equations presented Section~\ref{sec:sarc} admits a solution, 
the values of the $W_s^t$ variables are the weights we are trying to synthesize,
otherwise the intra-domain synthesis problem cannot be solved without using route-filters.
A route-filter  can selectively disable an
edge for a given destination by  blocking advertisements to a
particular destination along a link.
We start by observing that the intra-domain synthesis problem with route-filters
admits a trivial solution in which 
route-filters are used to enforce the exact set of input paths by blocking all other possible paths.
This can be done by adding a 
route-filter to $RF(d)$ for every link $l$ not in $\xi_d$. 
However, this solution will place many more filters than necessary.
Since the problem of optimally placing route-filters is computationally hard (Theorem~\ref{thm:ospfsynth}), 
we propose a non-optimal greedy strategy that works well in practice.

Our algorithm starts by trying to synthesize a solution
that does not use route-filters using the equations proposed in \secref{sec:sarc}. 
In the case of a failure, the algorithm uses the ``proof of unsatisfiability''---i.e., the unsatisfiable core---generated by 
the constraint solver 
to greedily add a small set of route-filters. 
Certain equations are modified to model the added route-filters and the approach is repeated until a solution is found.
We first describe how  
equations are modified when
route-filters are enabled, and then a
technique to choose route-filters. 

\minisection{Equations with route-filters}
\loris{why do we need to modify \eqref{eq:dist}? Can't we just add $D_s^t(d)\geq D_s^t$ for
all $d$}
We assume we are given a set of route-filters $RF$ and 
use $s\rightarrow_d^* t$ to denote that $s$ can reach $t$
in $T$ without using any edge in $RF(d)$.
We use the variable $D_{r_1}^{r_2}(d)$ to denote the shortest distance from $r_1$ to $r_2$
using only edges that are not filtered for destination $d$.
We can revise equation \eqref{eq:dist} and   to restrict the values of 
each variable $D_{r_1}^{r_2}(d)$
by  ignoring all the filtered edges---i.e., only consider a switch $r\in N(s)$ if
$(s,r)$ is not in $RF(d)$. 
In summary, we need to add one modified version of equation \eqref{eq:dist} 
and equation  \eqref{eq:uniq1} for each destination $d$.

If the encoding without route-filters produces $n$ equations, this
encoding produces $kn$ equations where $k$ is the total number of destinations.  
To mitigate this problem, we observe that the shortest distance $D_s^t(d)$ between two
nodes $s$ and $t$ without using edges filtered for $d$ i
greater or equal to the distance $D_s^t$ obtained without route-filters.  
We use this property to simplify the
encoding by only computing $D_s^t$ and by replacing each instance of
$D_s^t(d)$ with $D_s^t$ in equations \eqref{eq:uniq1}.  It is easy to see that 
if the set of constraints with variables $D_s^t(d)$ admits a solution,
the corresponding set of constraints with variables $D_s^t$ 
also admits a solution (because $D_s^t\leq
D_s^t(d)$).  However, the reverse is not true and the set of
simplified equations can be unsatisfiable in cases in which the
original set is satisfiable, causing addition of unnecessary filters.
Since our algorithm does not try to compute the optimal 
number of route filters, this is not a problem.


%% We discuss two schemes  to add route-filters:
%% the first scheme 
\minisection{Adding filters using unsatisfiable cores}
When the set of linear equations does not admit a solution, we 
need to add router-filters. We describe how the filters are chosen.

LP-solvers have practically efficient procedures to return an
unsatisfiable core, also called IIS (Irreducible Inconsistent Subsystem)~\cite{chinneck2007feasibility}. 
Intuitively, an IIS is a subset of the input constraints such that,
if all constraints except those in the IIS are removed, the resulting set of
linear equations is still unsatisfiable. Moreover, the set is irreducible---i.e., removing 
any one constraint from the IIS produces a satisfiable set of constraints. 
In our case, an ISS cannot consist of only 
constraints from \Cref{eq:dist} as these equations
admit a trivial solution with all variables set to 0. 
Therefore, an ISS must contain some constraint of the form
given in  \Cref{eq:uniq1}
that was added in to reason about some destination tree $\xi_d$.
\[
W_s^r + D_r^t  > \sum_{\mathclap{\substack{l_i=(r_1,r_2)}}} 
		W_{r_1}^{r_2}
\]		


To remove this inequality from the set of constraints, we add the route-filter $(s,r)$ to $RF(d)$.
This removes the given ISS, but the new set of equations may still be unsatisfiable because of other ISSs. 
The procedure is repeated and route-filters are added until the resulting set of constraints becomes satisfiable.




