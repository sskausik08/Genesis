\section{Synthesising OSPF Configurations} \label{sec:synthesis}
In this section, we present an algorithm for 
solving the path-compliance problem for 
single-domain OSPF networks
---i.e., $|\{\Theta(n) \mid n\in V\}|=1$.

%\subsection{Complexity} \label{sec:rfcomplexity}

Before we present our technique and its intricacies,
we justify the complexity of our approach by showing that
the problem of synthesizing an OSPF configuration
that contains an optimal number of route-filters is NP-complete.

\begin{theorem}
Given a set of paths $\Pi$,
a topology $G=(V,E)$,
a domain-assignment function $\Theta$, such that $|\{\Theta(n) \mid n\in V\}|=1$,
a number $n\geq 0$,
the problem of finding 
an $LP$, $W$, and $RF$,  such that
\loris{PC shouldn't appear here, make consistent.}
$\paths(\Theta, LP, W, RF, PC) = \Pi$,
and 
$\sum_{\lambda\in\Lambda} |RF(\lambda)|=n$, is NP-complete.
\end{theorem}
\iffull
%!TEX root = paper.tex
\begin{proof}
We show that the decision version of the minimum 
vertex cover problem, i.e., there exists a vertex cover
of size $ \leq k$, which is NP-complete, 
reduces to finding a set of static routes 
of size $ \leq k$ \
and OSPF weights for a network with only one domain. 
The latter is also in NP, so after the reduction we 
can conclude that it is also NP-complete.

Let $G = (V,E)$ be an instance of the 
minimum vertex cover problem. A set of
vertices $VC \subseteq V$ is the vertex cover
if $\forall (v_1, v_2) \in E. ~v_1 \in VC \vee v_2 \in VC$. 

We now show how to construct a topology $T=(S,L)$ 
and a corresponding set of paths $\Pi$ that can be enforced 
by configuration C which requires static routes $SR$ such that $|SR| \leq k$  
if and only if the corresponding $VC(SR)$ is a vertex cover of 
the graph $G$ and $|VC(SR)| \leq k$.

\paragraph{Construction.}
For every vertex $v \in V$: add a vertex $r_v$.
For every edge $(u,v) \in E$: add two vertices $s_{uv}$
and $t_{uv}$ to $S$. Add edges
connecting $s_{uv} \rightarrow r_{u}$, $s_{uv} \rightarrow r_{v}$,
$r_{u} \rightarrow t_{uv}$ and $r_{v} \rightarrow t_{uv}$. 
\Cref{fig:rfcomplexity} illustrates this construction.
\begin{figure}[H]
	\centering
	\begin{tikzpicture}[shorten >=0.5pt,node distance=1.5cm,on grid,auto,
	square/.style={regular polygon,regular polygon sides=4}] 
	\node[state] at (0,0) (s)  {$s_{uv}$}; 
	\node[state] at (3,-1) (v)  {$r_v$}; 
	\node[state] at (3,1) (u)  {$r_u$}; 
	\node[state] at (6,0) (t) {$t_{uv}$}; 	
	\node[state] (s1) [below=of s] {$s_{vw}$}; 	
	\node[state] (t1) [below=of t] {$t_{vw}$}; 	
	\path[-] 
	(s) edge  node {} (v)
	edge  node {} (u)
	edge [blue, dashed, bend right=40] node {} (t)
	edge [red, dashed, bend left=40] node {} (t)
	(u) edge node {} (t)
	(v) edge node {} (t)
	edge node {} (t1)
	edge node {} (s1);
	\path[->]
	(s)
	edge [blue, dashed, bend right=40] node {} (t)
	edge [red, dashed, bend left=40] node {} (t);
%	\path[-] (s) edge node[above] {} +(1,-0.1);
%	\path[-] (t1) edge node[above] {} +(-1,-0.1);
	\end{tikzpicture}
	\caption{Construction for reduction to Vertex Cover.}
	\label{fig:rfcomplexity}
\end{figure}
If there is another edge $(v,w) \in E$, then
$s_{vw}$ and $t_{vw}$ have an edge connecting to $r_v$ (shown
in \Cref{fig:rfcomplexity}). 

For each edge $(u,v) \in E$, we add two paths in $\Pi$: 
$s_{uv} \rightarrow r_u \rightarrow t_{uv}$
for destination host $d_u$ and 
$~s_{uv} \rightarrow r_v \rightarrow t_{uv}$ 
for destination host $d_v$.
(dashed paths in \Cref{fig:rfcomplexity}). 

We now prove that if there exists a set of static routes
$SR$ such that $|SR| \leq k$ such that the resulting configurations
are path-compliant for $\Pi$, then there exists a vertex cover $VC$
of $G$ such that $|VC| \leq k$. 

For each static route $sr \in SR$, the static route
has to be placed at the source, either going to $r_u$
or $r_v$ (structure of topology $T$). 
We construct a set $VC(SR)$ by adding the vertex $v$
based on the endpoints of each static route $sr \in SR$.
To show that $VC(SR)$ is a vertex cover of $G$, we first
prove \Cref{lemma:diamond}.

\begin{lemma} \label{lemma:diamond}
	 For each diamond formed by the input paths, atleast 1 
	 static route on one of the edges of the paths of the diamond 
	 is required to find a valid solution to the
	 OSPF edge weights.  
\end{lemma}

\begin{proof}
Given two paths $\pi_1$ and $\pi_2$ for destinations 
$d_u$ and $d_v$, we define these paths form a diamond
if these paths intersect at two routers ($s_{uv}$ and $t_{uv}$) 
without any common router in between. 
Consider the following diamond % in \Cref{fig:diamond}
constructed by paths $\pi_1$: $s_{uv} \rightarrow r_u \rightarrow t_{uv}$ 
for destination $d_u$ and $\pi_2$: $s_{uv} \rightarrow r_v \rightarrow t_{uv}$ 
for destination $d_v$. Let us assume there exists a solution 
for the OSPF edge weights without any static routes. 

We add the following inequality 
to make $\pi_1$ is the shortest 
path from $s_{uv}$ to $t_{uv}$ by
ensuring
 $\pi_1$ is shorter than the
path from $s_{uv}$ to $t_{uv}$ via $r_v$: 
\begin{equation} \label{eq:diamond1}
	W(s_{uv},r_u) + W(r_u, t_{uv}) < W(s_{uv}, r_v) + W(r_v,t_{uv})
\end{equation}
Since $\pi_2$ is also the shortest path from $s_{uv}$ 
to $t_{uv}$, the linear inequality added is:
\begin{equation}  \label{eq:diamond2}
W(s_{uv},r_v) + W(r_v, t) < W(s_{uv}, r_u) + W(r_u,t_{uv})
\end{equation}
Since there are no static routes on the edges
of $\pi_1$ and $\pi_2$, none of the above equations are 
eliminated. 
Adding equations \ref{eq:diamond1} and  \ref{eq:diamond2} 
yields the inequality $0 < 0$, which is inconsistent 
and therefore, no solution to 
the edge weights exists for this system of equations, 
which contradicts our assumption. Therefore,
for each diamond formed by the input paths, atleast 1 
static route on one of the edges of the paths of the diamond 
is required to find a valid solution to the
OSPF edge weights.  
\end{proof}

For every edge $(u,v) \in E$, the constructed paths from 
$s_{uv}$ to $t_{uv}$ form a diamond. Thus, by Lemma~\ref{lemma:diamond}, 
the diamond created by the paths corresponding to each edge in $G$ 
requires atleast one static route to eliminate
the inconsistency caused by the diamond. If a static route's endpoints
contains $r_u$, we put $u$ in $VC(SR)$ and similarily for $r_v$. 
Edge $(u,v)$ is covered since atleast one static route is added,
thus, atleast one of $\{u,v\}$ is in $VC(SR)$.  
Thus, if $SR$ eliminates all diamond inconsistencies
to find a solution to the OSPF weights, the corresponding set
$VC(SR)$ covers all edges in $E$. Therefore, $VC(SR)$ is a vertex
cover. 

Thus, by finding a set of static routes $SR$ such that $|SR| \leq k$
such that all the diamond inconsistencies are eliminated, and there
exists OSPF weights $W$ such that the configurations forward traffic
along $\Pi$, we can find a vertex cover $VC$ for graph $G$ such that
$|VC| \leq k$. 

This transformation is polynomial, the constructed 
network topology $T$ has $|V| + 2|E|$ nodes, 
$4|E|$ links and $2|E|$ paths. Therefore, OSPF
configuration synthesis with number of static routes $\leq k$ is
NP-complete. Thus, OSPF synthesis with minimal number of 
static routes is NP-hard. 
\end{proof}

\fi

%However, given a set of paths as input, there may
%not exist a solution to the edge weights. Consider the 
%input paths as shown in \Cref{fig:ospfexample}(b). 
%Both the red and blue paths are required 
%to be the unique shortest path between $s$ to $t$
%and, clearly, this is cannot be enforced for any 
%choice of the edge weights (as weights correspond 
%to all destinations). 
%One way to synthesize configurations in this scenario 
%is to ``disable'' the edge
%$(s, t)$ for destination $\lambda_1$.
%Using this technique, 
%there is only one possible path from $s$ to $t$
%for destination $\lambda_1$ ($s\rightarrow r_1 \rightarrow t$),
%therefore is chosen as the shortest path. For 
%destination $\lambda_2$, the $s\rightarrow t$ path
%has a smaller weight ($w=1$) than the
%$s\rightarrow r_1 \rightarrow t$ path ($w=1+2$), therefore,
%both traffic is forwarded through the input paths for
%both the destinations. 
%This blocking mechanism is called a route-filter, and
%we modify \name's OSPF synthesis algorithm to support
%route-filtering (\secref{sec:filtering}).
%

\begin{figure}
	\centering
	\subfloat[Edge Weights]{
		\raisebox{0.5cm}{\resizebox {0.5\columnwidth} {!} {
				\begin{tikzpicture}[shorten >=0.5pt,node distance=,on grid,auto,
				square/.style={regular polygon,regular polygon sides=4}] 
				\node[state] at (0,0) (s)  {$s$}; 
				\node[state] at (1.8,1) (v1)  {$r_1$}; 
				\node[state] at (3.6, 0)(t) {$t$};
				\node[state, rectangle] at (5, 0) (d1) {$\lambda$};
				\path[->] 
				(s) edge node {1} (v1)
				edge  node {5} (t)
				edge [red, dashed, bend left=90] node {} (t)
				(v1) edge node {2} (t)
				(t) edge [red, dashed] node {} (d1);
				\end{tikzpicture}
			}}}
			\subfloat[Route-Filters]{
				\resizebox {0.5\columnwidth} {!} {
					\begin{tikzpicture}[shorten >=0.5pt,node distance=,on grid,auto,
					square/.style={regular polygon,regular polygon sides=4}] 
					\node[state] at (0,0) (s)  {$s$}; 
					\node[state] at (2, 1) (v1)  {$r_1$}; 
					\node[state] at (4, 0)(t) {$t$};
					\node[state, rectangle] at (5.5, 0.75) (d1) {$\lambda_1$};
					\node[state, rectangle] at (5.5, -0.75) (d2) {$\lambda_2$};
					\path[->] 
					(s) edge node {1} (v1)
					edge  node [above] {1} node [below] {$rf((s,t),\lambda_1)$} (t)
					edge [red, dashed, bend left=90] node {} (t)
					edge [blue, dashed, bend right=45] node {} (t)
					(v1) edge node {2} (t)
					(t) edge [red, dashed] node {} (d1)
					(t) edge [blue, dashed] node {} (d2);
					\end{tikzpicture}
				}}
				\compactcaption{OSPF edge weights and filters such that the
					the routers forward traffic for destination along
					the input paths (dashed arcs).}
				\label{fig:ospfexample}
\end{figure}
			
\subsection{Simple OSPF Configurations} \label{sec:sarc}
 
For the OSPF protocol, configurations assign
 weights $W$ to links between routers (directed edges
 in the network topology);
 OSPF routers uses 
 Djikstra's algorithm to choose the
 shortest weighted path for a pair of endpoints. 
 Thus,
 given input paths, \name finds edge weights 
 (which are global for all paths) such that 
 the shortest path through the network
 for these endpoints exactly match the input paths. 
 For example in \Cref{fig:ospfexample}(a), if the input
 path is $s\rightarrow r_1 \rightarrow t$ for
 destination IP $\lambda$, \name assigns
 edge weights such that the input path has a strictly
 smaller weight ($W=1+2$) than the other path $s \rightarrow t$ 
 ($W=5$). Thus, the OSPF routers will forward traffic for
 $\lambda$ from $s$ to $t$ through $r_1$. We first
 consider the scenario of simple OSPF configurations (sOC)
 with no route-filters enabled.

The problem of synthesizing the
sOC that realizes an input set of paths reduces to a
variation of the so-called {\em inverse shortest path} 
problem~\cite{isp}. 
%Assume we are given the following inputs: (1) a directed graph $T = (S, L)$ (the network topology), 
%(2) a set of endpoints $\Gamma \subseteq S\times S$
%describing the sources and destinations of the input paths, and 
%(3) a function $P: \Gamma \rightarrow 2^{L^*}$
%that assigns to each pair of endpoints $(s,t) \in \Gamma$ 
%a set of \emph{acyclic} paths, such that for every path $l_0\cdots l_n\in P(s,t)$,
%$l_0=(s,s')$, for some $s'\in S$, and $l_n=(s'',t)$, for some $s''\in S$.\footnote{
%We use $L^*$ the denote the set of all finite sequences over $L$.}
%The 
%\emph{sARC synthesis}
%problem is to find rational weights for the edges in $L$ such that 
%for each pair of endpoints $(s,t) \in \Gamma$, 
%the paths in $P(s,t)$ are \emph{the} shortest paths from $s$ to $t$ 
%in the graph. Notice, that there can be multiple shortest
%paths of equal cost for multi-path support (e.g., for traffic engineering).
For a destination IP $\lambda$, we call $\xi_\lambda$ 
the directed tree of $T$ 
obtained by only keeping the nodes and edges 
that are traversed by paths in $\Pi$ for 
destination IP $\lambda$.
Since paths are acyclic, $\xi_d$ is a directed acyclic graph.
$\Delta=\{\xi_\lambda\mid \lambda \in \Lambda\}$ is   
the set of all destination trees. 

We use  $r_1\rightarrow r_2$ to denote $(r_1,r_2)\in L$ and
$r_1\rightarrow^* r_2$ (resp. $r_1\rightarrow^+ r_2$) to denote 
that $r_2$ is reachable from $r_1$ by crossing zero 
(resp. one) or more links in $T$.
Similarly, we use $\rightarrow_{\xi_\lambda}$ 
to denote the same relations in the destination trees.


\minisection{Distance equations}
To solve the sOC synthesis problem, 
we generate a set of linear equations
to find the edge weight $W(r_1, r_2)$ 
for all $(r_1, r_2) \in L$.
We use 
$D(r_1, r_2)$ to denote the 
shortest distance from $r_1$ to $r_2$.
We add the equation $D(s,s) = 0$ 
for every $s\in S$ to denote that the distance
from a node to itself is $0$.
The
following equation guarantees that $D(s,t)$ is not greater than 
the actual shortest distance from $s$ to $t$.
\begin{equation} \label{eq:dist}
\forall s, t. ~\forall r. ~r \in N(s).~
D(s, t) \leq W(s, r) + D(r, t)
\end{equation}

For each destination tree $\xi_d\in\Delta$, we add equations to ensure 
that the input paths with destination $d$ are indeed the shortest ones.
 If a path
is the shortest path between its endpoints, then every 
subpath of the path has to be the shortest between its endpoints
as well (otherwise the complete path would not be the shortest).

Consider a tree $\xi_d$ for destination $d$. We define two neighbour
functions: $N_T(s)$ denotes the set of neighbours of ritch $s$ 
in the input graph $T$, and $N_{\xi_d}(s)$ denote the set of
neighbours of ritch $s$ in the destination tree $\xi_d$. 
Given a destination $d\in \Omega$,
we use the following equations to ensure that, given two nodes $s$ and $t$ in
$\xi_d$, 
the set of paths from $s$ to $t$ in $\xi_d$ are
exactly
\emph{the} shortest paths from $s$ to $t$ in $T$.
Let $s$ and $t$ be two nodes in $\xi_d$ and let  $Paths_{\xi_d}(s,t)$ be the set of paths from $s$ to $t$ in $\xi_d$.
\begin{multline} \label{eq:uniq1}
		\forall l_0\cdots l_n\in Paths_{\xi_d}(s,t).
		\forall n' \in N(s) \setminus N_{\xi_d}(s). \\
		W(s, n') + D(n', t) > \sum_{\mathclap{\substack{l_i=(s_i,t_i)}}} 
		W(s_i, t_i) 
\end{multline}
\begin{multline} \label{eq:uniq2}
		\forall l_0\cdots l_n\in Paths_{\xi_d}(s,t).
		\forall n' \in N_{\xi_d}(s). n' \not\rightarrow^+_{\xi_d} t.  \\
		W(s, n') + D(n', t) > \sum_{\mathclap{\substack{l_i=(s_i,t_i)}}} 
		W(s_i, t_i) 
\end{multline}
\vspace{-2mm}
\begin{multline} \label{eq:uniq3}
		\forall l_0\cdots l_n, l_0'\cdots l_n'\in Paths_{\xi_d}(s,t).\\
		\sum_{\mathclap{\substack{l_i=(s_i,t_i)}}} 
		W(s_i, t_i)  =\sum_{\mathclap{\substack{l_i'=(s_i',t_i')}}} 
		W(s_i', t_i') 
\end{multline}
Equation~\ref{eq:uniq1} guarantees that 
the sum of the weights belonging to a path from $s$ to $t$ in $\xi_d$ is smaller than 
any path that goes to $t$ via a node $n'$ that is a neighbour of $s$ in $T$ but not in $\xi_d$.
Equation~\ref{eq:uniq2} guarantees that
the sum of the weights belonging to a path from $s$ to $t$ in $\xi_d$ is smaller than 
any path that goes to $t$ via a node $n'$ that is a neighbour of $s$ in $\xi_d$ but such that
$t$ is not reachable from $n'$ in $\xi_d$.
Finally, Equation~\ref{eq:uniq3} guarantees that all the paths from $s$ to $t$ in $\xi_d$ have the same weight.
