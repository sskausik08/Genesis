\section{Complexity}
\begin{theorem}
Finding the set of route-filters $R$ of smallest size such
 that there exists OSPF edge weights $W$ for topology $G=(V,E)$ 
 and the resulting configurations forward traffic in the network
 topology $T=(S,L)$ along the
input paths $P$ under non-failure scenarios, is NP-hard.
\end{theorem}

\begin{proof}
We show that the decision version of the minimum 
vertex cover problem, i.e., there exists a vertex cover
of size $ \leq k$, which is NP-complete, 
reduces to finding a set of route-filters of size $ \leq k$ \
and OSPF weights which forward traffic along the
input paths under non-failure scenarios. 
The latter is also in NP, so after the reduction we 
can conclude that it is also NP-complete.

Let $G = (V,E)$ be an instance of the 
minimum vertex cover problem. A set of
vertices $VC \subseteq V$ is the vertex cover
if $\forall (v_1, v_2) \in E. ~v_1 \in VC \vee v_2 \in VC$. 

We now show how to construct a topology $T=(S,L)$ 
and a corresponding set of paths $P$ that can be 
enforced by route-filter set $R$ such that $|R| \leq k$  
iff the corresponding $VC(R)$ is a vertex cover of 
the graph $G$ and $|VC(R)| \leq k$.

For every vertex $v \in V$: add two vertices $s_v^1$ 
and $s_v^2$ to $S$ and a directed link $s_v^1 \rightarrow s_v^2$ to $L$. 
We also assign a destination host $d_v$ for each vertex $v$. 

For every edge $(u,v) \in E$: add two vertices $s_{uv}$
and $t_{uv}$ to $S$. Add edges
connecting $s_{uv} \rightarrow s_{u}^1$, $s_{uv} \rightarrow s_{v}^1$,
$s_{u}^2 \rightarrow t_{uv}$ and $s_{v}^2 \rightarrow t_{uv}$. \Cref{fig:rfcomplexity} illustrates this construction.

\begin{figure}[H]
	\centering
	\begin{tikzpicture}[shorten >=0.5pt,node distance=1.5cm,on grid,auto,
	square/.style={regular polygon,regular polygon sides=4}] 
	\node[state] (s)  {$s_{uv}$}; 
	\node[state] (v1) [below right=of s] {$s_v^1$}; 
	\node[state] (v2) [below=of v1] {$s_v^2$};
	\node[state] (u1) [below left=of s] {$s_u^1$}; 
	\node[state] (u2) [below=of u1] {$s_u^2$}; 
	\node[state] (t) [below right=of u2] {$t_{uv}$};
	\node[state] (s1) [above right=of v1] {$s_{vw}$}; 	
	\node[state] (t1) [below right=of v2] {$t_{vw}$}; 	
	\path[->] 
	(s) edge  node {} (v1)
	edge  node {} (u1)
	edge [blue, dashed, bend right=100] node {} (t)
	edge [red, dashed, bend left=100] node {} (t)
	(u1) edge node {} (u2)
	(u2) edge node {} (t)
	(v1) edge node {} (v2)
	(v2) edge node {} (t)
	edge node {} (t1)
	(s1) edge node {} (v1);
	\path[-] (s1) edge node[above] {} +(1,-1);
	\path[-] (t1) edge node[above] {} +(1,1);
	\end{tikzpicture}
	\caption{Construction}
	\label{fig:rfcomplexity}
\end{figure}
If there was another edge $(v,w) \in E$, then
$s_{vw}$ has an edge connecting to $s_v^1$ and
$s_v^2$ has an edge connecting to $t_{vw}$ (shown
in \Cref{fig:rfcomplexity}). $s_{vw}$ will also 
have an edge to $s_w^1$ and similarily from $s_w^2$ to $t_{uw}$
as per the construction outlined above.

For each edge $(u,v) \in E$, we add two paths in $P$: 
$s_{uv} \rightarrow s_u^1 \rightarrow s_u^2 \rightarrow t_{uv}$
for destination host $d_v$ and 
$~s_{uv} \rightarrow s_v^1 \rightarrow s_v^2 \rightarrow t_{uv}$ 
for destination host $d_u$.
(dashed paths in \Cref{fig:rfcomplexity}). 

We now prove that if there exists a set of route-filters
$R$ such that $|R| \leq k$ such that the resulting configurations
forwards traffic along $P$, then there exists a vertex cover $VC$
of $G$ such that $|VC| \leq k$. 

For each route-filter $r \in R$, one of its endpoints 
has to be either $s_v^1$ or $s_v^2$, corresponding
to some vertex $v \in V$ (structure of topology $T$). 
We construct a set $VC(R)$ by adding the vertex $v$ 
based on the endpoints of each route-filter $r \in R$.
To show that $VC(R)$ is a vertex cover of $G$, we first
prove \Cref{lemma:diamond}.

\begin{lemma} \label{lemma:diamond}
	Atleast 1 route-filter is required to find a valid solution to the
	OSPF edge weights for each diamond formed by the input paths.
\end{lemma}
\begin{proof}
	Proof by contradiction. We can show that any diamond inconsistency 
	cannot be removed if no route-filters are set on the paths.
\end{proof}

For every edge $(u,v) \in E$, the constructed paths from 
$s_{uv}$ to $t_{uv}$ form a diamond. Thus, by lemma 1, 
the diamond corresponding to each edge in $G$ 
requires atleast one route-filter to eliminate
the inconsistency caused by the diamond, thus, one 
of vertices $\{u,v\}$ will be in $VC(R)$, and edge $(u,v)$
is covered. Thus, if $R$ eliminates all diamond inconsistencies
to find a solution to the OSPF weights, the corresponding set
$VC(R)$ covers all edges in $E$. Therefore, $VC(R)$ is a vertex
cover. 

Thus, by finding a set of route-filters $R$ such that $|R| \leq k$
such that all the diamond inconsistencies are eliminated, and there
exists OSPF weights $W$ such that the configurations forward traffic
along $P$, we can find a vertex cover $VC$ for graph $G$ such that
$|VC| \leq k$. 

This transformation is polynomial, the constructed 
network topology $T$ has $2|V| + 2|E|$ nodes, 
$|V| + 4|E|$ links and $2|E|$ paths. Therefore, OSPF
configuration synthesis with number of route-filters $\leq k$ is
NP-complete. Thus, OSPF synthesis with minimal number of 
route-filters is NP-hard. 
\end{proof}