\subsection{Problem Definition}
In this section, we formally define the problem addressed by this paper.

We represent the physical router topology as a directed graph $T=(V, L)$,
where $V$ is the set of routers and $L\subseteq V\times V$ is the set of links. 
We use the neighbour function $N(s) = \{s'\ | \ (s,s') \in L \}$ to denote 
the set of neighbour routers of $s$. 
We define $\Lambda$ to denote the set of destination IP subnets, 
distributed protocols make forwarding decisions based on the 
destination IP address/subnet.

A path $\pi = (u_1,v_1) (u_2, v_2) \ldots (u_n, v_n) \in L^*$ is a loop-free valid path if
a  vertex is not visited more than once ($\forall i.j \leq n. 
~i \not= j \wedge u_i \not= u_j$) and adjacent links in the
path share a router ($\forall i < n. ~v_i = u_{i+1}$).
If the path $\pi$ has source $s$ ($u_1=s$) and target $t$ ($v_n=t$),
we say that $\pi$ is a path from $s$ to $t$.
Given two nodes $s$ and $t$, we use $\Pi_s^t$ to denote the set of all paths
$\pi$ with source $s$ ($u_1=s$) and target $t$ ($v_n=t$).

\subsubsection{Distributed Routing Model}
\minisection{Static Routes}
Static routing occurs when a router uses a 
manually-configured routing entry, 
rather than information from dynamic protocols like
OSPF or BGP.  
For a destination $\lambda$, a static route configured
at router $r$ ensures the traffic to forward to a 
neighbouring router $r_1, r_1 \in N(r)$. 
Static routes have the highest
preference over other routes for the same destination
at a router. We define the static route
partial function $SR: V \times \Lambda \mapsto V$ such
that, for a router $r$ and destination $\lambda$, if $SR(r,\lambda)=r'\in V$, 
then traffic for destination $\lambda$ reaching node $r$ is always
forwarded to node $r'$. Using $SR$, we construct the complete
static paths between routers in the network $\Pi_{SR}$. These 
paths would have a highest preference. 

\minisection{OSPF}
Open Shortest Path First (OSPF) is a routing protocol used inside a
domain. Each router receives link state information from other routers
and uses this to create a map of the network. Positive weights
are defined for each link in the network, and 
each router used Djikstra's algorithm to
forward to the destination along the weighted shortest path 
to the router the destination IP is connected to. 

Let us
define the OSPF weight function $W: L \rightarrow \rat$ which 
maps edges of the topology to positive rational weights. 
% todo: When to talk about converting LRA to LIA.
For a given
path $\pi=l_1\ldots l_n$, we indicate the sum of weights of the
links in the path by $W(\pi)=\sum l_i$. 


We allow route-filters
to selectively disable an
edge for a given destination by  
blocking advertisements to a
particular destination along a link. 
Formally, a route-filter is a pair $(l,\lambda)\in L\times V$
which disables the link $l$ for destination $\lambda$. 
We define the route-filter function 
$RF: \Lambda \rightarrow 2^L$ which maps each destination IP
to a set of filtered links in the topology. 
Given an IP destination $\lambda\in \Lambda$, 
a legal path with respect to $RF$ and $\lambda$
is a valid path $\pi=l_1\cdots l_k$ such that for every $i$,
$l_i\not\in RF(\lambda)$.
We use $\Pi_\lambda$ to denote the set of all such paths.

We define the routing function 
$\route_{ospf}$ 
that given a 
a weight function $W$,
a route-filter function $RF$,
a source node $s$,
a target node $t$,
and destination IP 
$\lambda$,
returns the OSPF-induced paths from $s$ to $t$ for destination IP $\lambda$.
The paths between two nodes are
the shortest paths in the network
that do not cross any link that belongs to $RF(\lambda)$. We also
account the higher priority of static routes 
in the network in the routing function.
\begin{multline} \label{eq:ospfmodel}
\route_{ospf}(W,RF,s,t,\lambda) = \\
\begin{cases}
\text{$\Pi_s^t \cap \Pi_{SR}$}  & \text{if $\Pi_s^t \cap \Pi_{SR} \not=\emptyset$} \\
\text{$\{\pi \mid  \pi\in \Pi_s^t\cap \Pi_\lambda ~\wedge $} & \\
\text{$\neg\exists \pi'\in\Pi_s^t\cap \Pi_\lambda . W(\pi')< W(\pi)\}$} & \text{otherwise}
\end{cases}
\end{multline}

\minisection{BGP}
BGP is a path-vector inter-domain 
routing protocol that connects 
different autonomous systems (ASes), where each AS
comprises of one or more routers (typically managed
by a single entity). A BGP router receives routes 
from BGP peers (internal peers send iBGP routes, 
external peers send eBGP routes). Each route 
for the destination comprises of a domain path
 to the destination domain, and in the default
 scenario, a BGP router will pick the shortest
 path (in terms of domain count) to forward
 the packet for a destination. 
 
To support policy routing, BGP routers can be 
configured with a \emph{local preference} variable
to assign higher priorities to routes for particular
destinations. We define the local preference function 
$LP: V \times V \times \Lambda \mapsto \nat$ as follows.
Given a BGP router $r$, neighbouring router $n$,
and IP destination $\lambda$,
  $LP(r, n, \lambda)$ 
specifies the local preference for a route received 
for destination $\lambda$ at BGP router $r$ 
from next-hop BGP router $n$. 
On all other links the function is undefined. From 
the routes received at $r$, the router
picks the route with greatest local preference. 

Let us define the BGP routing function 
$\route_{bgp}: LP \times V \times \Lambda \rightarrow V$.
For traffic at the BGP router $r$ to
destination IP $\lambda$, $\route_{bgp}(LP,r,\lambda)$ 
is the next-hop BGP router in the path. BGP 
chooses a route with highest local preference, and
if there is a tie, it then chooses the route with smallest
AS path length~\cite{bgp}. While, the actual implementation
of BGP uses several other criteria,
\name uses these rules to synthesize BGP configurations with 
ordering maintained with respect to actual BGP implementations.
We ensure all ties are broken, and $\route_{bgp}(LP,r,\lambda)$
returns a single router. Also, static routes 
have a higher preference than BGP, therefore a static route
$SR(r, \lambda)$ will be preferred over the BGP route. We
would account for static routes for $\route_{bgp}(LP,r,\lambda)$
similarily as we defined in \Cref{eq:ospfmodel}.

\begin{algorithm} [t]
	\begin{footnotesize} 
		\caption{BGP Best Path Algorithm}
		\label{alg:bgppathrules}
		\begin{algorithmic}[1]
			\State{[Input] $R \leftarrow$ eBGP and iBGP routes for $\lambda$ at BGP router $g$} 
			\State{[Output] $r_{best}$ : Best BGP route} \newline \newline
			/* Find routes with highest \emph{local preference} */
			\State{$R_{lp} = \{r \in R ~| ~\forall r_1 \in R. ~r.local\_pref \geq r1.local\_pref\}$}
			\If{$|R_{lp}| = 1$}
			\State{$r_{best} = r \in R_{lp}$}
			\Else \newline
			\indent /* Prefer the path with the smallest \emph{AS Path length} */
			\State{$R_{as} = \{r \in R_{lp}  ~|~ \forall r_1 \in R_{lp}. ~r.path\_len \geq r_1.path\_len\} $}
			\If{$|R_{as}| = 1$}
			\State{$r_{best} = r \in R_{as}$}	
			\EndIf
			\EndIf
			\If{$r_{best}.type = eBGP$}
			\State{Redistribute $r_{best}$ to OSPF domain}
			\EndIf
			%				\State{ . }
			%				\State{Prefer the path with the lowest \emph{multi-exit discriminator} (MED).}
			%				\State{Prefer eBGP over iBGP paths.}
			%				\State{Prefer the route that comes from the BGP router with the lowest \emph{router ID}.}
		\end{algorithmic}
	\end{footnotesize}
\end{algorithm}


\minisection{OSPF+BGP+Static Routes}
We define the router domain assignment function
$\Theta: V \rightarrow \nat$ which maps each router to a domain 
(denoted by a natural number). Each domain uses OSPF as the 
intra-domain routing protocol, and BGP for the inter-domain routing
protocol. 

In domain $d$, let us consider a source router $s$ ($\Theta(s) = d$) for 
sending traffic for destination $\lambda$. Note that, the traffic
could originate from other domains, then $s$ indicates the first
router in the domain. If the target router $t$ 
connected to the destination belongs to domain $d$, then there exists
a OSPF route for $\lambda$ at $s$, and the path taken would be 
$R_{ospf}(W_d, RF_d, s,t,\lambda)$.

Consider the case of destination $\lambda$ being connected 
to an external domain, thus, traffic for destination $\lambda$
will be sent to a BGP gateway router of domain $d$ which 
will forward it to its neighbouring domains till
it reaches the internal domain of $\lambda$. A BGP gateway
router is a router $g$ such that $\exists g' \in N(g). 
~\Theta(g') \not= \Theta(g)$. 

However, not all BGP routes for $\lambda$
are distributed 
into OSPF (to prevent explosion of forwarding tables). Instead,
BGP gateways in a domain exchange routes using iBGP to choose
the best gateway, which then redistributes the external
(eBGP route) into OSPF (\Cref{alg:bgppathrules}). If multiple
gateways advertise the destination $\lambda$ into the OSPF 
domain, then the closest gateway (in terms of OSPF distance)
is chosen. We define a gateway function $G: \nat \times LP \times
W \times RF \times V \times \Lambda \rightarrow V$ which specifies
the gateway chosen for a destination based on both BGP and OSPF
configurations. Therefore, the routing function 
$\route:\Theta \times LP \times W \times RF \times V \times V \times \Lambda \rightarrow L^*$ models routing in the inter-domain network. 
For a destination $\lambda$ connected to target router $t$, the 
path from source router $s$ is described as follows:
\begin{multline}
	\route(\Theta, LP, W, RF, s, t, \lambda) = \\
	\route_{ospf}(W,RF,s,g_1=G(\Theta(s),LP,W,RF,s,\lambda), \lambda) + \\
	 (g_1, s_2=\route_{bgp}(LP, g_1, \lambda))+ \\
	\route_{ospf}(W,RF,s_2,g_2=G(\Theta(s_2),LP,W,RF,s_2,\lambda), \lambda) + \\
	\ldots \route_{ospf}(W,RF,s_n,t,\lambda)
\end{multline}

\subsubsection{Policy Compliance}
Since, most distributed routing protocols only support
destination forwarding, the predicate for a reachability
policy is of the form \texttt{dst.IP} = $\lambda$. To 
model policies for different source IP subnets, we can create
multiple reachability policies with different source routers
\texttt{dst.IP} = $\lambda$ : \texttt{s >> t} (where $s$ is
the router connected to the source IP subnet, $t$ is the router
which $\lambda$ is connected to).
We assume a set of packet classes $PC : [0,C_{pc}]$ 
and map each reachability policy to a unique integer in $PC$.
For a packet class $pc$, $src_{pc}$ denotes the source router,
$dst_{pc}$ denotes the destination router, and $\lambda_{pc}$
denotes the destination IP address. 

\begin{definition}[Induced Paths]
For a set of packet classes $PC$, the set of induced
paths by the network configurations $\Pi_{ind} = \paths(\Theta, LP$, 
$W, RF, PC)$ is defined as follows: \\
$\forall pc \in PC.$
$~\route(\Theta, LP, W, RF,$ $src_{pc}, dst_{pc}, \lambda_{pc}) \in \Pi_{ind} $
\end{definition}

\noindent We address the notion of policy-compliance. Let 
$\Psi$ denote the set of constraints corresponding to the 
path-based input policies (\Cref{tab:policysupport}). 
\begin{definition}[Policy-Compliance]
	The configurations $(\Theta,$ $LP, W, RF)$ is policy-compliant,
	$(\Theta, LP, W, RF) \models \Psi$, if the set of
	induced paths $\paths(\Theta, LP$, $W, RF, PC) \models \Psi$.
\end{definition}

However, this approach is intractable for real-world scenarios,
because simply computing static routes for the set of constraints 
$\Psi$ is computationally hard. Instead, we use a two-phased approach
to tackle the problem: we first synthesize a set of policy-compliant
paths $\Pi$, and then synthesize the configurations which induces
the set of policy-compliant paths. 
\begin{definition}[Path-Compliance]
	The configurations $(\Theta,$ $LP, W, RF)$ is policy-compliant,
	$(\Theta, LP, W, RF) \models \Psi$, if there exists a set of
	policy compliant paths $\Pi \models \Psi$ and the set of
	induced paths $\paths(\Theta, LP$, $W, RF, PC) = \Pi$.
\end{definition}

\begin{table}[!t]
\begin{small}
	\begin{center}
		\begin{tabular}{m{7.8em}  m{15.9em} } 
			{\bf Policy} & {\bf Description} \\ 
			\hline
			Reachability & There is a path from router $src$ to router $dst$ for destination $\lambda$ \\ \hline
			Reachability with \newline Ordered Waypoints & The path  from $src$ to $dst$ for destination $\lambda$ 
			traverses some switch in the set $W_1$, \ldots, then some switch in the set $W_k$.\\ \hline
			Traffic Isolation & Paths of two reachability policies $R1$ and $R2$ do not share  links \\ \hline
			Traffic Engineering  & Minimize total/max link utilization \\
		\end{tabular}
	\end{center}
	\compactcaption{\genesis path-based policy support.} \label{tab:policysupport} 
\end{small}
\end{table}


%\section{Policy Support} \label{sec:policy}
%We design a language GPL (Genesis Policy Language) for network operators to express the desired end-to-end policies in a declarative manner which is interpreted by the Genesis synthesizer to find the forwarding rules for the network topology which enforce the input policies (\cref{fig:arch}). Genesis supports the following policies : 
%% Figure of GPL's syntax
%\begin{enumerate} 
%	\item \textbf{Reachability}: $predicate : src >> dst$ \\
%	This policy specifies the packets satisying $predicate$ have ingress router $src$ and egress router $dst$, and requires rules forwarding packets satisfying $predicate$ from $src$ to $dst$. There must be no forwarding loops in the network. 
%	\item \textbf{Waypoint}: $predicate : src >> W >> dst$ \\
%	The waypoint policy is a stronger reachability policy, and specifies that packet satisfying $predicate$ with ingress router $src$ and egress router $dst$ must pass through the set of waypoints $W$ in no particular order. The waypoint policy helps operators and tenants to specify the middleboxes the packets must traverse through without worrying about order, or having to use header tags to enforce a particular order \cite{flowtags}. 
%	\item \textbf{Traffic Isolation}:  $R1 \ || \ R2$ \\
%	The traffic isolation policy ensures that the . This policy can be used to provide fairness guarantees, since the paths of $R1$ and $R2$ don't share a link, the bandwidth used by $R1$ will not affect the bandwidth used by $R2$ and vice-versa. The condition of sharing a link in the same direction is due to the fact that links are full-duplex so, traffic flowing in one direction is not affected by the traffic flowing in the other direction.
%	\item \textbf{Security Isolation}: $R1 <> R2$ \\
%	The security isolation policy is stronger than the traffic isolation policy, and ensures that the path of the reachabiltiy/waypoint policies $R1$ and $R2$ do not share a link in both directions for increased security.
%	\item \textbf{}: $sw_1 \rightarrow sw_2 : capacity$ \\
%	The link capacity policy specifies that the capacity for link $sw_1 \rightarrow sw_2$ is $capacity$, and the weights of flows traversing the link in the direction of $sw_2$ do not exceed the capacity of the link.  
%	\item \textbf{Switch Table Size}: $sw : size$ \\
%	The router table size policy is used to specify the size of the forwarding table of the router $sw$ and ensures that the number of flows traversing through $sw$ does not exceed $size$ as each flow would require a forwarding rule at the switch.
%\end{enumerate}


\subsubsection{Configuration Compliance}
We defined policy-compliance in terms of the 
paths of the packet classes satisfying path-based 
policies like isolation or traffic engineering. However,
depending on the assignment of routers to different domains,
the configurations generated may not be efficient
in deployment. therefore, we extend the notion of 
policy-compliance to define policies on configurations 
(\Cref{tab:configpolicysupport}).

\paragraph{Number and size of OSPF domains.} 
The OSPF routing protocol does not scale 
with increasing network sizes
as it uses
flooding of link-state packets. Flooding 
of updates can  
overwhelm the network when links fail. 
Ideally, operators would want to specify
limits on the size of an OSPF routing domain. 
Operators can also specify the number of OSPF domains
for administrative reasons (each domain can be
managed by a separate entity). For the  
set of domains $D = \{\Theta(r) \mid r \in V\}$:
\begin{equation}
	|D| \leq N_{D} ~\wedge \forall d \in D. ~l \leq |\{r \mid \Theta(r) = d\}| \leq u
\end{equation}

\paragraph{BGP-Compatibility} Certain 
	routers may not be suited to run BGP due to resource
	constraints. Thus, the operator can specify if a 
	router is BGP compatible or not.  
	For the set of routers $\bar{B}$ which cannot run BGP, the
	domain assignment $\Theta$ must ensure the router
	does not lie on the boundary of a domain (and therefore,
	run BGP):
\begin{equation}
	\forall r \in \bar{B}. ~\not\exists r_1 \in N(r).~\Theta(r) \not= \Theta(r_1) 
\end{equation} 

\paragraph{Optimizations}  
Ideally, \name should provide mechanisms to optimize
different configuration metrics. For example,
we allow route-filters in OSPF synthesis 
(\secref{sec:ospfsynthesis}), which may yield undesirable
endpoint resilience, thereby, synthesizing configurations
which provide a specified metric of resilience or maximize
the metric. Similarily, to enable path-based inter-domain routing, 
\name needs	to set up static routes along the path, or configure BGP 
variables like local preferences to ensure the configurations 
induce the input paths.
These increase the size of the configurations,
thus increasing the complexity of verifying correctness either 
manually or using verification tools~\cite{batfish, arc, era}. 
Therefore,
another objective \name considers is the configuration overhead
to ensure path-compliance.

\kausik{Sec 3.2 seems kind of small, comments about that?}