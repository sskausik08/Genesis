\section{Problem Statement}
We define the physical switch topology as an directed graph $T=(S, L)$,
where $S$ is the set of switches and $L$ is the set of links. 
We use the neighbour function $N(s) = \{s'\ | \ (s,s') \in L \}$ to denote 
the set of neighbour switches of $s$. 
We use $\Lambda$ to denote the set of destination IP subnets.

A path $\pi \in L^*$ is a loop-free valid path if for 
$\pi = (u_1,v_1) (u_2, v_2) \ldots (u_n, v_n)$, a 
vertex is not visited more than once ($\forall i.j \leq n. 
~i \not= j \wedge u_i \not= u_j$) and adjacent links in the
path share a router ($\forall i < n. ~v_i = u_{i+1}$).

\minisection{OSPF}
Open Shortest Path First (OSPF) is a routing protocol used inside a
domain. Each router receives link state information from other routers
and uses this to create a map of the network. This link state
information consists of link weights, and each router
forwards to the destination along the weighted shortest path 
to the router the destination IP is connected to. Let us
define the OSPF weight function $W: L \rightarrow \rat$ which 
maps edges of the topology to positive rational weights. 
% todo: When to talk about converting LRA to LIA.

As shown in \Cref{fig:ospfexample}(b), 
we require route-filters
to selectively disable an
edge for a given destination by  
blocking advertisements to a
particular destination along a link. 
Formally, a route-filter is a pair $(l,\lambda)\in L\times S$
which disables the link $l$ for destination $\lambda$. 
We define the route-filter function 
$RF: \Lambda \rightarrow 2^L$ which maps each destination IP
to a set of filtered links in the topology. 

We now define the routing function 
$\route_{ospf}: W \times RF \times V \times V \times \Lambda \rightarrow L^*$ 
for a OSPF network with edge weights $W$ and route-filters
$R$ which maps a source and target router
for destination IP to a path in the network. For a 
OSPF network without route-filters, the path chosen is
be the shortest path in the network. 
For a given
path $\pi$, we indicate the sum of weights of the
links in the path by $w(\pi)$. Therefore, for a
source router $s$, target router $t$, destination IP 
$\lambda$, the routing function is as follows:
\begin{multline}
	\route_{ospf}(W,s,t,\lambda) = \pi \implies \exists u,v. \pi=(s,u).^*(v,t) ~\wedge\\ 
	(\forall \pi' \exists u',v'. ~\pi'=(s,u').^*(v',t) \wedge w(\pi') \geq w(\pi))
\end{multline}

\noindent 
For a OSPF network with route-filters set, the
path chosen in the network will be the shortest path
excluding disabled links for the destination IP. The routing
function $\route_{ospf}(W,RF,s,t,\lambda)$ for a
source router $s$, target router $t$, destination IP 
$\lambda$: 
\begin{multline}
\route_{ospf}(W,RF,s,t,\lambda) = \pi \implies 
\\ \exists u,v. \pi=(s,u).^*(v,t) ~\wedge 
\forall l \in \pi.~l \not\in RF(\lambda) ~\wedge  \\
(\forall \pi' \exists u',v'. ~\pi'=(s,u').^*(v',t) \wedge \\
\forall l' \in \pi'.~l' \not\in RF(\lambda) ~\wedge
w(\pi') \geq w(\pi))
\end{multline}


