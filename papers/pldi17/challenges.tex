\subsection{Challenges}
\minisection{OSPF Synthesis}
For the OSPF protocol, configurations assign
weights to links between routers (directed edges
in the network topology). When the network has
to forward a packet from $s$ to $t$, the
OSPF routers uses 
Djikstra's algorithm to chose the
shortest weighted path from $s$ to $t$. Thus,
given input paths, \name finds edge weights 
(which are global for all paths) such that 
the shortest path through the network
for these endpoints exactly match the input paths. 
For example in \Cref{fig:ospfexample}(a), if the input
path is $s\rightarrow r_1 \rightarrow t$ for
destination IP $\lambda$, \name assigns
edge weights such that the input path has a strictly
smaller weight ($w=1+2$) than the other path $s \rightarrow t$ 
($w=5$). Thus, the OSPF routers will forward traffic for
$\lambda$ from $s$ to $t$ through $r_1$. \name 
efficiently computes weights by generating constraints
in the theory of Linear rational arithmetic (LRA) and
uses fast off-the-shelf LP Solvers 
(\secref{sec:ospfsynthesis}). 

However, given a set of paths as input, there may
not exist a solution to the edge weights. Consider the 
input paths as shown in \Cref{fig:ospfexample}(b). 
Both the red and blue paths are required 
to be the unique shortest path between $s$ to $t$
and, clearly, this is cannot be enforced for any 
choice of the edge weights (as weights correspond 
to all destinations). 
One way to synthesize configurations in this scenario 
is to ``disable'' the edge
$(s, t)$ for destination $\lambda_1$.
Using this technique, 
there is only one possible path from $s$ to $t$
for destination $\lambda_1$ ($s\rightarrow r_1 \rightarrow t$),
therefore is chosen as the shortest path. For 
destination $\lambda_2$, the $s\rightarrow t$ path
has a smaller weight ($w=1$) than the
$s\rightarrow r_1 \rightarrow t$ path ($w=1+2$), therefore,
both traffic is forwarded through the input paths for
both the destinations. 
This blocking mechanism is called a route-filter, and
we modify \name's OSPF synthesis algorithm to support
route-filtering (\secref{sec:filtering}).
 

\begin{figure}
	\centering
	\subfloat[Edge Weights]{
		\raisebox{0.5cm}{\resizebox {0.5\columnwidth} {!} {
	\begin{tikzpicture}[shorten >=0.5pt,node distance=,on grid,auto,
	square/.style={regular polygon,regular polygon sides=4}] 
	\node[state] at (0,0) (s)  {$s$}; 
	\node[state] at (1.8,1) (v1)  {$r_1$}; 
	\node[state] at (3.6, 0)(t) {$t$};
	\node[state, rectangle] at (5, 0) (d1) {$\lambda$};
	\path[->] 
	(s) edge node {1} (v1)
	edge  node {5} (t)
	edge [red, dashed, bend left=90] node {} (t)
	(v1) edge node {2} (t)
	(t) edge [red, dashed] node {} (d1);
	\end{tikzpicture}
	}}}
	\subfloat[Route-Filters]{
			\resizebox {0.5\columnwidth} {!} {
		\begin{tikzpicture}[shorten >=0.5pt,node distance=,on grid,auto,
		square/.style={regular polygon,regular polygon sides=4}] 
		\node[state] at (0,0) (s)  {$s$}; 
		\node[state] at (2, 1) (v1)  {$r_1$}; 
		\node[state] at (4, 0)(t) {$t$};
		\node[state, rectangle] at (5.5, 0.75) (d1) {$\lambda_1$};
		\node[state, rectangle] at (5.5, -0.75) (d2) {$\lambda_2$};
		\path[->] 
		(s) edge node {1} (v1)
		edge  node [above] {1} node [below] {$rf((s,t),\lambda_1)$} (t)
		edge [red, dashed, bend left=90] node {} (t)
		edge [blue, dashed, bend right=45] node {} (t)
		(v1) edge node {2} (t)
		(t) edge [red, dashed] node {} (d1)
		(t) edge [blue, dashed] node {} (d2);
		\end{tikzpicture}
	}}
	\compactcaption{OSPF edge weights and filters such that the
		the routers forward traffic for destination along
		the input paths (dashed arcs).}
	\label{fig:ospfexample}
\end{figure}



\minisection{Dynamic Domain assignment}
The OSPF routing protocol does not scale 
with increasing network sizes
as it uses reliable
flooding of link-state packets. Flooding 
of updates can  
overwhelm the network when links fail. 
Ideally, operators would want to specify
limits on the size of an OSPF routing domain. 
Thus, a network could be 
split into multiple continuous OSPF domains,
which exchange routes across domains using
a inter-domain protocol like BGP.

We augment \name to synthesize 
inter-domain routing configurations 
such that each router is assigned to
a particular domain. 
Each domain is continous (all routers
are reachable to one another) and 
uses OSPF for intra-domain routing.
Domains exchange routes among  
themselves using BGP, a path-vector 
protocol which primarily selects routes by 
the number of domains in the route. 
However, \name can 
use BGP's powerful path selection metrics 
like local preferences such that  
paths with greater path lengths are selected.
OSPF has better convergence times than BGP,
thus, it is advantageous to use OSPF for 
routing in small domains and using BGP for
inter-domain routing. 

We consider the network to be managed by a 
single entity, therefore, the network can 
be split into domains\footnote{
The Internet is split into domains depending 
on ownership.} in numerous ways. Depending
on the network and input data plane, a certain
domain assignment of routers 
can optimize different metrics, like the
inter-domain configuration overhead (like BGP local
preferences and static routes) and OSPF route-filters
in the different domains. Thus, instead of operators
specifying a static domain assignment, \name stochastically
searches for the best domain assignment from the space of 
allowed assignments (for e.g., adhering to domain size limits)
to optimize different metrics. \name uses 
\emph{Markov chain Monte Carlo} sampling (MCMC) to perform
this stochastic search, and operators can specify parameters
to tune the cost function used in the search to assign priorities
to different metrics. 

\todo{Write about paper outline}