\section{Conclusion}

We presented \name, a system for automatically synthesizing a
distributed network control plane (i.e., router configurations) from
high-level policies. \name allows operators to provide constraints on
how they wish their control plane to be designed---e.g., the number and size
of constituent domains---as well as optimize metrics that are
central to management and operations---e.g., ensuring network paths
are resilient to failures and that  synthesized configurations  are
simple. \name leverages a novel combination of ideas: Linear rational
arithmetic (LRA) and a unsatisfiable-core-based procedure for synthesizing
intra-domain OSPF configurations, and MCMC sampling to stochastically
search for the best way to partition the network into multiple domains
and to configure BGP. To the best of our knowledge, \name is the first
configuration synthesis system to provide comprehensive coverage of
many operational goals (support for many types of policies,
optimization of management and operational metrics, domain
decomposition, etc). Detailed evaluation on realistic topologies shows
that \name can synthesize policy-compliant configurations quickly enough to be
useful in practice.
