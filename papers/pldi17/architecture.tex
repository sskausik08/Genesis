\section{Architecture} \label{sec:architecture}
In this section, we describe the architecture of our system.
Since directly generating policy-compliant configurations
is a challenging problem, we use a two-phase approach in which
we first synthesize policy-compliant \emph{paths}
using the tool \genesis~\cite{genesis} and then present
a new tool, \name, that synthesises configuration that produce the
paths we synthesized in the first phase. 
Since the first phase may produce \emph{bad} paths for which the
configuration synthesis is complicated or not possible,
we then propose ways to restart the algorithm by computing new paths.

\subsection{\genesis: From Policies to Paths} \label{sec:genesis}
The first phase of our approach produces a set of paths adhering to
different policies using \genesis~\cite{genesis}, a network management
system which synthesizes forwarding tables enforcing the policies. 
Some of the policies supported by \genesis are specified in 
\Cref{tab:policysupport}. 
Due to the rich policy language, generating policy-compliant paths is an NP-complete problem;
\genesis leverages fast off-the-shelf SMT solvers to provide
support for policy enforcement for a diverse set of policies.
Given a set of policies, \genesis generates a set of constraints 
over propositional logic (SAT) and Linear rational arithmetic (LRA),
such that the solutions to these constraints are forwarding
tables, which can be used to extract the 
policy-compliant paths.

\genesis uses the relation $Fwd \subseteq V \times V \times PC $  
to capture 
the network forwarding behavior,---i.e. 
$(r_1, r_2$, $pc)\in$ $Fwd$ if 
$r_1$ forwards packets of class $pc$ to router $r_2$---
and 
the relation $Reach \subseteq V \times PC \times \nat$ 
to captures
the path reachability---i.e. $(r, pc, k)\in Reach$ if 
the router $r$ is reachable in the path from the source
router of packet class $pc$ in exactly $k$ steps. Based on
the input policies, \genesis generates constraints over the $Fwd$
and $Reach$ relations and finds a solution to the relations. 
\genesis then extracts the concrete paths from the solution to the $Fwd$ and $Reach$
relations.

\paragraph{Modifications to \genesis}
We  discuss how the
constraints of \genesis~\cite{genesis}
need to be modified to avoid generating paths that 
cannot be enforced by any routing configuration.

OpenFlow switches~\cite{openflow} support match predicates for
forwarding rules on different packet header fields like source
and/or destination IP address, ports etc. However, legacy protocols
like OSPF and BGP only support destination-based forwarding, 
forcing the paths to a destination subnet to 
form a directed tree rooted at the destination. 
Thus,
at any router, there exists at most one forwarding rule per destination. 
Since \genesis was built for
SDN management,  a switch may forward to different switches
packets that are directed to the same destination,
and thus cannot be induced by any router
configuration. 

\loris{yet another name for switches $sw$, before
it was $r$ etc.}
We add constraints to \genesis to ensure that
if any two paths to a destination subnet intersect at a router,
the subsequent downstream paths to the destination from the
router do not \emph{diverge} at a router.  
We define the relation $Reach^*(sw,pc)$ to model reachability 
of $sw$ in the path of $pc$, i.e., it can be reached in zero or more
steps from the source. We can be expressed $Reach^*(sw,pc)$ 
in the network forwarding model of \genesis as:
\loris{$\mu$ not defined}
\begin{equation}
	Reach^*(sw,pc) = \bigvee_{k \in [0, \mu]} Reach(sw, pc, k)
\end{equation}
$\mu$ denotes the synthetic limit of the length of the path. 
$Reach(sw, pc, k)$ is valid if $sw$ is reachable in the path of
class $pc$ in $k$ steps. Thus, to enforce the destination-based
forwarding using $Reach^*(sw,pc)$, we add
constraints for every pair ($pc_1$,$pc_2$) having the same 
 destination subnet:
 \begin{multline}
 \forall sw. Reach^*(sw, pc_1) \wedge Reach^*(sw, pc_2) \implies \\ \bigvee_{n \in N(sw)} Fwd(sw, n, pc_1) \wedge Fwd(sw, n, pc_2)
 \end{multline}
 \loris{
 I think the constraint should be
 $\bigwedge_{n \in N(sw)} Fwd(sw, n, pc_1) \Leftrightarrow Fwd(sw, n, pc_2)$.
 }
 $Fwd(sw_1, sw_2,pc)$ is valid if $sw_1$ forwards $pc$ to next-hop $sw_2$ and
 $N(sw)$ denotes the neighbours of $sw$. Basically, 
 if a switch is reachable for both packet classes, 
 then the next switch must be the same for both classes
 (paths will not diverge), and thus, the paths obtained
 from \genesis for a destination subnet will form a 
 directed destination tree. 

\subsection{\name: From Paths to Router Configurations} 
\loris{consider adding more text to this sec}
After we have computed a set of policy-compliant paths,
we need to solve the path-compliance problem using such paths.
To solve this problem, we propose the tool \name, which
combines techniques in constraint solving and randomized search
to efficiently generate router configurations for the paths obtained from \genesis.
For a given domain division of the network,
\name uses linear constraints to generate OSPF weights and BGP preferences.
If the constraints fail, \name uses the unsatisfiable cores to
identify where to place route-filters.
To decide how to split the network into domains,
\name uses Markov Chain Monte Carlo (MCMC) search to find
domain assignments that satisfy the configuration policies and have good resilience.
When the MCMC search does not progress and cannot find good solutions,
we restart the search by asking \genesis to generate a new set of paths.
These techniques are described in detail in the next two sections.

%\subsection{Challenges}
%\minisection{OSPF Synthesis}
%For the OSPF protocol, configurations assign
%weights to links between routers (directed edges
%in the network topology). When the network has
%to forward a packet from $s$ to $t$, the
%OSPF routers uses 
%Djikstra's algorithm to chose the
%shortest weighted path from $s$ to $t$. Thus,
%given input paths, \name finds edge weights 
%(which are global for all paths) such that 
%the shortest path through the network
%for these endpoints exactly match the input paths. 
%For example in \Cref{fig:ospfexample}(a), if the input
%path is $s\rightarrow r_1 \rightarrow t$ for
%destination IP $\lambda$, \name assigns
%edge weights such that the input path has a strictly
%smaller weight ($w=1+2$) than the other path $s \rightarrow t$ 
%($w=5$). Thus, the OSPF routers will forward traffic for
%$\lambda$ from $s$ to $t$ through $r_1$. \name 
%efficiently computes weights by generating constraints
%in the theory of Linear rational arithmetic (LRA) and
%uses fast off-the-shelf LP Solvers 
%(\secref{sec:ospfsynthesis}). 
%
%However, given a set of paths as input, there may
%not exist a solution to the edge weights. Consider the 
%input paths as shown in \Cref{fig:ospfexample}(b). 
%Both the red and blue paths are required 
%to be the unique shortest path between $s$ to $t$
%and, clearly, this is cannot be enforced for any 
%choice of the edge weights (as weights correspond 
%to all destinations). 
%One way to synthesize configurations in this scenario 
%is to ``disable'' the edge
%$(s, t)$ for destination $\lambda_1$.
%Using this technique, 
%there is only one possible path from $s$ to $t$
%for destination $\lambda_1$ ($s\rightarrow r_1 \rightarrow t$),
%therefore is chosen as the shortest path. For 
%destination $\lambda_2$, the $s\rightarrow t$ path
%has a smaller weight ($w=1$) than the
%$s\rightarrow r_1 \rightarrow t$ path ($w=1+2$), therefore,
%both traffic is forwarded through the input paths for
%both the destinations. 
%This blocking mechanism is called a route-filter, and
%we modify \name's OSPF synthesis algorithm to support
%route-filtering (\secref{sec:filtering}).
%
%
%\begin{figure}
%	\centering
%	\subfloat[Edge Weights]{
%		\raisebox{0.5cm}{\resizebox {0.5\columnwidth} {!} {
%				\begin{tikzpicture}[shorten >=0.5pt,node distance=,on grid,auto,
%				square/.style={regular polygon,regular polygon sides=4}] 
%				\node[state] at (0,0) (s)  {$s$}; 
%				\node[state] at (1.8,1) (v1)  {$r_1$}; 
%				\node[state] at (3.6, 0)(t) {$t$};
%				\node[state, rectangle] at (5, 0) (d1) {$\lambda$};
%				\path[->] 
%				(s) edge node {1} (v1)
%				edge  node {5} (t)
%				edge [red, dashed, bend left=90] node {} (t)
%				(v1) edge node {2} (t)
%				(t) edge [red, dashed] node {} (d1);
%				\end{tikzpicture}
%			}}}
%			\subfloat[Route-Filters]{
%				\resizebox {0.5\columnwidth} {!} {
%					\begin{tikzpicture}[shorten >=0.5pt,node distance=,on grid,auto,
%					square/.style={regular polygon,regular polygon sides=4}] 
%					\node[state] at (0,0) (s)  {$s$}; 
%					\node[state] at (2, 1) (v1)  {$r_1$}; 
%					\node[state] at (4, 0)(t) {$t$};
%					\node[state, rectangle] at (5.5, 0.75) (d1) {$\lambda_1$};
%					\node[state, rectangle] at (5.5, -0.75) (d2) {$\lambda_2$};
%					\path[->] 
%					(s) edge node {1} (v1)
%					edge  node [above] {1} node [below] {$rf((s,t),\lambda_1)$} (t)
%					edge [red, dashed, bend left=90] node {} (t)
%					edge [blue, dashed, bend right=45] node {} (t)
%					(v1) edge node {2} (t)
%					(t) edge [red, dashed] node {} (d1)
%					(t) edge [blue, dashed] node {} (d2);
%					\end{tikzpicture}
%				}}
%				\compactcaption{OSPF edge weights and filters such that the
%					the routers forward traffic for destination along
%					the input paths (dashed arcs).}
%				\label{fig:ospfexample}
%			\end{figure}
%			
%			
%			
%\minisection{Dynamic Domain assignment}
%The OSPF routing protocol does not scale 
%with increasing network sizes
%as it uses reliable
%flooding of link-state packets. Flooding 
%of updates can  
%overwhelm the network when links fail. 
%Ideally, operators would want to specify
%limits on the size of an OSPF routing domain. 
%Thus, a network could be 
%split into multiple continuous OSPF domains,
%which exchange routes across domains using
%a inter-domain protocol like BGP.
%
%We augment \name to synthesize 
%inter-domain routing configurations 
%such that each router is assigned to
%a particular domain. 
%Each domain is continous (all routers
%are reachable to one another) and 
%uses OSPF for intra-domain routing.
%Domains exchange routes among  
%themselves using BGP, a path-vector 
%protocol which primarily selects routes by 
%the number of domains in the route. 
%However, \name can 
%use BGP's powerful path selection metrics 
%like local preferences such that  
%paths with greater path lengths are selected.
%OSPF has better convergence times than BGP,
%thus, it is advantageous to use OSPF for 
%routing in small domains and using BGP for
%inter-domain routing. 
%
%We consider the network to be managed by a 
%single entity, therefore, the network can 
%be split into domains\footnote{
%	The Internet is split into domains depending 
%	on ownership.} in numerous ways. Depending
%on the network and input data plane, a certain
%domain assignment of routers 
%can optimize different metrics, like the
%inter-domain configuration overhead (like BGP local
%preferences and static routes) and OSPF route-filters
%in the different domains. Thus, instead of operators
%specifying a static domain assignment, \name stochastically
%searches for the best domain assignment from the space of 
%allowed assignments (for e.g., adhering to domain size limits)
%to optimize different metrics. \name uses 
%\emph{Markov chain Monte Carlo} sampling (MCMC) to perform
%this stochastic search, and operators can specify parameters
%to tune the cost function used in the search to assign priorities
%to different metrics. 