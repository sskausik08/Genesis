\section{Introduction}
Programming networks to correctly forward flows according to user- and
application-induced policies is difficult and
error-prone~\cite{troubleshooting, bgpmisconfig}. At least three
common characteristics of network policies are to blame: (1) a network
may need to satisfy several types of policies, including reachability,
isolation, service chaining, resilience, and traffic engineering \loris{in 
this paper we don't do traffic eng do we?}; (2)
the network must provide certain guarantees in the event of failures;
and (3) most policies are global---i.e., they concern end-to-end
paths, not individual devices/links.

%(2) SDN tries to make realization centralized like policies, but it is hard
%to ensure link failures don't impact the controller's ability to update the
%network to accommodate the failures; a distributed SDN controller doesn't fix
%this problem. Legacy control planes are preferable from a failure
%tolerance-perspective, but determining the appropriate distributed
%realization of policies is hard.

The global nature of network policies is one motivation for
software-defined networking (SDN). SDN allows paths to be centrally
computed over a global view of the network. However, it is difficult
to ensure forwarding paths are correctly computed and installed in the
presence of failures, even if the SDN controller is
distributed~\cite{hasdn}.  Traditional control planes that rely on
distributed protocols to compute paths offer greater fault tolerance,
but determining the appropriate distributed realization of policies is
hard.

%(3) Our goal is to automate the process of producing this distributed
%realization. We want to handle a wide range of policies and leverage the many
%protocols/mechanisms available in legacy control planes for realizing these
%policies.


%(4) We can view the control plane as a program whose input is set of
%endpoints and the state of the links in the network and whose output is a set
%of paths.  Our goal is to synthesize this program.

Our goal is to {\em automate the process of creating a correct
  distributed realization of policies in a traditional control
  plane}. We wish to handle a wide range of policies, e.g.,
reachability, service chaining, and traffic engineering, to meet
applications' diverse security and compliance requirements. With
increasing sizes of networks, we must further provide support for
realizing hierarchical control planes to ensure scalability,
necessitating the use of multiple intra- and inter-domain routing
protocols (e.g., OSPF and BGP). Finally, we want to ensure the
generated control plane satisfies certain failure-tolerance
properties. Thus, our work contrasts with prior efforts~\cite{netegg,
  propane, merlin,simple,fattire, netkat, netkatcompiler, sol}, which
generate SDN- or BGP-specific control planes for a limited range of
policies (e.g., peering).

%\kausik{Do we need this paragragh?}
%Although a distributed control plane is composed of a collection of routing
%processes running on individual switches, we can view the control plane as a
%single program. The input is a set of endpoints plus the state of network
%links; the output of this program is a set of paths that conform to the
%policies encoded in the program. This view inspires us to explore {\em program
%	synthesis} as a mechanism for generating a policy-compliance control plane.


The problem of synthesizing router configurations
such that the distributed control plane  
generates policy-compliant paths 
is computationally hard. 
%In our setting, the specifications are the policies we want to realize and the programs 
%are the control plane.
First, even generating a set of policy-compliant 
paths for a SDN  is 
computationally hard (e.g., enforcing isolated
paths is NP-complete). 
Second, to infer the concrete
set of paths induced by network configurations, 
one has to incorporate
into synthesis
complex concepts such as shortest path algorithms
using theories like propositional logic (SAT) 
and linear rational
arithmetic (LRA). Even with recent 
advances in Satisfiability Modulo Theories
(SMT) solvers, 
approaches that directly generate configurations  from policies
do not scale to
even moderately-sized networks or 
sets of policies~\cite{synet}.

%(6) We propose a two-phased approach that synthesizes a partially-correct
%control plane from examples, and then applies a series of transformations to
%produce a fully policy-compliant control plane.

Instead, we propose \name, a two-phased approach to synthesizing
configurations.  First, using the tool \genesis~\cite{genesis}, we synthesize a
concrete set of paths---i.e., the forwarding state---satisfying all
the specified policies, such as, reachability, waypoint, isolation,
and traffic engineering.
%a partially-correct control plane from a set of example paths, such
%as those contained within a sample data plane. 
Second, we generate
domain-level split of the routers in the network,
and corresponding 
intra- and inter-domain router configurations
that induce the forwarding
state synthesized by \genesis. We show that this approach makes the synthesis tractable, and we can
perform multiple iterations to eventually produce policy-compliant
configurations.

%% We present \name, 
%% a framework to synthesize router configurations which 
%% converge to a forwarding state such traffic is
%% forwarded through the policy-compliant paths. 
%% \name allows operators to use a combination of 
%% protocols to impose a hierarchy of domains in their network.
To elaborate, in \name, each router is mapped to a domain and runs
OSPF for intra-domain routing, while domains use BGP for inter-domain
routing. We use constraint solving in Linear rational arithmetic (LRA)
and a learning-based procedure which extracts information from
unsatisfiable cores for synthesizing intra-domain OSPF configurations.
BGP configuration can be directly extracted from the domain mapping and the paths.
We use MCMC sampling to stochastically search for the best way to
partition the network into multiple domains which satisfy hard
operator policies on configurations such as size of the each OSPF
domain. Further, \name allows operators to optimize two key management
objectives. The first is maximizing endpoint resiliency \loris{revise this after deciding what to do
about resiliency vs rf}, as \name
requires route filters to induce certain paths, thereby affecting
failure resilience. The second minimizing the configuration overhead imposed
by inter-domain routing (BGP and static routes), which helps decrease
configuration complexity, thus, easing automated
verification~\cite{batfish, arc, era} and improving readability for
performing future manual changes.

\aditya{2-3 sentences on key experimental results. The following
  sentence could be part of that.}  On average, we can increase
endpoint resilience by $1.6\times$ and reduce configuration overhead
(static routes and BGP variables) by $0.3\times$ using MCMC sampling.


%% Automatically synthesizing distributed realizations 
%% of network policies is an
%% important step towards simplifying 
%% network management and providing an 
%% SDN-like interface for programming networks 
%% running distributed routing protocols. 
%% Our approach is an important
%% contribution towards the vision of 
%% intent-driven networking~\cite{intent}.

\minisection{Contributions} We make the following contributions:
\begin{itemize}
	\item \name: a modular framework 
	for policy enforcement in `traditional' networks
	by synthesizing router configurations from policy-compliant paths (\S~\ref{sec:architecture}). 
	\item Algorithms that synthesize OSPF and BGP routing configurations that use
	constraint solving in Linear rational arithmetic (LRA) and 
	a procedure to extracts OSPF configurations from 
	unsatisfiable cores (\S~\ref{sec:config-synthesis}). 
	\item Stochastic search to find 
	partitions of the network into multiple routing domains which
	satisfy policies and optimization objectives on configurations (\S~\ref{sec:synth-dom-ass}).
	\item An implementation of \name and evaluation of the 
	intra- and inter-domain configuration synthesis for different
	topologies and workloads (\S~\ref{sec:evaluation}). 
\end{itemize}
