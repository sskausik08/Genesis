\subsection{Resilient \ARCs to device configurations} \label{sec:configs}
To deploy the resulting control plane, the \ARC must be
transformed to actual device configurations expressed in a
generic~\cite{openconfig} or vendor-specific~\cite{ciscoios} language. \ARC's
semantics are equivalent to the semantics of OSPF and layer-3 access control
lists (ACLs). Thus, we can trivially compile the \ARC into device
configurations by: (1) setting the OSPF link weights to the edge weights of
the \ARC, and (2) adding a layer-3 ACL for each route-filter in the ARC.
However, the resulting configurations may be overly
complex~\cite{complexitymetrics} or use features that make the network more
prone to failures~\cite{mpa-imc15}. Consequently, part of our future work is to
explore how to generate more optimal configurations from \ARCs.

%\aaron{TODO: talk about the task of generating configurations from an \ARC}
%Gember-Jacobson et al.~\cite{arc} describe the process 
%of generating the \ARC from the device configurations by transforming 
%the concrete network protocol parameters to create weighted digraphs. 
%Our three-phased architecture to synthesize distributed control planes
%from policies produces the \ARC which has to be compiled to 
%individual device configurations, which is the inverse of 
%the process tackled by Gember-Jacobson et al. ~\kausik{something more?}
%
%For example, suppose the underlying network comprises 
%one complete OSPF domain. For this, we can trivially
%compile the \ARC to individual device configurations by: (1)
%setting the OSPF link weights to the edge weights of the \ARC, and
%(2) adding a route-filter or Access Control List (ACL) for the 
%route-filters in the ARC. However, the compilation process becomes
%complicated if the network consists of multiple OSPF 
%domains, and use BGP for intra-domain communication. 
