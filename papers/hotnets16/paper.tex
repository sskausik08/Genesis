\documentclass{hotnets16}

\usepackage{times}  
\usepackage{epsfig}
\usepackage[TABBOTCAP]{subfigure}
\usepackage{tabularx}
\usepackage{graphicx} 
\usepackage{color}
\usepackage{xspace}
\usepackage{thumbpdf}
\usepackage{listings}
\usepackage{verbatim}
\usepackage{hyperref}
\usepackage{booktabs}
\usepackage{colortbl}

\hypersetup{pdfstartview=FitH,pdfpagelayout=SinglePage}

\setlength\paperheight {11in}
\setlength\paperwidth {8.5in}
\setlength{\textwidth}{7in}
\setlength{\textheight}{9.25in}
\setlength{\oddsidemargin}{-.25in}
\setlength{\evensidemargin}{-.25in}
%\setlength{\headsep}{0in}
%\pagenumbering{arabic}

\begin{document}

\conferenceinfo{HotNets 2016} {}
\CopyrightYear{2016}
\crdata{X}
\date{}

%%%%%%%%%%%% THIS IS WHERE WE PUT IN THE TITLE AND AUTHORS %%%%%%%%%%%%

\title{HotNets 2016 Paper}

\author{Anonymous}

\maketitle

%\thispagestyle{empty}

%%%%%%%%%%%%%  ABSTRACT GOES HERE %%%%%%%%%%%%%%
\subsection*{Abstract}

Everybody loves TCP~\cite{vanjacobson}. The paper body has two copies
of the call for papers text to show how a standard text page should be
formatted.

\section{Call for Papers}

The 15th ACM Workshop on Hot Topics in Networks (HotNets 2016) will
bring together researchers in computer networks and systems to engage
in a lively debate on the theory and practice of computer networking.
HotNets provides a venue for presenting innovative ideas 
and for debating future research agendas in networking.

We invite researchers and practitioners to submit short position
papers. The workshop seeks to foster discussions
that can shape research agendas for the networking community as a
whole. Thus, we strongly encourage papers that identify fundamental
open questions, offer a constructive critique of the state of
networking research, or present and explain new opportunities as well
as challenges.

We also encourage submissions of early-stage work describing promising
but unproven ideas. Submissions can, for example, advocate a new
approach, re-frame or debunk existing work, report unexpected early
results from a deployment, or propose new evaluation
methodologies. Novel ideas need not be supported by full evaluations; 
well-reasoned arguments or preliminary
evaluations can support the possibility of the paper’s claims. 
When fully developed and supported by more experimental
evidence, we expect work to be published at fulllength conferences
such as SIGCOMM, SOSP, OSDI, SenSys, NSDI, MobiCom, MobiSys, PODC,
CoNEXT, or INFOCOM. Finished work that fits in a short paper is 
is likely a better fit with the short-paper tracks at either CoNEXT or IMC.

HotNets takes a broad view of networking research. This includes new ideas relating to
(but not limited to) mobile, wide-area, data-center, home, and enterprise networks using a
variety of link technologies (wired, wireless, visual, and acoustic), as well as social networks
and network architecture. It encompasses all aspects of networks, including (but not limited to)
packet-processing hardware and software, virtualization, mobility, provisioning and resource management,
performance, energy consumption, topology, robustness and security, measurement, diagnosis,
verification, privacy, economics and evolution, interactions with applications, and usability
of underlying networking technologies.

Position papers will be selected based on originality, likelihood of
stimulating insightful discussion at the workshop, and technical
merit. Accepted papers will be posted online prior to the workshop and
will be published in the ACM Digital Library, thereby widely
disseminating the ideas discussed at the workshop.

\section{Workshop Participation}

HotNets attendance is limited to roughly 90 people to facilitate
lively discussion. Invitations will be allocated first to one author
of each paper, HotNets organizers and committee members, and
conference sponsors. To promote a more inclusive workshop, HotNets
will also make a limited number of open registration slots available
to the community.  

\section{Submission Instructions}

Submitted papers must be no longer than 6 pages (10 point font, 12
point leading, 7 inch by 9.25 inch text block) including all content except
references. Authors can take up to one extra page for references
beyond the 6 pages. The submission site will provide a style file and
example paper so authors can check that their paper matches in
formatting. All submissions must be blind: submissions must not
indicate the names or affiliations of the authors in the paper. Only
electronic submissions in PDF will be accepted. Submissions must be
written in English, render without error using standard tools (e.g.,
Acrobat Reader), and print on US Letter paper. Papers must contain
novel ideas and must differ significantly in content from previously
published papers and papers under simultaneous submission.

\section{Important Dates}

{
\small 
\begin{tabular}{ll}
Abstract registration: 	        & July 10, 2016 (11:59PM GMT) \\
Paper submission: 	        & July 17, 2016 (11:59PM GMT) \\
%Notification of decision: 	& September 21, 2016 \\
%Camera-ready submission: 	& October 16, 2016 \\
Workshop dates: 	        & November 9-10, 2016 \\
\end{tabular}
}

\section{Call for Papers}

The 15th ACM Workshop on Hot Topics in Networks (HotNets 2016) will
bring together researchers in computer networks and systems to engage
in a lively debate on the theory and practice of computer networking.
HotNets provides a venue for presenting innovative ideas
and for debating future research agendas in networking.

We invite researchers and practitioners to submit short position
papers. The workshop seeks to foster discussions
that can shape research agendas for the networking community as a
whole. Thus, we strongly encourage papers that identify fundamental
open questions, offer a constructive critique of the state of
networking research, or present and explain new opportunities as well
as challenges.

We also encourage submissions of early-stage work describing promising
but unproven ideas. Submissions can, for example, advocate a new
approach, re-frame or debunk existing work, report unexpected early
results from a deployment, or propose new evaluation
methodologies. Novel ideas need not be supported by full evaluations;
well-reasoned arguments or preliminary
evaluations can support the possibility of the paper’s claims.
When fully developed and supported by more experimental
evidence, we expect work to be published at fulllength conferences
such as SIGCOMM, SOSP, OSDI, SenSys, NSDI, MobiCom, MobiSys, PODC,
CoNEXT, or INFOCOM. Finished work that fits in a short paper is
is likely a better fit with the short-paper tracks at either CoNEXT or IMC.

HotNets takes a broad view of networking research. This includes new ideas relating to
(but not limited to) mobile, wide-area, data-center, home, and enterprise networks using a
variety of link technologies (wired, wireless, visual, and acoustic), as well as social networks
and network architecture. It encompasses all aspects of networks, including (but not limited to)
packet-processing hardware and software, virtualization, mobility, provisioning and resource management,
performance, energy consumption, topology, robustness and security, measurement, diagnosis,
verification, privacy, economics and evolution, interactions with applications, and usability
of underlying networking technologies.

Position papers will be selected based on originality, likelihood of
stimulating insightful discussion at the workshop, and technical
merit. Accepted papers will be posted online prior to the workshop and
will be published in the ACM Digital Library, thereby widely
disseminating the ideas discussed at the workshop.

\section{Workshop Participation}

HotNets attendance is limited to roughly 90 people to facilitate
lively discussion. Invitations will be allocated first to one author
of each paper, HotNets organizers and committee members, and
conference sponsors. To promote a more inclusive workshop, HotNets
will also make a limited number of open registration slots available
to the community.

\section{Submission Instructions}

Submitted papers must be no longer than 6 pages (10 point font, 12
point leading, 7 inch by 9.25 inch text block) including all content except
references. Authors can take up to one extra page for references
beyond the 6 pages. The submission site will provide a style file and
example paper so authors can check that their paper matches in
formatting. All submissions must be blind: submissions must not
indicate the names or affiliations of the authors in the paper. Only
electronic submissions in PDF will be accepted. Submissions must be
written in English, render without error using standard tools (e.g.,
Acrobat Reader), and print on US Letter paper. Papers must contain
novel ideas and must differ significantly in content from previously
published papers and papers under simultaneous submission.

\section{Important Dates}

{
\small
\begin{tabular}{ll}
Abstract registration:          & July 10, 2016 (11:59PM GMT) \\
Paper submission:               & July 17, 2016 (11:59PM GMT) \\
%Notification of decision:      & September 21, 2016 \\
%Camera-ready submission:       & October 16, 2016 \\
Workshop dates:                 & November 9-10, 2016 \\
\end{tabular}
}






\section*{Acknowledgments}

\bibliographystyle{abbrv} 
\begin{small}
\bibliography{refs}
\end{small}
\label{last-page}

\end{document}

