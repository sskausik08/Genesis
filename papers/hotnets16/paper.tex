\documentclass{hotnets16}

\usepackage{times}  
\usepackage{epsfig}
\usepackage[TABBOTCAP]{subfigure}
\usepackage{tabularx}
\usepackage{graphicx} 
\usepackage{color}
\usepackage{xspace}
\usepackage{thumbpdf}
\usepackage{listings}
\usepackage{verbatim}
\usepackage{hyperref}
\usepackage{booktabs}
\usepackage{colortbl}

\hypersetup{pdfstartview=FitH,pdfpagelayout=SinglePage}

\setlength\paperheight {11in}
\setlength\paperwidth {8.5in}
\setlength{\textwidth}{7in}
\setlength{\textheight}{9.25in}
\setlength{\oddsidemargin}{-.25in}
\setlength{\evensidemargin}{-.25in}
%\setlength{\headsep}{0in}
%\pagenumbering{arabic}

\begin{document}

\conferenceinfo{HotNets 2016} {}
\CopyrightYear{2016}
\crdata{X}
\date{}

%%%%%%%%%%%% THIS IS WHERE WE PUT IN THE TITLE AND AUTHORS %%%%%%%%%%%%

\title{Network Configuration Synthesis}

\author{Anonymous}

\maketitle

%\thispagestyle{empty}

%%%%%%%%%%%%%  ABSTRACT GOES HERE %%%%%%%%%%%%%%
\subsection*{Abstract}

Synthesizing ARCs ...

\section{Introduction}
---Motivating ARC synthesis---


%With growing diversity of user applications, need for security and compliance,
%and the advent of cloud computing, operators of datacenters require a 
%fine-grained control over the multitude of flows to provide functionalities 
%for the diverse and complex demands by applications using the datacenter 
%network like waypoints, isolation, bandwidth guarantees etc.
%While software-defined networks have been a major research area
%in this direction to have more control over the network, for 
%various reasons, the adoption of SDNs have been slow. While 
%legacy networks running distributed protocols like OSPF, 
%BGP have various benefits like scalability and fast recovery to failures, 
%it is very difficult to program these networks to accommodate the 
%diverse requirements of cloud applications. For example, in a network
%which uses the OSPF protocol, the path between two endpoints 
%is the shortest path in the network depending on the link weights 
%of the network, unlike SDNs where operators can add forwarding 
%rules according the path they desire.
Modern networks, including data center, campus, 
and wide area networks, 
must satisfy increasingly complex security, 
availability, and performance objectives to 
meet the demands of multitude of users and applications. 
Many common network management systems provide support
for reachability (or lack thereof for access control),
forms of traffic engineering for optimizing performance
and fairness, service chaining through ``middleboxes'' and
other policies like isolation for performance and privacy reasons.
Thus, a high degree of programmability is required in networks
for enforcing the aforementioned policies.

The software-defined networking (SDN) architecture was introduced
for creating programmable networks, where a centralized controller
manages the forwarding rules in the switches based on the policies.
SDN enables operators to enforce various policies in their networks 
with ease. 
However, the controller is responsible for updating the forwarding
rules during a failure scenario, thus creating a central point of 
failure in the network. Funnily, this was one of the design principles
for legacy control planes running distributed networking protocols 
(e.g., OSPF, BGP, and MPLS). 
Legacy control planes are scalable, robust, maintain a smaller amount 
of state in switches compared to SDNs, 
and split the computational work over all of the switches. 
Importantly, there is no central point of failure and the switches
responded locally to failures and converged to a consistent state. 

The major shortcoming of using distributed control planes is that
they are very difficult to reason about and program to cater to 
complex policies required by operators, which is why SDNs came into
prominence. However, for various reasons, 
SDN deployment in the industry has been slow, and many 
networks still rely on these ``tried-and-tested" distributed 
control planes. 

Para 3: Motivate distributing the control plane
->  scalable, robust, maintain small amount of state compared to SDNs
	and split the computational work over all of the routers. 
->  No Central point of failure, respond locally to failures.
->  For various reasons, SDN deployment has been slow, and many 
	networks still rely on legacy ``tried-and-tested" protocols 

Para 4: Architecture with policy control afford by SDN and advantages of using
distributed data planes. Synthesis of resilient distributed control planes
->  Develop applications enforcing proactive policies
	which construct the network data plane (forwarding rules at switches)
	agnostic to actual network infrastructure, using which we synthesize
	ARCs, thus increasing programmability of 
	legacy networks
->  By incorporating resilience in the synthesis of 
	control planes, failure handling will be 
	performed in a distributed manner, thus removing any central point 
	of failure. This approach eliminates the ad-hoc failure behaviors of
	control planes deployed in practice due to manual error-prone practices. 
->  Eliminates need of verification (batfish, anteater, arc)

Para 5: Abstract Representation of Control Plane: Cost based path selection. 
Actual configurations can be chosen according to needs (Research idea)
->  Existing approaches to constructing distributed control planes 
	from policies tied inherently with the actual network protocol
->  Instead, an ARC effectively decouples the policy component with the 
	actual network implementation, thus, the networking infrastructure could be
	transitioned without affecting the policy controller
->  The ARC paper validates this notion of most routing protocols in use 
	today employing a cost-based path selection algorithm. 
->  With an ARC synthesized from policies, we 
	can produce network configurations based on the
	actual network infrastructure requirements(for now, we can do OSPF trivially). 
	Thus, applications tasked with enforcing network policies can be 
	implemented without any knowledge of underlying network protocols used.
->  Ideal network configurations can be constructed from the ARC by using 
	inferences from healthy network practices in real-life networks~\cite{mpa-imc15}

Synthesis of ARCs is difficult. Sometimes, with global edge weights, 
	certain data planes cannot be enforced by any solution to the weights
	(example). Using route filtering mechanisms can help solve this problem, but 
	filters are related to resilience properties as more filters effectively 
	means lesser links, and thus is an important
	problem to optimize the number of filters to get maximal resilient planes. We describe
	a LP-based approach to synthesizing an ARC from a input data plane. 

Para 6: Features like ``Default Off", incremental changes 
->  A new packet class not matching any policy should ideally be sent to the 
	policy controller for action. Traditional control planes lack this
	``Default Off" behaviour. Changes to switch hardware/firmware(?) to 
	send new packets to the policy controller
->  Enforcing new/modified policies requires a mechanism to modify the 
	existing configuration with minimal changes to enforce the new policies.

--- Network operators can design network management systems
providing different functionalities like service chaining~\cite{simple} or traffic
engineering~\cite{} oblivious of the underlying networking protocol.

--- To achieve this

\section{Related Work}

Fibbing: Lies can cause loops if the centralized controller fails,
our approach is sounder

\section{ARC}
From the ARC paper:
Modeling the collective behavior of multiple routing in- stances in a series of weighted digraphs in the ARC is en- abled by the fact that most routing protocols in use today employ a cost-based path selection algorithm. For exam- ple, OSPF uses Dijkstra’s algorithm to compute min-cost paths from a source to all destinations; RIP computes short- est paths using the Bellman-Ford algorithm.1 BGP asso- ciates cost labels with paths based on numeric metrics: e.g., operator-defined local preference, path length, and multi- exit discriminator (MED) [5, 11]. These have similar proper- ties to link costs used in IGPs, except BGP costs are per-path rather than per-link.


Describe ARC features we will synthesize.

\section{Model}

\section{Route Filtering}

\section{Preliminary Evaluation}

\section{Future Work}

\section{Conclusion}

\section*{Acknowledgments}

\bibliographystyle{abbrv} 
\begin{small}
\bibliography{refs}
\end{small}
\label{last-page}

\end{document}

