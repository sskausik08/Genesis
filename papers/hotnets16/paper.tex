\documentclass{hotnets16}

\usepackage{times}  
\usepackage{epsfig}
\usepackage[TABBOTCAP]{subfigure}
\usepackage{tabularx}
\usepackage{graphicx} 
\usepackage{color}
\usepackage{xspace}
\usepackage{thumbpdf}
\usepackage{listings}
\usepackage{verbatim}
\usepackage{hyperref}
\usepackage{booktabs}
\usepackage{colortbl}

\hypersetup{pdfstartview=FitH,pdfpagelayout=SinglePage}

\setlength\paperheight {11in}
\setlength\paperwidth {8.5in}
\setlength{\textwidth}{7in}
\setlength{\textheight}{9.25in}
\setlength{\oddsidemargin}{-.25in}
\setlength{\evensidemargin}{-.25in}
%\setlength{\headsep}{0in}
%\pagenumbering{arabic}

\begin{document}

\conferenceinfo{HotNets 2016} {}
\CopyrightYear{2016}
\crdata{X}
\date{}

%%%%%%%%%%%% THIS IS WHERE WE PUT IN THE TITLE AND AUTHORS %%%%%%%%%%%%

\title{Network Configuration Synthesis}

\author{Anonymous}

\maketitle

%\thispagestyle{empty}

%%%%%%%%%%%%%  ABSTRACT GOES HERE %%%%%%%%%%%%%%
\subsection*{Abstract}

Synthesizing ARCs ...

\section{Introduction}
---Motivating ARC synthesis---

Para 1: Policy Diversity required in enterprise and datacenters
%With growing diversity of user applications, need for security and compliance,
%and the advent of cloud computing, operators of datacenters require a 
%fine-grained control over the multitude of flows to provide functionalities 
%for the diverse and complex demands by applications using the datacenter 
%network like waypoints, isolation, bandwidth guarantees etc.
%While software-defined networks have been a major research area
%in this direction to have more control over the network, for 
%various reasons, the adoption of SDNs have been slow. While 
%legacy networks running distributed protocols like OSPF, 
%BGP have various benefits like scalability and fast recovery to failures, 
%it is very difficult to program these networks to accommodate the 
%diverse requirements of cloud applications. For example, in a network
%which uses the OSPF protocol, the path between two endpoints 
%is the shortest path in the network depending on the link weights 
%of the network, unlike SDNs where operators can add forwarding 
%rules according the path they desire.

Para 2: SDN Architecture: Centralized Policy Control, Centralized Control Plane

Para 3: Motivate distributing the control plane for fast failover and resilience.

Para 4: Architecture with policy control afford by SDN and advantages of using
distributed data planes

Para 5: Features of the architecture

--- Network operators can design network management systems
providing different functionalities like service chaining~\cite{simple} or traffic
engineering~\cite{} oblivious of the underlying networking protocol.

--- To achieve this

\section{ARC}

Describe ARC features we will synthesize.

\section{Model}

\section{Route Filtering}

\section{Preliminary Evaluation}

\section{Future Work}

\section{Conclusion}

\section*{Acknowledgments}

\bibliographystyle{abbrv} 
\begin{small}
\bibliography{refs}
\end{small}
\label{last-page}

\end{document}

