\section{Motivation}
% Programming a legacy network control plane to satisfy a variety of
% connectivity, security, and performance policies is a complex and error-prone
% task. We have shown that program synthesis is a promising approach to automate
% this process and produce a control plane that is ``correct-by-construction.''
% In particular, we presented an architecture where a network operator provides
% a policy-compliant data plane and a set of hard and soft policies as input,
% and the system automatically provides a set of device configurations that
% leverage the control plane features available on each device to compute
% policy-compliant paths, even in the presence of failures.  Formulating a
% tractable synthesis problem and maximizing the number of satisfied soft
% policies are the key challenges in realizing this vision. We show these
% challenges can be overcome by modeling the control plane using a graph-based 
% abstract representation tied to a traditional shortest path algorithm and
% iteratively transforming the abstract representation based on feedback from
% the satisfiability (SAT) solver and characteristics of the graph. However, we
% have only explored how to address a subset of important policies and leverage a
% few available control plane features on traditional networking hardware. Our
% future work will focus on satisfying a wider range of policies using more
% features, and we will study how to best translate our abstract representation
% into actual device configurations to make our system practical for use with
% real networks.

Network management typically entails managing the flow of packets
through the network based on the policies required by the operator. 
The tasks include which end-hosts can communicate (or be blocked) from
each other, and the path taken in the network for these flows which are
decided from different performance and security considerations like 
service chaining, traffic engineering and isolation. A controller
of a software-defined network can decide the paths for each flow
based on the policies and install forwarding rules accordingly. 
However, legacy control planes use distributed protocols relying on 
cost-based path selection algorithms. Thus, translation of policies to
actual device configurations is a difficult and error-prone task. 

To tackle synthesis 
of distributed control planes, 
we use as input the network data plane 
(set of forwarding paths) \aaron{Where does this data plane come from and
    why is it reasonable to assume an operator can provide one?} which 
enforces the operator-specified policies to
synthesize an abstract representation of the control plane
called ARC. \aaron{Our approach (ARC) should not come up until later in the
    motivation, if at all.}
The advantage of using network data planes as 
input is the ease of developing
different network management applications 
enforcing proactive policies\footnote{
Traditional control planes cannot support reactive policies, however
middleboxes can overcome this limitation.} 
as if operating over a software-defined
network, agnostic of the actual network protocols used in the network. 
Many existing network management systems like Merlin~\cite{merlin} 
and SIMPLE~\cite{simple}\footnote{
Traditional control planes cannot support 
paths with loops for service chaining.} developed for SDNs 
could be seamlessly integrated to the architecture with minimal changes.
\aaron{Do we need the preceding sentence?}
Thus, using the data plane as input, we synthesize a control plane
such that the paths decided by the control plane are the same 
as the data plane input, and these paths satisfy the operator 
policies.

However, the control plane does not guarantee policy compliance 
when failures occur. For
example, if a link along the shortest path 
%(\aaron{refer to some example figure}) 
fails, \aaron{Need to mention earlier that we assume shortest-path-based route
computations, ala OSPF} the next shortest path will become the new path to reach the
destination. The control plane automatically computes the new shortest path
(assuming one exists), thus preserving connectivity. 
However, the new path may
not conform to the same policies as the path in the 
original failure-free data
plane from which the control plane was synthesized: 
e.g., the new path may no longer
traverse a waypoint or have the same bandwidth capacity. Thus,
synthesis must take into account policy-compliance under failures.

Policy-compliance under failures is difficult to achieve due 
to the large number of failure scenarios to consider and can 
be impossible to synthesize a control plane which is 
policy-compliant under all failures. Operators require 
varying degrees of policy-compliance under failures for 
which we propose two classes of policies: {\em hard} and
{\em soft} policies. Operators can specify 
hard policies pertaining
to security are of the utmost importance, 
because they protect the network and
its services from attacks and unauthorized accesses. Some 
examples of hard policies are as follows: 
a particular flow must always traverse through a waypoint 
under any failure scenario, or a certain pair of hosts must 
never be able to communicate (always blocked). 
Soft policies can be considered as objectives which improve 
the control plane, but are not strict requirements, typically
pertaining to performance. The consequences of violating
soft policies are less severe. Operators can provide 
backup paths for flows (generated such that they 
satisfy the data plane policies) as soft policies, 
and the synthesis of the control plane 
tries to satisfy as many soft policies as 
possible. 

\aaron{Discussion of the available control plane constructs should come up
    earlier.}
While we use an abstract representation of the control
plane (ARC) to simplify the synthesis, operators need control 
planes conforming to certain requirements of the 
underlying network infrastructure. For example, a 
network may comprise of switches which do not have custom
features like ACLs or route-filters, thus the ARC must take 
into consideration these requirements. Without losing 
generality, we translate these 
requirements as policies determining the properties of the
ARC, and thus, still use a protocol-independent representation
to represent the control plane. 

Our architecture envisions network operator providing
a policy-compliant data plane and a set of hard and soft policies
pertaining to policy-compliance under failure and network 
infrastructure requirements as input to synthesize resilient 
control planes. Towards this vision, we present  
approaches to synthesize different types of the ARC and the 
challenges involved in making synthesis efficient and
extending these approaches to support a wider range of 
policies and functionalities. 
