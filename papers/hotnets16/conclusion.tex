\section{Conclusion}
Programming a legacy network control plane to satisfy a variety of
connectivity, security, and performance policies is a complex and error-prone
task. We have shown that program synthesis is a promising approach to automate
this process and produce a control plane that is ``correct-by-construction.''
In particular, we presented an architecture where a network operator provides
a policy-compliant data plane and a set of hard and soft policies as input,
and the system automatically provides a set of device configurations that
leverage the control plane features available on each device to compute
policy-compliant paths, even in the presence of failures.  Formulating a
tractable synthesis problem and maximizing the number of satisfied soft
policies are the key challenges in realizing this vision. We show these
challenges can be overcome by modeling the control plane using a graph-based 
abstract representation tied to a traditional shortest path algorithm and
iteratively transforming the abstract representation based on feedback from
the satisfiability (SAT) solver and characteristics of the graph. However, we
have only explored how to address a subset of important policies and leverage a
few available control plane features on traditional networking hardware. Our
future work will focus on satisfying a wider range of policies using more
features, and we will study how to best translate our abstract representation
into actual device configurations to make our system practical for use with
real networks.
