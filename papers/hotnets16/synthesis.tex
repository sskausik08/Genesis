\section{Synthesis}
We first describe the synthesis of {\em pure} ARCs, i.e., 
ARC with weighted edges and no other mechanisms like route-filters.
A pure ARC enforces the policies, and since no filters are used, the
control plane is {\em maximal resilient} (all links can be used 
to reach a destination). 
% Provide insight about route-filters before.

The inputs to the synthesis algorithm are:

(1) Network Data Plane:
Since most legacy routing protocols are destination-based,
from an input data plane, we extract the forwarding
DAG (directed acyclic graph) for each destination
as the sink. The DAG will be a directed tree with the
destination as root if the multi-path
functionality is not used. We define $\Omega$ for the 
set of destinations, and $\Delta$ for  
the set of destination DAGs. 
% Write something about changes to the network manager

(2) Network Topology: 
We define the physical network topology $T = (S, L)$
as a directed graph where $S$: set of switches, and 
$L$: set of directional links. 

The output of the synthesis algorithm is a 
pure ARC with positive weights on the directional links,
such that the set of forwarding DAGs $\Delta$ are the 
shortest paths in the ARC. We define $E(sw_1, sw_2)$ as
the edge weight variable for link $sw_1 \rightarrow sw_2$. 

\subsection{Linear Equations}
We construct an directed overlay graph
by combining all the individual destination DAGs in $\Delta$.
We can disable all switches and links not in the overlay. 
These links can have arbitary large weights as they do not
feature in the shortest path for any destination. 
Given $\Omega$ and $\Delta$, we generate a set of linear equations
to find the requisite edge weights. For a path connecting two switches 
$s \rightarrow^+ t$ in a DAG, 
to enforce that the path will be the shortest, we need equations
which ensure the sum of edge weights of the path is strictly smaller than
the weight of all other paths from $s$ to $t$. However, this can incur
an exponential blowup in number of equations. Instead, we use distances 
to reduce the number of equations to a polynomial order. 

\minisection{Distance Equations}
We define $D(s,t)$ to be the distance variable from $s$ to $t$
which denotes the absolute shortest distance from $s$ to $t$. 
Trivially, $D(s,s) = 0$. For a path $s \rightarrow^+ t$ ($len \geq 1$),
the shortest distance from $s$ to $t$ is the smallest of the distances
of the paths traversed through the neighbours of $s$ to $t$. Therefore, the
following set of equations enforce the semantics of distances; 
here $s \rightarrow sw$ denotes that $sw$ is a neighbour
of $s$, and $sw \rightarrow^* t$ denotes that the path from $sw$ to $t$ contains
zero or more edges.
\begin{multline} \label{eq:dist}
\forall s, t, sw. (s \rightarrow sw \rightarrow^* t).\\
D(s, t) \leq E(s, sw) + D(sw, t)
\end{multline}
These equations ensure that $D(s,t)$ is not greater than 
the actual shortest distance from $s$ to $t$.

\minisection{Destination DAG Equations}
For each destination DAG, we add equations to ensure the 
edge weights are set such that a shortest-path forwarding in the ARC will 
follow the same paths as specified in the DAG. We use the following
property: If a path is the shortest 
path, then every subpath of the path has to be the shortest path.

Consider a DAG $\xi_d$ for destination $d$. We define two neighbour
functions: $N(s)$ denotes the neighbours of switch $s$ 
in the directed overlay graph, and $N(s, \xi_d)$ denotes
the neighbours of switch $s$ in the destination DAG. 
We add the following inductive equations to ensure that
 the shortest distance
for each subpath of $\xi_d$ is the sum of weights of the edges in the path (as
each subpath is the shortest path).
\begin{multline} \label{eq:shortest}
	\forall d \in \Omega. \forall s, t \in \xi_d. (s \rightarrow^+ t).
	\forall n \in N(s, \xi_d). \\
	 D(s,t) = E(s, n) + D(n, t)
\end{multline}
These equations ensure that the weights of paths in the DAG $\xi_i$ are equal to
sum of edges, and thus would be the shortest paths. 
However, we also need to ensure that other 
paths are not shorter or equal to these paths. 
Strict inequality is required because routing protocols
can load-balance or choose one of the equal weights paths at
random, both of which violate the DAG requirements.
Thus, we add the following equations to ensure this property:
\begin{multline} \label{eq:uniq}
		\forall d \in \Omega. \forall s, t \in \xi_d. (s \rightarrow^+ t).\\ 
		\forall n'. (n' \in N(s) \wedge n' \not\in N(s, \xi_d)). \\
		D(s,t) < E(s, n') + D(n', t)
\end{multline}

If the path $n' \rightarrow^* t$ 
is not in any DAG completely, then
$D(n',t)$ can be smaller than the actual shortest distance by the
semantics of \Cref{eq:dist} (as
$D(n',t)$ is not equal to any quantity by \Cref{eq:shortest}).
However, since $D(n',t)$ is on the RHS of the equations in \Cref{eq:uniq},
the equations will ensure that the path $s\rightarrow n \rightarrow^* t$
has strictly greater weight than the path of the DAG.

%TODO: Add some figures