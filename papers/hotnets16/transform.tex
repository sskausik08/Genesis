\section{\ARC Transformation}
Although the \ARC synthesized using the aforementioned approach represents a
control plane that will compute policy-compliant paths in the absence of
failures, it does not guarantee policy compliance is preserved when failures
occur. For example, if a link along the shortest path (\aaron{refer to some
example figure}) fails, the next shortest path will become the new path to
read the destination. The control plane will automatically compute the new
shortest path (assuming one exists), thus preserving connectivity. However,
the new path may not conform to the same policies as the path in the original
failure-free data plane from which the \ARC was synthesized: e.g., the new
path may no longer traverse a waypoint or have the same bandwidth capacity.

Given the high frequency of failures in data center and enterprise campus
networks, it is desirable to synthesize a control plane that computes
policy-compliant paths under a reasonable number (denoted by $t$) of link
failures.\footnote{The precise number of link failures that can/should be
tolerated depends on the redundancy available in the physical topology and the
importance of the network.}  \aaron{TODO: Talk about synthesizing from
multiple data planes and the difficulty in providing multiple data planes as
well as having pre-determined paths being too constraining. Suggest repair as
an alternative.}
