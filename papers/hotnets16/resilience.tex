\subsection{From ARCs to resilient ARCs}

%\subsubsection{Backup paths}
%\kausik{Encoding backup paths}
%
%\subsubsection{Resilience Policies}
%\kausik{Talk here about ARC repair (some other term for that).}

The \ARC synthesized using the aforementioned approach represents a control
plane that will compute policy-compliant paths in the absence of failures.
However, it does not guarantee policy compliance when failures occur. For
example, if a link along the shortest path fails, the next shortest path (if
one exists) will become the new path to reach the destination. However, the
new path may not conform to the same policies as the path in the original
failure-free data plane from which the \ARC was synthesized: e.g., the new
path may no longer traverse a waypoint or have the same bandwidth capacity.

Given the high frequency of failures in data center~\cite{datacenterfailures},
campus, and wide-area~\cite{turner10:sigcomm} networks, it is desirable to
generate a control plane that computes {\em policy-compliant backup paths}.
In particular, we want a network to be {\em t-resilient}~\cite{plinko}: i.e.,
the control plane will compute a policy-compliant path when there are $t$ or
fewer link failures in the network.\footnote{The value of $t$ depends on
operator objectives and the level of redundancy available in the physical
topology.} 

%In some cases, network operators may only require a subset of
%policies to be enforced during failures---e.g., enforce reachability and
%waypoint policies but 

%While it is desirable for the backup paths to satisfy all of the same policies
%as the primary path---e.g., reachability, waypoint traversals, bandwidth
%capacity, etc.---not all policies are equally important.  Policies pertaining
%to security (e.g., reachability and some waypoint traversals, such as
%firewalls) are of the utmost importance, because they protect the network and
%its services from attacks and unauthorized accesses.  In contrast, policies
%pertaining to performance (e.g., bandwidth capacity and some waypoint
%traversals, such as caches) are desirable, but the consequences of violating
%these policies are often less severe, partially because such policies are
%typically ``best effort'' even in the non-failure case.\footnote{In some
%networks, performance-related policies can have a major impact on a company's
%bottom-line: e.g., degradation in video streaming performance due to
%insufficient bandwidth may cause customers to cancel their
%subscription~\cite{TBD}. However, a centrally controlled, programmable control
%plane~\cite{swan, b4} is better suited to these scenarios, so we do not
%consider them in this work.} Consequently, we focus on ensuring the control
%plane we generate computes backup paths that satisfy the former class of
%policies. \aaron{Are we doing ourselves a disservice by eschewing
%performance-related policies for backup paths, especially for a HotNets
%paper?} 

We could synthesize a t-resilient control plane by generating a data plane
(\secref{sec:policies_to_dp}) for each possible combination of up to $t$
failed links and updating our constraints (\secref{sec:dps_to_arc}) to account
for all backup paths. However, for a network with $\ell$ links, this requires
generating $\sum_{i=0}^{t} \frac{\ell!}{t! * (\ell-t)!}$ data planes.
Moreover, some of the data planes we generate may not be compatible---i.e., it
may not be possible to assign edge weights and route filters such that the
paths computed under a given failure scenario match the provided backup paths.
Thus, we will need many iterations of data plane generation to find a set of
data planes from which we can synthesize a t-resilient control plane.

%\minisection{Approaches}
%One approach to generating a t-resilient control plane is to synthesize an
%\ARC from a {\em set} of data planes. The set must include a policy-compliant
%data plane for each possible combination of up to $t$ failed links. For a
%network with $n$ links, there are $\sum_{i=0}^{t} \frac{n!}{t! * (n-t)!}$ such
%data planes. Expecting a network operator to provide these data planes,
%constructed using a manual or automated process, goes against the purpose of
%automatically generating a policy-compliant control plane: reducing the burden
%on network operators and the potential for human error. Moreover, if the data
%planes needed for all possible failure scenarios are already available, then
%we can trivially install the corresponding rules on network devices, even if
%the devices are not directly programmable~\cite{TODO}. This avoids the need
%for a control plane that (re-)computes the data planes at runtime.
%Additionally, it may not be possible to produce the collective set of backup
%paths using a single assignment of edge weights. For example, consider the two
%backup paths shown in \figref{fig:TODO}; there is no possible assignment of
%edge weights such that the shortest path computed over the resulting graph
%under each of the failure scenarios equals the depicted backup paths.
%\kausik{I think the above para kind of downplays our
%motivation to synthesize control planes. I think we should write
% it like: given backup paths, we try to incorporate those in the ARC
% during synthesis. But providing backup paths can be difficult, so we
% provide functionality to have high level ARC policies like traverses through
% a waypoint. What do you think?}

\begin{table}
\footnotesize
\setlength{\tabcolsep}{0.2em}
\begin{tabular}{p{0.43\columnwidth}p{0.52\columnwidth}}
{\bf Policy} & {\bf Graph characteristic} \\
\hline
{\em P1}: Flow blocked & After removing the edges with filters for the flow, there is no
path from $\srcSwitch$ to $\dstSwitch$ \\
\hline
{\em P2}: Flow traverses waypoint & After removing the vertices corresponding to
waypoints, there is no path from $\srcSwitch$ to $\dstSwitch$ \\
\hline
{\em P3}: Destination reachable\newline under $\leq k$ link failures & Max-flow from
$\srcSwitch$ to $\dstSwitch$ is $\geq$~$k$~+~1 \\
\hline
\end{tabular}
\label{t:policy_characteristics}
\caption{Graph characteristics an \ARC must satisfy to ensure backup paths are
policy compliant; policies and characteristics are inspired by Gember-Jacobson
et. al \protect\cite{arc}}
\end{table}


\minisection{Transformations}
Instead, we {\em transform} the 0-resilient \ARC synthesized from a single
data plane (\secref{sec:dps_to_arc}) to a t-resilient \ARC.
%such that all paths computed over the transformed \ARC under up to $t$ link
%failures are policy-compliant. This avoids the need for an operator to provide
%a data plane for every failure scenario. Instead, an operator only needs to
%provide: (1) a single data plane corresponding to a failure-free state, and
%(2) the policies that should apply to all backup paths computed for a flow. 
The transformations are based on simple graph characteristics that an \ARC
must satisfy to be policy-compliant.  \tabref{tab:policy_characteristics}
lists several policies and the requisite graph characteristics of an
$\ell$-resilient control plane---i.e., a control plane that complies with this
policy under all possible combinations of link failures. 
%\aaron{The graph characteristics in~\cite{arc} are actually if we want the policies to be satisfied under an
%arbitrary (up to $n$) number of link failures. This diverges from the
%discussion of t-resilience above. Does t-resilience actually correspond to the
%``always reachable with $<t$ failures'' policy?} 

For example, to satisfy the policy that a flow from switch $\srcSwitch$ to
switch $\dstSwitch$ is blocked on all backup paths ({\em P1}), it must be the
case that all paths between $\srcSwitch$ and $\dstSwitch$ in the \ARC contain
at least one edge on which the flow is filtered. If we remove all edges with
such a filter from the graph, and there remains a path from $\srcSwitch$ to
$\dstSwitch$, then there is some backup path for which the policy does not
hold; (one of) the remaining path(s) is the backup path for a scenario in
which all of the links with filters have failed. Compliance with a ``flow
traverses waypoint'' policy ({\em P2}) requires the flow's graph to have a
similar characteristic.

If the aforementioned graph characteristics are not satisfied, then we need to
add filters or waypoints to the graph until the characteristic is satisfied.
We can do this using a simple iterative process: 
\begin{enumerate}
\vspace{-1ex}
\setlength{\topsep}{0pt}
\setlength{\itemsep}{0em}
\setlength{\parskip}{0pt}
\setlength{\parsep}{0pt}
\item Remove all edges with a filter (or vertices corresponding to waypoints) 
\item Find a path from $\srcSwitch$ to $\dstSwitch$; if none exists,
    then the control plane is $\ell$-resilient w.r.t. the current policy
\item Otherwise, add a filter (waypoint) to some edge along the path, and repeat
\vspace{-1ex}
\end{enumerate}
Note that adding filters does not impact the selection of paths for a flow, as
is the case in \secref{sec:dps_to_arcs}, because we are adding filters to
ensure there is {\em no} path for the flow.

Although the above steps will produce an $\ell$-resilient control plane, the
resulting control plane may contain more filters or waypoints than necessary.
We can find the minimal number of filters or waypoints to add by computing a
min-cut after the first step and adding a filter or waypoint to each edge in the cut. 

Other policies require more complex transformations. For example, to ensure
two endpoints can communicate even in the presence of up to $k$ link failures
({\em P3}), where $k \leq t$, the \ARC must contain at least $k+1$
edge-disjoint, filter-free paths; this is equivalent to a unit-weight,
filter-edges-removed version of the \ARC having a max-flow of at least
$k+1$~\cite{arc}. By removing filters, we may be able to increase the number
of available edge-disjoint, filter-free paths. However, this may counteract
the addition of filters that occurred during the \ARC synthesis phase
(\secref{sec:dps_to_arc}). Fortunately, the data plane from which the
0-resilient \ARC was synthesized is just one possible policy-compliant data
plane; other paths are acceptable as long as they conform to the provided
policies. We can ensure policy-compliance is preserved by applying the
transformations for other policies ({\em P1} and {\em P2}) second.

The graph characteristics that need to be satisfied when $t < \ell$ are more
complex, because only some of the possible paths between $\srcSwitch$ and
$\dstSwitch$ need to conform to the policies. Identifying these
characteristics and an efficient way to enforce them is part of our future
work.
