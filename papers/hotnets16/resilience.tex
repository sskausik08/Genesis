\subsection{Resiliency}

%\subsubsection{Backup paths}
%\kausik{Encoding backup paths}
%
%\subsubsection{Resilience Policies}
%\kausik{Talk here about ARC repair (some other term for that).}

Although the \ARC synthesized using the aforementioned approach represents a
control plane that will compute policy-compliant paths in the absence of
failures, it does not guarantee policy compliance when failures occur. For
example, if a link along the shortest path (\aaron{refer to some example
figure}) fails, the next shortest path will become the new path to reach the
destination. The control plane automatically computes the new shortest path
(assuming one exists), thus preserving connectivity. However, the new path may
not conform to the same policies as the path in the original failure-free data
plane from which the \ARC was synthesized: e.g., the new path may no longer
traverse a waypoint or have the same bandwidth capacity.

\minisection{Backup paths}
Given the high frequency of failures in data center and enterprise campus
networks~\cite{TODO}, it is desirable to synthesize a control plane that
computes {\em policy-compliant backup paths}. In particular, we want a network
to be {\em t-resilient}~\cite{TODO}: i.e., the control plane will compute a
policy-compliant backup path when there are $t$ or fewer link failures in the
network.\footnote{The value of $t$ depends on operator objectives and the
level of redundancy available in the physical topology.}

While it is desirable for the backup paths to satisfy all of the same policies
as the primary path---e.g., reachability, waypoint traversals, bandwidth
capacity, etc.---not all policies are equally important.  Policies pertaining
to security (e.g., reachability and some waypoint traversals, such as
firewalls) are of the utmost importance, because they protect the network and
its services from attacks and unauthorized accesses.  In contrast, policies
pertaining to performance (e.g., bandwidth capacity and some waypoint
traversals, such as caches) are desirable, but the consequences of violating
these policies are often less severe, partially because such policies are
typically ``best effort'' even in the non-failure case.\footnote{In some
networks, performance-related policies can have a major impact on a company's
bottom-line: e.g., degradation in video streaming performance due to
insufficient bandwidth may cause customers to cancel their
subscription~\cite{TBD}. However, a centrally controlled, programmable control
plane~\cite{swan, b4} is better suited to these scenarios, so we do not
consider them in this work.} Consequently, we focus on ensuring the control
plane we generate computes backup paths that satisfy the former class of
policies. \aaron{Are we doing ourselves a disservice by eschewing
performance-related policies for backup paths, especially for a HotNets
paper?}

\minisection{Approaches}
One approach to generating a t-resilient control plane is to synthesize an
\ARC from a {\em set} of data planes. The set must include a policy-compliant
data plane for each possible combination of up to $t$ failed links. For a
network with $n$ links, there are $\sum_{i=0}^{t} \frac{n!}{t! * (n-t)!}$ such
data planes. Expecting a network operator to provide these data planes,
constructed using a manual or automated process, goes against the purpose of
automatically generating a policy-compliant control plane: reducing the burden
on network operators and the potential for human error. Moreover, if the data
planes needed for all possible failure scenarios are already available, then
we can trivially install the corresponding rules on network devices, even if
the devices are not directly programmable~\cite{TODO}. This avoids the need
for a control plane that (re-)computes the data planes at runtime.
Additionally, it may not be possible to produce the collective set of backup
paths using a single assignment of edge weights. For example, consider the two
backup paths shown in \figref{fig:TODO}; there is no possible assignment of
edge weights such that the shortest path computed over the resulting graph
under each of the failure scenarios equals the depicted backup paths.

Instead, we 
%synthesize an \ARC that computes paths equivalent to those present
%in an operator-provided data plane corresponding to a failure-free state, as
%described above (\secref{sec:}), and then we
{\em transform} the \ARC synthesized from a single data plane
(\secref{sec:}) such that all paths computed over the transformed \ARC under
up to $t$ link failures are policy-compliant. This avoids the need for an
operator to provide a data plane for every failure scenario. Instead, an
operator only needs to provide: (1) a single data plane corresponding to a
failure-free state, and (2) the policies that should apply to all backup paths
computed for a flow. 

\begin{table}
\footnotesize
\setlength{\tabcolsep}{0.2em}
\begin{tabular}{p{0.43\columnwidth}p{0.52\columnwidth}}
{\bf Policy} & {\bf Graph characteristic} \\
\hline
{\em P1}: Flow blocked & After removing the edges with filters for the flow, there is no
path from $\srcSwitch$ to $\dstSwitch$ \\
\hline
{\em P2}: Flow traverses waypoint & After removing the vertices corresponding to
waypoints, there is no path from $\srcSwitch$ to $\dstSwitch$ \\
\hline
{\em P3}: Destination reachable\newline under $\leq k$ link failures & Max-flow from
$\srcSwitch$ to $\dstSwitch$ is $\geq$~$k$~+~1 \\
\hline
\end{tabular}
\label{t:policy_characteristics}
\caption{Graph characteristics an \ARC must satisfy to ensure backup paths are
policy compliant; policies and characteristics are inspired by Gember-Jacobson
et. al \protect\cite{arc}}
\end{table}


\minisection{Transformations}
The transformations are based on simple graph characteristics that an \ARC
must satisfy to conform to specific policies
(\tabref{tab:policy_characteristics}).  \aaron{The graph characteristics
in~\cite{arc} are actually if we want the policies to be satisfied under an
arbitrary (up to $n$) number of link failures. This diverges from the
discussion of t-resilience above. Does t-resilience actually correspond to the
``always reachable with $<t$ failures'' policy?}
For example, to satisfy the policy that a flow from switch $\srcSwitch$ to
switch $\dstSwitch$ is blocked on all backup paths, it must be the case that
all paths between $\srcSwitch$ and $\dstSwitch$ in the flow's graph \aaron{do
we have a name for this?} contain at least one edge on which the flow is
filtered. If we remove all edges with such a filter from the graph, and there
remains a path from $\srcSwitch$ to $\dstSwitch$, then there is some backup
path for which the policy does not hold; (one of) the remaining path(s) is the
backup path that will be computed when all of the links with filters have
failed. Compliance with a ``flow traverses waypoint'' policy requires the
flow's graph to have a similar characteristic.

If the aforementioned graph characteristics are not satisfied, then we need to
add filters or waypoints to the graph until the characteristic is satisfied.
We can do this using a simple iterative process: 
\begin{enumerate}
\item Remove all edges with a filter (or vertices corresponding to waypoints) 
\item Find a path from $\srcSwitch$ to $\dstSwitch$; if no such path exists,
    then the control plane is policy-compliant w.r.t. the current policy
\item Otherwise, add a filter (waypoint) to an arbitrary edge along the path
    and repeat
\end{enumerate}

Although the above steps will produce a policy-compliant control plane, the
resulting control plane may contain more filters or waypoints than necessary.
\aaron{TODO: talk about finding an optimal transformation by computing the
    min-cut}

\aaron{TODO: talk about making the graph satisfy the max-flow characteristic
    to comply with a destination reachable policy}

