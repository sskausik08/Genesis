\subsection{From ARCs to Resilient ARCs}

The \ARC synthesized using the aforementioned approach represents a control
plane that will compute policy-compliant paths in the absence of failures.
However, it does not guarantee policy compliance when failures occur. For
example, if a link along the shortest path fails, the next shortest path (if
one exists) will become the new path to reach the destination. However, the
new path may not conform to the same policies as the path in the original
failure-free data plane from which the \ARC was synthesized---e.g., the new
path may no longer traverse a waypoint or have the same bandwidth capacity.

Given the high frequency of failures in data center~\cite{datacenterfailures},
campus, and wide-area~\cite{turner10:sigcomm} networks, it is desirable to
generate a control plane that computes {\em policy-compliant backup paths}.
In particular, we want a network to be {\em k-resilient}~\cite{plinko}---i.e.,
the control plane will compute a policy-compliant path when there are $k$ or
fewer link failures in the network.\footnote{The value of $k$ depends on
operator objectives and the level of redundancy available in the physical
topology.} 


\begin{table}
\footnotesize
\setlength{\tabcolsep}{0.2em}
\begin{tabular}{p{0.43\columnwidth}p{0.52\columnwidth}}
{\bf Policy} & {\bf Graph characteristic} \\
\hline
{\em P1}: Flow blocked & After removing the edges with filters for the flow, there is no
path from $\srcSwitch$ to $\dstSwitch$ \\
\hline
{\em P2}: Flow traverses waypoint & After removing the vertices corresponding to
waypoints, there is no path from $\srcSwitch$ to $\dstSwitch$ \\
\hline
{\em P3}: Destination reachable\newline under $\leq k$ link failures & Max-flow from
$\srcSwitch$ to $\dstSwitch$ is $\geq$~$k$~+~1 \\
\hline
\end{tabular}
\label{t:policy_characteristics}
\caption{Graph characteristics an \ARC must satisfy to ensure backup paths are
policy compliant; policies and characteristics are inspired by Gember-Jacobson
et. al \protect\cite{arc}}
\end{table}


Our propose  approach is to attempt to {\em transform} the \ARC synthesized from a single
data plane (\secref{sec:dps_to_arc}) into a k-resilient \ARC.
The transformations are based on simple graph characteristics that an \ARC
must satisfy to be policy-compliant.  \tabref{policy_characteristics}
lists several policies and the requisite graph characteristics of a
$k$-resilient control plane. 

For example, to satisfy the policy that a flow from switch $\srcSwitch$ to
switch $\dstSwitch$ is blocked on all backup paths ({\em P1}), it must be the
case that all paths between $\srcSwitch$ and $\dstSwitch$ in the \ARC contain
at least one edge on which the flow is filtered. If we remove all edges with
such a filter from the graph, and there remains a path from $\srcSwitch$ to
$\dstSwitch$, then there is some backup path for which the policy does not
hold; (one of) the remaining path(s) is the backup path for a scenario in
which all of the links with filters have failed. Compliance with a ``flow
traverses waypoint'' policy ({\em P2}) requires the flow's graph to have a
similar characteristic.

If the aforementioned graph characteristics are not satisfied, then we need to
add filters or waypoints to the graph until the characteristic is satisfied.
We can do this using a simple iterative process: 
\begin{enumerate}
\vspace{-1ex}
\setlength{\topsep}{0pt}
\setlength{\itemsep}{0em}
\setlength{\parskip}{0pt}
\setlength{\parsep}{0pt}
\item Remove all edges with a filter (or vertices corresponding to waypoints) 
\item Find a path from $\srcSwitch$ to $\dstSwitch$; if none exists,
    then the control plane is k-resilient w.r.t. the current policy
\item Otherwise, add a filter (waypoint) to some edge along the path, and repeat
\vspace{-1ex}
\end{enumerate}
Note that adding filters does not impact the selection of paths for a flow, as
is the case in \secref{sec:dps_to_arcs}, because we are adding filters to
ensure there is {\em no} path for the flow.

Although the above steps will produce an k-resilient control plane, the
resulting control plane may contain more filters or waypoints than necessary.
We can find the minimal number of filters or waypoints to add by computing a
min-cut after the first step and adding a filter or waypoint to each edge in the cut. 

Other policies require more complex transformations. For example, to ensure
two endpoints can communicate even in the presence of up to $k$ link failures
({\em P3}), where $k \leq t$, the \ARC must contain at least $k+1$
edge-disjoint, filter-free paths; this is equivalent to a unit-weight,
filter-edges-removed version of the \ARC having a max-flow of at least
$k+1$~\cite{arc}. By removing filters, we may be able to increase the number
of available edge-disjoint, filter-free paths. However, this may counteract
the addition of filters that occurred during the \ARC synthesis phase
(\secref{sec:dps_to_arc}). Fortunately, the data plane from which the
0-resilient \ARC was synthesized is just one possible policy-compliant data
plane; other paths are acceptable as long as they conform to the provided
policies. We can ensure policy-compliance is preserved by applying the
transformations for other policies ({\em P1} and {\em P2}) second.

The graph characteristics that need to be satisfied when $t < \ell$ are more
complex, because only some of the possible paths between $\srcSwitch$ and
$\dstSwitch$ need to conform to the policies. Identifying these
characteristics and an efficient way to enforce them is part of our future
work.
