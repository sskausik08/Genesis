\section{Introduction}
\label{sec:intro}

%(1) Programming networks is hard, because: (1) there are many types of
%policies, (2) policies are inherently centralized/global, and (3) policies
%should also hold under failures. 

Programming networks to correctly forward flows according to user- and
application-induced policies is notoriously difficult and
error-prone~\cite{TODO-see-Propane}. At least three common characteristics of
network policies are to blame: (1) there are several types of policies a
network may need to satisfy, including reachability, isolation, service
chaining, resilience, and traffic engineering; (2) many of these policies must
also hold under failures; and (3) most policies are global---i.e., they
concern end-to-end paths as opposed to individual devices and links.

%(2) SDN tries to make realization centralized like policies, but it is hard
%to ensure link failures don't impact the controller's ability to update the
%network to accommodate the failures; a distributed SDN controller doesn't fix
%this problem. Legacy control planes are preferable from a failure
%tolerance-perspective, but determining the appropriate distributed
%realization of policies is hard.

The global nature of network policies is one motivation for software-defined
networking (SDN). SDN allows paths to be centrally computed over a global view
of the network. However, it is difficult to ensure forwarding paths are
correctly computed and installed in the presence of failures, even if the SDN
controller is distributed~\cite{TODO-see-Propane-or-Akella-HotNets}.
``Traditional'' control planes that rely on distributed routing protocols to
compute paths are preferable from a failure-tolerance perspective. But
determining the appropriate distributed realization of policies is hard.
   
%(3) Our goal is to automate the process of producing this distributed
%realization. We want to handle a wide range of policies and leverage the many
%protocols/mechanisms available in legacy control planes for realizing these
%policies.


%(4) We can view the control plane as a program whose input is set of
%endpoints and the state of the links in the network and whose output is a set
%of paths.  Our goal is to synthesize this program.

Our goal is to {\em automate the process of producing the distributed
realization of policies in a traditional control plane}. We seek to handle
a wide range of policies and leverage the many protocols (e.g., OSFP, BGP) and
mechanisms (e.g., route filters, route redistribution) available in
traditional control planes. Although each network switch runs its own routing
processes, we can view the collection of processes (i.e., the control plane)
as a single program whose input is a set of endpoints plus the state
of network links; the output of this program is a set of paths that conform to
the policies encoded in the program (\figref{fig:cp_program}). Our goal is
thus to synthesize this program out of the constructs \aaron{word choice?}
(e.g., OSPF performs shortest-path routing) available in network switches.

%(5) PL community has adopted two primary approaches to synthesis: synthesis
%from specifications and synthesis from examples. In our case, the
%specifications are the policies we want to realize, and examples are paths
%that satisfy these policies. We show the former does not scale, and the
%latter does not work well because we need lots of examples to get the control
%plane right and generating example paths is hard.

\loris{The problem of synthesising programs that meet a given specification
is computationally hard. In our setting, 
the specifications are the policies we want to realize and the programs 
are the control plane.
As we show in \secref{sec:TODO}, synthesizing a control plane
directly from policies is complicated by the fact that, to infer the concrete
set of paths induced by a given control plane, one has to incorporate
into the synthesis procedure complex concepts such as shortest path algorithms.
}

%(6) We propose a two-phased approach that synthesizes a partially-correct
%control plane from examples, and then applies a series of transformations to
%produce a fully policy-compliant control plane.

Instead, we propose 
a three-phased approach (\secref{sec:synthesis}). First, we
synthesize 
\loris{a concrete set of paths---i.e., a data plane---meeting 
all the functional policies, such as reachability, waypoint, and isolation.
%a partially-correct control plane from a set of example paths, such
%as those contained within a sample data plane. 
Second, we generate a simple control plane that,
under no failures,
induces the set of synthesized paths.
Finally, we 
apply a series of
transformations
to obtain a control plane that  computes policy-compliant paths under a bounded set of failures (e.g.,
all single- and double-link failures).
If the procedure fails, we produce a different set of paths using the approach in the first phase 
and try again.}

%(7) A major challenge is abstracting away the low-level details of control
%plane configurations. We do this using an abstract representation.

Making this approach scalable and practical requires overcoming several
challenges. First, control plane configurations are low-level (e.g., on which
interfaces should a routing protocol operate and what costs should be assigned
to the incident links) while policies are high-level. Second, we must ensure
the transformations we apply to the synthesized control plane cause the
program to no longer conform to the example paths. \aaron{What else?} We
discuss these challenges in more depth in \secref{sec:synthesis} and present
preliminary solutions to address them.

\aaron{Needs a closing, perhaps an outline, a summary of contributions, some
results, and/or something visionary?}
