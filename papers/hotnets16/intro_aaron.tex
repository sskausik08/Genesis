\section{Introduction}
\label{sec:intro}

%(1) Programming networks is hard, because: (1) there are many types of
%policies, (2) policies are inherently centralized/global, and (3) policies
%should also hold under failures. 

Programming networks to correctly forward flows according to user- and
application-induced policies is notoriously difficult and
error-prone~\cite{troubleshooting, bgpmisconfig}. At least three common characteristics of
network policies are to blame: (1) a network may need to satisfy several
types of policies, including reachability, isolation, service
chaining, resilience, and traffic engineering; (2) many of these policies must
also hold under failures; and (3) most policies are global---i.e., they
concern end-to-end paths, not individual devices and links.

%(2) SDN tries to make realization centralized like policies, but it is hard
%to ensure link failures don't impact the controller's ability to update the
%network to accommodate the failures; a distributed SDN controller doesn't fix
%this problem. Legacy control planes are preferable from a failure
%tolerance-perspective, but determining the appropriate distributed
%realization of policies is hard.

The global nature of network policies is one motivation for software-defined
networking (SDN). SDN allows paths to be centrally computed over a global view
of the network. However, it is difficult to ensure forwarding paths are
correctly computed and installed in the presence of failures, even if the SDN
controller is distributed~\cite{hasdn}.
Traditional control planes that rely on distributed protocols to
compute paths are preferable from a failure-tolerance perspective, but
determining the appropriate distributed realization of policies is hard.
   
%(3) Our goal is to automate the process of producing this distributed
%realization. We want to handle a wide range of policies and leverage the many
%protocols/mechanisms available in legacy control planes for realizing these
%policies.


%(4) We can view the control plane as a program whose input is set of
%endpoints and the state of the links in the network and whose output is a set
%of paths.  Our goal is to synthesize this program.

Our goal is to {\em automate the process of creating a correct distributed
realization of policies in a traditional control plane}. We seek to handle
a wide range of policies (e.g., reachability, service chaining, traffic
engineering) and leverage the many protocols (e.g., OSFP, BGP) and mechanisms
(e.g., route filters, route redistribution) available in traditional control
planes. We also want to ensure the generated control plane computes
policy-compliant paths under a bounded number of failures. This contrasts with
prior works~\cite{netegg, propage} which generate SDN- or BGP-based
control planes for a limited range of policies (e.g., inter-domain
reachability).

Although a distributed control plane is composed of a collection of routing
processes running on individual switches, we can view the control plane as a
single program. The input is a set of endpoints plus the state of network
links; the output of this program is a set of paths that conform to the
policies encoded in the program. This view inspires us to explore {\em program
synthesis} as a mechanism for generating a policy-compliance control plane.

%Our goal is thus to synthesize this program out of the constructs \aaron{word
%choice?} (e.g., OSPF performs shortest-path routing) available in network
%switches.

%(5) PL community has adopted two primary approaches to synthesis: synthesis
%from specifications and synthesis from examples. In our case, the
%specifications are the policies we want to realize, and examples are paths
%that satisfy these policies. We show the former does not scale, and the
%latter does not work well because we need lots of examples to get the control
%plane right and generating example paths is hard.

The problem of synthesising programs that meet a given specification
is computationally hard. 
%In our setting, the specifications are the policies we want to realize and the programs 
%are the control plane.
As we show in \secref{sec:vision}, synthesizing a control plane
directly from policies is complicated by the fact that, to infer the concrete
set of paths induced by a given control plane, one has to incorporate
into the synthesis procedure complex concepts such as shortest path algorithms
using theories like propositional logic (SAT) and linear rational
arithmetic (LRA). Even with recent advances in Satisfiability Modulo Theories
(SMT) solvers (e.g., Z3~\cite{z3}), this approach does not scale to
even moderately-sized networks or sets of policies.

%(6) We propose a two-phased approach that synthesizes a partially-correct
%control plane from examples, and then applies a series of transformations to
%produce a fully policy-compliant control plane.

Instead, we propose 
a three-phased approach (\secref{sec:synthesis}; \figref{fig:process}). First, we
synthesize a concrete set of paths---i.e., a data plane---meeting 
all specified policies.
%, such as reachability, waypoint, and isolation.
%a partially-correct control plane from a set of example paths, such
%as those contained within a sample data plane. 
Second, we generate a simple control plane that,
under no failures,
induces the set of synthesized paths. This makes the synthesis 
tractable, as the second phase uses only the theory of linear 
rational arithmetic (LRA) to generate a control plane. 
Finally, we 
apply a series of
transformations
to obtain a control plane that  computes policy-compliant paths under a bounded set of failures (e.g.,
all single- and double-link failures).
If the procedure fails, we produce a different set of paths using the approach in the first phase 
and try again. 

%(7) A major challenge is abstracting away the low-level details of control
%plane configurations. We do this using an abstract representation.

Making this approach scalable and practical requires overcoming several
challenges. First, control plane configurations are low-level (e.g., on which
interfaces should a routing protocol operate and what costs should be assigned
to the incident links) while policies are high-level. Second, we must ensure
the transformations we apply to the synthesized control plane cause the
program to no longer conform to the input paths. We discuss these challenges
in more depth in \secref{sec:synthesis} and present preliminary solutions to
address them.

Automatically synthesizing distributed realizations of network policies is an
important step towards simplifying network management and reducing the
frequency and impact of network failures. Our approach is an important
contribution towards the vision of intent-driven networking~\cite{intent}.


%\aaron{Needs a closing, perhaps an outline, a summary of contributions, some
%results, and/or something visionary?}
